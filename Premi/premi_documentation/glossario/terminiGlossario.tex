\newglossaryentry{android}
{
name={Android},
description={Sistema operativo per dispositivi mobili sviluppato da Google Inc. sulla base del kernel Linux.}
}

\newglossaryentry{angularJS}
{
name={Angular.js},
description={Framework open-source Javascript mantenuto da Google. Viene utilizzato per creare componenti front-end.},
first={Angular.js},
firstplural={AngularJS},
text={angular}
}

\newglossaryentry{api}
{
name={API},
description={Acronimo di \textit{Application Programming Interface}, insieme di procedure disponibili al programmatore, di solito raggruppate a formare un set di strumenti specifici per l'espletamento di un determinato compito all'interno di un certo programma.},
first={Application Programming Interface (API)}
}

\newglossaryentry{applicationServer}
{
name={Application Server},
description={Tipologia di server che fornisce l'infrastruttura e le funzionalità di supporto, sviluppo ed esecuzione di applicazioni nonché altri componenti server in un contesto distribuito. Si tratta di un complesso di servizi orientati alla realizzazione di applicazioni ad architettura multilivello ed enterprise, con alto grado di complessità, spesso orientate al il web.}
}

\newglossaryentry{applogic}
{
name={Application logic},
description={Logica specifica di una applicazione. In particolare definisce come le componenti interagiscono tra loro e gestisce il flusso del lavoro.}
}

\newglossaryentry{architetturasw}
{
name={Architettura Software},
description={Organizzazione fondamentale di un sistema, definita dai suoi componenti, dalle relazioni reciproche tra i componenti e con l'ambiente, e i principi che ne governano la progettazione e l'evoluzione.}
}

\newglossaryentry{backend}
{
name={Back-end},
description={Parte di un sistema che elabora i dati generati dalla parte front-end senza interagire con l'utente. Spesso si trova in una macchina diversa da quella dove risiede la parte front-end del sistema.},
first={back-end},
text={backend}
}

\newglossaryentry{bannerpubblicitario}
{
name={Banner Pubblicitario},
description={Forma di pubblicità, molto diffusa su Internet, inserita sottoforma di annuncio all'interno di pagine web.},
text={banner}
}

\newglossaryentry{bdriven}
{
name={Behaviour driven},
description={Metodologia che prevedere la definizione di test basandosi sul comportamento dei moduli testati.}
}

\newglossaryentry{bigdata}
{
name={Big Data},
description={Raccolta di dataset così grande e complessa da richiedere strumenti differenti da quelli tradizionali, in tutte le fasi del processo: dall'acquisizione, alla curation, passando per condivisione, analisi e visualizzazione.}
}

\newglossaryentry{bitbucket}
{
name={Bitbucket},
description={Servizio di file hosting per progetti software che usano sistemi di controllo di versione distribuiti come Git.}
}

\newglossaryentry{bom}
{
name={BOM},
description={Acronimo di \textit{Byte Order Mark}, piccola sequenza di byte che viene posizionata all'inizio di un flusso di dati di puro testo (tipicamente un file) per indicarne il tipo di codifica Unicode.},
first={Byte Order Mark (BOM)}
}

\newglossaryentry{browser}
{
name={Browser},
description={Programma che consente di usufruire dei servizi di connettività in Internet, o di una rete di computer, e di navigare sul World Wide Web.},
first={browser web},
firstplural={web browser}
}

\newglossaryentry{bug}
{
name={Bug},
description={Errore in un programma software.},
text={baco}
}

\newglossaryentry{busilogic}
{
name={Business logic},
description={Parte dell'architettura di un programma che rappresenta il dominio applicativo dell'applicazione. Definisce il formato dei dati e le regole che specificano come i dati si combinano tra loro.}
}

\newglossaryentry{calendar}
{
name={Google Calendar},
description={Sistema di calendari concepito da Google, agenda sulla quale inserire degli eventi. L'utilizzo può essere come agenda personale (calendario privato), come agenda di un'organizzazione (calendario pubblico) o come agenda di una risorsa (ad esempio un'aula o un campo da tennis).},
text={calendar},
plural={calendars}
}

\newglossaryentry{callback}
{
name={Callback},
description={Funzione o blocco di codice che viene passato come parametro ad un altra funzione.
In particolare, quando ci si riferisce alla callback richiamata da una funzione, la callback viene passata come parametro alla funzione chiamante. In questo modo la chiamante può realizzare un compito specifico (quello svolto dalla callback) che non è, molto spesso, noto al momento della scrittura del codice.},
first={callback},
firstplural={callbacks},
text={Callback},
plural={Callbacks}
}

\newglossaryentry{chat}
{
name={Chat},
description={In questo progetto, il termine è usato come sinonimo di \textit{Instant Messaging}.}
}

\newglossaryentry{chrome}
{
name={Google Chrome},
description={Browser basato su WebKit sviluppato da Google.},
text={chrome}
}

\newglossaryentry{client}
{
name={Client},
description={Componente che accede ai servizi o alle risorse di un'altra componente detta server.}
}

\newglossaryentry{cloudcomputing}
{
name={Cloud Computing},
description={Insieme di tecnologie che permettono, tipicamente sotto forma di un servizio offerto da un provider al cliente, di memorizzare/archiviare e/o elaborare dati (tramite CPU o software) grazie all'utilizzo di risorse hardware/software distribuite e virtualizzate in Rete in un'architettura tipica client-server.},
text={cloud}
}

\newglossaryentry{cmm}
{
name={CMM},
description={Acronimo di \textit{Capability Maturity Model}, approccio al miglioramento dei processi il cui obiettivo è di aiutare un'organizzazione a migliorare le sue prestazioni.},
first={Capability Maturity Model (CMM)}
}

\newglossaryentry{cnvs}
{
name={Canvas},
description={\begin{enumerate}
\item \textit{Area di lavoro}: area dell’applicazione dove l’utente può disegnare la mappa mentale;
\item \textit{Tag HTML5}: elemento del linguaggio HTML5 sul quale è possibile disegnare mediante linguaggi di scripting come JavaScript.
\end{enumerate}
}
}

\newglossaryentry{committente}
{
name={Committente},
description={Individuo che assegna un progetto ad un individuo/gruppo/azienda. In questo progetto, questo ruolo è ricoperto dai professori \textit{Tullio Vardanega} e \textit{Riccardo Cardin}.}
}

\newglossaryentry{compilatore}
{
name={Compilatore},
description={Programma che traduce una serie di istruzioni scritte in uno specifico linguaggio di programmazione (codice sorgente) in istruzioni di un altro linguaggio (codice oggetto).},
first={compilatore},
text={Compilatore}
}

\newglossaryentry{complessitaciclomatica}
{
name={Complessità Ciclomatica},
description={Metrica software sviluppata da Thomas J. McCabe nel 1976 ed utilizzata per misurare la complessità di un programma. Misura direttamente il numero di cammini linearmente indipendenti attraverso il grafo di controllo di flusso.}
}

\newglossaryentry{console}
{
name={Console},
description={Tipologia di interfaccia utente caratterizzata da un'interazione di tipo testuale tra utente ed elaboratore: l'utente impartisce comandi testuali in input mediante tastiera alfanumerica e riceve risposte testuali in output dall'elaboratore mediante display o stampante alfanumerici.}
}

\newglossaryentry{controller}
{
name={Controller},
description={Parte del pattern architetturale MVC che si occupa di aggiornare lo stato del Model e della View. Aggiorna la visione che la View ha del Model secondo le modifiche attuate in quest'ultimo.},
first={controller},
text={Controller}
}

\newglossaryentry{crossplatform}
{
name={Cross-platform},
description={Tradotto in italiano con \textit{Multipiattaforma}, viene riferito ad un linguaggio di programmazione, ad un'applicazione software o ad un dispositivo hardware che funziona su più di un sistema o appunto, piattaforma (es. Unix/Linux, Windows e Macintosh).},
text={multipiattaforma}
}

\newglossaryentry{css}
{
name={CSS},
description={Acronimo di \textit{Cascading Style Sheets}, linguaggio usato per definire la formattazione di documenti HTML e XML, ad esempio di siti web e relative pagine web.},
first={Cascading Style Sheets (CSS)}
}

\newglossaryentry{csslint}
{
name={CSSLint},
description={Software open-source che permette di effettuare analisi statica di codice CSS.}
}

\newglossaryentry{debugging}
{
name={Debugging},
description={Attività che consiste nell'individuazione da parte del programmatore della porzione di software affetta da errore (\textit{bug}) rilevata nei software a seguito dell'utilizzo del programma.}
}

\newglossaryentry{designpattern}
{
name={Design Pattern},
description={Soluzione progettuale generale ad un problema ricorrente, ossia descrizione o modello logico da applicare per la risoluzione di un problema che può presentarsi in diverse situazioni durante le fasi di progettazione e sviluppo del software, ancor prima della definizione dell'algoritmo risolutivo della parte computazionale.}
}

\newglossaryentry{designresponsivo}
{
name={Design Responsivo},
description={Tecnica di web design per la realizzazione di siti in grado di adattarsi graficamente in modo automatico al dispositivo coi quali vengono visualizzati (computer con diverse risoluzioni, tablet, smartphone, cellulari, web tv), riducendo al minimo la necessità per l'utente di ridimensionamento e scorrimento dei contenuti.}
}

\newglossaryentry{diagrammagantt}
{
name={Diagramma di Gantt},
description={Strumento di supporto alla gestione dei progetti; costruito partendo da un asse orizzontale - a rappresentazione dell'arco temporale totale del progetto, suddiviso in fasi incrementali (ad esempio, giorni, settimane, mesi) - e da un asse verticale - a rappresentazione delle mansioni o attività che costituiscono il progetto.},
plural={diagrammi di gantt}
}

\newglossaryentry{docs}
{
name={Google Docs},
description={Software che permette creazione e modifica di documenti di testo, fogli di calcolo e presentazioni all'interno del servizio Google Drive.},
text={gdoc}
}

\newglossaryentry{drive}
{
name={Google Drive},
description={Servizio web di storage e sincronizzazione online introdotto da Google il 24 aprile 2012 che permette il file hosting, file sharing e editing collaborativo di documenti.},
text={drive}
}

\newglossaryentry{dvcs}
{
name={DVCS},
description={Acronimo di \textit{Distributed Version Control System}, ossia \textit{Sistema di Controllo di Versione Distribuito}, sistema che tiene traccia delle versioni del software e permette a molti sviluppatori di lavorare su un dato progetto senza necessariamente essere connessi ad una rete comune.},
first={Distributed Version Control System (DVCS)},
text={controllo di versione distribuito}
}

\newglossaryentry{ecmascript}
{
name={ECMAScript},
description={Standard di riferimento per il linguaggio JavaScript; l'ultimo standard, ECMA-262 Edition 5.1 (marzo 2011), è anche uno standard ISO.}
}

\newglossaryentry{eventdriven}
{
name={Event-Driven},
description={Paradigma di programmazione in cui il flusso del programma è largamente determinato dal verificarsi di eventi esterni, ossia non segue percorsi fissi (che si ramificano soltanto in punti ben determinati predefiniti dal programmatore).}
}

\newglossaryentry{facade}
{
name={Facade},
description={Pattern strutturale che ha lo scopo di fornire un'interfaccia semplice per utilizzare un sottosistema complesso formato da diversi moduli.},
first={facade},
text={Facade}
}

\newglossaryentry{faulttolerance}
{
name={Fault-Tolerance},
description={Capacità di un sistema di non subire avarie (cioè interruzioni di servizio) anche in presenza di errori o guasti.}
}

\newglossaryentry{firefox}
{
name={Mozilla Firefox},
description={Web browser open source multipiattaforma prodotto da Mozilla Foundation.},
text={firefox}
}

\newglossaryentry{fornitore}
{
name={Fornitore},
description={Individuo/azienda che ha il compito di produrre e consegnare il progetto assegnato dal committente. In questo progetto, questo ruolo è ricoperto dal gruppo \textit{Pragma}.}
}

\newglossaryentry{frame}
{
name={Frame},
description={Nodo della mappa mentale visto come contenitore di elementi grafici quali: testo, immagini e video. Lo stesso frame può essere presente in più percorsi di presentazione e può comparire più volte in uno stesso percorso.}
}

\newglossaryentry{framework}
{
name={Framework},
description={Architettura logica di supporto (spesso un'implementazione logica di un particolare design pattern) su cui un software può essere progettato e realizzato facilitandone lo sviluppo da parte del programmatore.}
}

\newglossaryentry{frontend}
{
name={Front-end},
description={Parte di un sistema software che gestisce l'interazione con l'utente o con sistemi esterni che producono dati in ingresso.},
first={front-end},
text={front end}
}

\newglossaryentry{git}
{
name={Git},
description={Sistema software di controllo di versione distribuito, creato da Linus Torvalds nel 2005.}
}

\newglossaryentry{gmail}
{
name={GMail},
description={Servizio gratuito di posta elettronica offerto da Google.}
}

\newglossaryentry{gplus}
{
name={Google+},
description={Rete sociale gratuita creata da Google Inc. nel 2011.},
text={plus}
}

\newglossaryentry{gui}
{
name={GUI},
description={Acronimo di \textit{Graphical User Interface}, tipo di interfaccia utente che consente all'utente di interagire con la macchina controllando oggetti grafici convenzionali.},
first={Graphical User Interface (GUI)}
}

\newglossaryentry{gulpease}
{
name={Indice Gulpease},
description={Indice di leggibilità di un testo tarato sulla lingua italiana; ha il vantaggio di utilizzare la lunghezza delle parole in lettere anziché in sillabe, semplificandone il calcolo automatico.}
}

\newglossaryentry{hangouts}
{
name={Google Hangouts},
description={Software di messaggistica istantanea e di VoIP sviluppato da Google; è disponibile per le piattaforme mobili Android e iOS e come estensione per il browser web Google Chrome. Inoltre è possibile sfruttare i servizi offerti da Hangouts anche all'interno dalla web mail Gmail o dal proprio profilo sul social network di Google, Google+.},
text={hangout},
plural={hangouts}
}

\newglossaryentry{hardware}
{
name={Hardware},
description={Parte fisica di un computer, ovvero tutte quelle parti elettroniche, elettriche, meccaniche, magnetiche, ottiche che ne consentono il funzionamento.},
text={hw}
}

\newglossaryentry{hashing}
{
name={Hashing},
description={La funzione hash è una funzione non iniettiva (e quindi non invertibile) che mappa una stringa di lunghezza arbitraria in una stringa di lunghezza predefinita. L'algoritmo di hash elabora qualunque mole di bit.
 \begin{enumerate}
\item L'algoritmo restituisce una stringa di numeri e lettere a partire da un qualsiasi flusso di bit di qualsiasi dimensione (può essere un file ma anche una stringa). L'output è detto digest;
\item La stringa di output è univoca per ogni documento e ne è un identificatore;
\item L'algoritmo non è invertibile, ossia non è possibile ricostruire il documento originale a partire dalla stringa che viene restituita in output ovvero è una funzione unidirezionale.
 \end{enumerate}},
first={hash},
text={hashing},
plural={hash}
}

\newglossaryentry{highOrder}
{
name={High-order},
description={Significa che una funzione che può prendere altre funzioni come parametri e/o restituire funzioni come risultato. L'operatore differenziale in matematica è un esempio di funzione high-order.},
first={high-order},
text={high order},
plural={High Order}
}

\newglossaryentry{host}
{
name={Host},
description={Terminale collegato, attraverso link di comunicazione, ad una rete informatica (es. Internet).}
}

\newglossaryentry{hosting}
{
name={Hosting},
description={Servizio di archiviazione su Internet appositamente progettato per ospitare i file degli utenti, permettendo loro di caricare file che possono poi essere scaricati da altri utenti.}
}

\newglossaryentry{html}
{
name={HTML},
description={Acronimo di \textit{HyperText Markup Language}, linguaggio di markup solitamente usato per la formattazione di documenti ipertestuali disponibili nel World Wide Web sotto forma di pagine web.},
first={HyperText Markup Language (HTML)}
}

\newglossaryentry{htmlcinque}
{
name={HTML5},
description={Linguaggio di markup per la strutturazione delle pagine web, da ottobre 2014 pubblicato come W3C Recommendation. \`{E} stato sviluppato con lo scopo di migliorare il disaccoppiamento fra struttura, definita dal markup, caratteristiche di resa (tipo di carattere, colori, eccetera), definite dalle direttive di stile, e contenuti di una pagina web, definiti dal testo vero e proprio.}
}

\newglossaryentry{iaas}
{
name={IaaS},
description={Acronimo di \textit{Infrastructure as a Service}, categoria di servizio di cloud computing che fornisce, come servizio, l'accesso a una un'infrastruttura di elaborazione: spazio virtuale su server, connessioni di rete, larghezza di banda, indirizzi IP e bilanciatori di carico.},
first={Infrastructure as a Service (IaaS)}
}

\newglossaryentry{ide}
{
name={IDE},
description={Acronimo di \textit{Integrated Development Environment}, software che, in fase di programmazione, aiuta i programmatori nello sviluppo del codice sorgente di un programma segnalando errori di sintassi del codice direttamente in fase di scrittura e fornendo una serie di strumenti e funzionalità di supporto alla fase di sviluppo e debugging.},
first={Integrated Development Environment (IDE)}
}

\newglossaryentry{infografica}
{
name={Infografica},
description={Informazione proiettata in forma più grafica e visuale che testuale; alcuni esempi sono: tabelle, diagrammi di flusso, mappe concettuali, schemi a blocco, istogrammi, grafici, mappe, schemi.}
}

\newglossaryentry{inspection}
{
name={Inspection},
description={Tecnica di analisi statica che ha come obiettivo il rilevamento della presenza di difetti eseguendo una lettura mirata di un documento o del codice di un programma.}
}

\newglossaryentry{instantmessaging}
{
name={Instant Messaging},
description={Tradotto in italiano con \textit{Messaggistica Istantanea}, categoria di sistemi di comunicazione in tempo reale in rete, tipicamente Internet o una rete locale, che permette ai suoi utilizzatori lo scambio di brevi messaggi.},
text={messaggistica istantanea}
}

\newglossaryentry{integrazionecontinua}
{
name={Integrazione Continua},
description={Pratica che si applica in contesti in cui lo sviluppo del software avviene attraverso un sistema di versioning; consiste nell'allineamento frequente dagli ambienti di lavoro degli sviluppatori verso l'ambiente condiviso (mainline).},
text={continuous integration}
}

\newglossaryentry{ios}
{
name={iOS},
description={Sistema operativo sviluppato da Apple per i suoi dispositivi mobili.}
}

\newglossaryentry{iso}
{
name={ISO},
description={Acronimo di \textit{International Organization for Standardization}, indica la più importante organizzazione a livello mondiale per la definizione di norme tecniche.}
}

\newglossaryentry{javascript}
{
name={JavaScript},
description={Linguaggio di scripting orientato agli oggetti e agli eventi, comunemente utilizzato nella programmazione Web lato client per la creazione, in siti web e applicazioni web, di effetti dinamici interattivi tramite funzioni di script invocate da eventi innescati a loro volta in vari modi dall'utente sulla pagina web in uso (mouse, tastiera ecc...).}
}

\newglossaryentry{jqlite}
{
name={jqLite},
description={Versione ridotta della libreria jQuery presente di default in AngularJS, permette di effettuare le modifiche più comuni ad oggetti del DOM. Maggiori informazioni disponibili su \url{https://docs.angularjs.org/api/ng/function/angular.element}.}
}

\newglossaryentry{jshint}
{
name={JSHint},
description={Software di analisi statica che permette di verificare se codice JavaScript rispetta determinate norme di codifica.}
}

\newglossaryentry{json}
{
name={JSON},
description={Formato aperto standard che usa testo leggibile dall'uomo per trasmettere oggetti dati, costituiti da coppie attributo-valore. Viene usato principalmente per la trasmissione di dati tra sever e applicazioni web. },
first={JavaScript Object Notation}
}

\newglossaryentry{karma}
{
name={Karma},
description={Framework per Node.js utilizzato per effettuare analisi dinamica su codice Javascript.}
}

\newglossaryentry{kernel}
{
name={Kernel},
description={Costituisce il nucleo di un sistema operativo. Si tratta di un software avente il compito di fornire ai processi in esecuzione sull'elaboratore un accesso sicuro e controllato all'hardware.}
}

\newglossaryentry{latenza}
{
name={Latenza},
description={Intervallo di tempo che intercorre fra il momento in cui arriva l'input al sistema ed il momento in cui è disponibile il suo output. In altre parole, misura della velocità di risposta di un sistema.}
}

\newglossaryentry{libreria}
{
name={Libreria},
description={Insieme di funzioni o strutture dati predefinite e predisposte per essere collegate ad un programma software attraverso opportuno collegamento.},
plural={librerie}
}

\newglossaryentry{linux}
{
name={Linux},
description={Kernel distribuito con licenza GNU General Public License creato nel 1991 da Linus Torvalds.}
}

\newglossaryentry{mailinglist}
{
name={Mailing List},
description={Servizio/strumento offribile da una rete di computer verso vari utenti e costituito da un sistema organizzato per la partecipazione di più persone ad una discussione o per la distribuzione di informazioni utili agli interessati/iscritti attraverso l'invio di email ad una lista di indirizzi di posta elettronica di utenti iscritti.},
plural={mailing lists}
}

\newglossaryentry{mappam}
{
name={Mappa mentale},
description={Rappresentazione schematizzata di un tema o argomento. Consiste in un albero che ha come radice l’argomento della mappa mentale e come figli le varie idee ad essa correlate. A queste idee possono essere correlate ulteriori idee di secondo livello e così via. Su questa struttura è inoltre possibile definire delle associazioni tra i vari nodi della mappa, aumentandone l’espressività e evidenziando la presenza di legami trasversali tra i vari elementi.}
}

\newglossaryentry{mdesign}
{
name={Material Design},
description={Insieme di linee guida proposte da Google per la definizione di interfacce grafiche, sia desktop sia mobile, in modo da fornire all'utente un'esperienza d'uso simile su piattaforme diverse.}
}

\newglossaryentry{mean}
{
name={MEAN},
description={Solution stack open-source basato completamente su tecnologia JavaScript, per lo sviluppo di applicazioni web usando MongoDB, Node.js, Express e AngularJS.}
}

\newglossaryentry{milestone}
{
name={Milestone},
description={Importante traguardo intermedio nello svolgimento di un progetto; indica il raggiungimento di un obiettivo stabilito in fase di definizione del progetto stesso.}
}

\newglossaryentry{mongodb}
{
name={MongoDB},
description={Database non relazionale orientato ai documenti di tipo NoSQL. MongoDB si allontana dalla struttura tradizionale basata su tabelle dei database relazionali utilizzando documenti in un formato ispirato allo stile JSON con schema dinamico (denominato BSON).
}
}

\newglossaryentry{mongoose}
{
name={Mongoose},
description={Libreria Javascript per Node.js per l’Object Data Mapping (ODM) che consente di definire schemi con i quali creare e modificare documenti nei database MongoDB. Mongoose consente di trattare gli schemi realizzati come classi e quindi di sfruttare conversioni di tipo, metodi di istanza e metodi statici.}
}

\newglossaryentry{monospace}
{
name={Monospace},
description={Carattere Unicode monospazio, creato da George Williams. Presenta, oltre alla sua versione roman, anche l'italico e il grassetto. Lo stile ricorda quello delle macchine da scrivere.}
}

\newglossaryentry{mvc}
{
name={MVC},
description={Acronimo di \textit{Model-View-Controller}, pattern architetturale software per l'implementazione di interfacce utente. Divide un'applicazione software in tre parti interconnesse, in modo da separare la rappresentazione interna delle informazioni dal modo in cui tali informazioni vengono presentate all'utente.},
first={Model-View-Controller (MVC)},
text={model-view-controller}
}

\newglossaryentry{mysql}
{
name={MySQL},
description={Relational database management system (\textit{RDBMS}) composto da un client a riga di comando e un server.}
}

\newglossaryentry{nodejs}
{
name={Node.js},
description={Framework event-driven per il motore JavaScript V8, su piattaforme UNIX like, relativo all'utilizzo server-side di Javascript.}
}

\newglossaryentry{objectid}
{
name={ObjectId},
description={Tipo BSON, utilizzato da MongoDB, che consente di ottenere valori alfanumerici univoci. In particolare, MongoDB aggiunge un campo \texttt{\_id} di tipo ObjectId come chiave primaria nelle collezioni, qualora l'utilizzatore non ne specifichi una.}
}

\newglossaryentry{observer}
{
name={Observer},
description={Pattern comportamentale nel quale uno o più soggetti (Subject) mantengono una lista di osservatori (Observer) e li notificano qualora lo stato del soggetto sia modificato ed interessi l'osservatore registrato alla lista di quel particolare evento nel soggetto.},
first={observer},
text={Observer}
}

\newglossaryentry{paas}
{
name={PaaS},
description={Acronimo di \textit{Platform as a Service}, distribuzione di piattaforme di elaborazione (Computing platform) e di solution stack come servizio.}
}

\newglossaryentry{parser}
{
name={Parser},
description={Strumento che si occupa di fare il parsing},
first={parser},
text={Parser}
}

\newglossaryentry{parsing}
{
name={Parsing},
description={Processo di analisi sintattica atto ad analizzare uno stream continuo in input in modo da determinarne la sua struttura grammaticale. Ciò è possibile avendo definito una grammatica formale.},
first={parsing},
text={Parsing}
}

\newglossaryentry{pdf}
{
name={PDF},
description={Acronimo di \textit{Portable Document Format}, formato di file basato su un linguaggio di descrizione di pagina sviluppato da Adobe Systems nel 1993 per rappresentare documenti in modo indipendente dall'hardware e dal software utilizzati per generarli o per visualizzarli.},
first={Portable Document Format (PDF)}
}

\newglossaryentry{pdp}
{
name={Percorso di presentazione},
description={Ordine nel quale vengono presentati alcuni frame della mappa mentale. Un progetto può contenere più percorsi di presentazione. In un progetto esiste sempre il percorso di presentazione di default che contiene tutti i frame ordinati in base all’ordine di creazione.},
plural={percorsi di presentazione}
}

\newglossaryentry{pdv}
{
name={Percorso di visualizzazione},
description={Sinonimo di percorso di presentazione.},
plural={percorsi di visualizzazione}
}

\newglossaryentry{percorso}
{
name={Percorso},
description={\begin{enumerate}
\item Abbreviazione di percorso di presentazione;
\item Posizione logica del file all’interno di un filesystem.
\end{enumerate}},
plural={percorsi}
}

\newglossaryentry{php}
{
name={PHP},
description={Acronimo ricorsivo di \textit{PHP: Hypertext Preprocessor}, linguaggio di programmazione interpretato, originariamente concepito per la programmazione di pagine web dinamiche e attualmente utilizzato anche per sviluppare applicazioni web lato server, scrivere script a riga di comando o applicazioni con interfaccia grafica.}
}

\newglossaryentry{piattaforma}
{
name={Piattaforma},
description={Base software e/o hardware su cui sono sviluppate e/o eseguite applicazioni.}
}

\newglossaryentry{plugin}
{
name={Plugin},
description={Programma non autonomo che interagisce con un altro programma per ampliarne le funzioni.},
text={plug-in}
}

\newglossaryentry{png}
{
name={PNG},
description={Acronimo di \textit{Portable Network Graphics}, formato di file per memorizzare immagini capace di immagazzinare immagini in modo \textit{lossless}, ossia senza perdere alcuna informazione.},
first={Portable Network Graphics (PNG)}
}

\newglossaryentry{premiproj}
{
name={Progetto},
description={\begin{enumerate}
\item \textit{Progetto Premi}: mappa mentale realizzata con Premi sulla quale è possibile definire dei percorsi di presentazione sfruttando il contenuto dei nodi della mappa mentale. La presentazione potrà essere eseguita in modo lineare, seguendo l'ordine prestabilito, oppure non lineare, visitando liberamente i nodi della mappa. Quando viene creato un progetto, questo contiene il nodo radice della mappa mentale e un percorso di presentazione di default.
\item \textit{Progetto software}: insieme di attività organizzate sotto vincoli di efficacia ed efficienza, che consentono di passare tra i vari stati del ciclo di vita del software
\end{enumerate}
},
plural={progetti}
}

\newglossaryentry{presenter}
{
name={Presenter},
description={Parte del pattern architetturale MVP incorpora la business logic della View. Aggiorna la View conseguentemente a modifiche del Model.},
first={presenter},
text={Presenter}
}

\newglossaryentry{profiling}
{
name={Profiling Software},
description={Forma di analisi dinamica del software che misura parametri quali complessità di un programma (sia in termini di memoria che di tempo), uso di particolari istruzioni, frequenza e durata di chiamate a funzioni. Tale analisi è di grande aiuto nell'ottimizzazione di programmi.}
}

\newglossaryentry{proponente}
{
name={Proponente},
description={Individuo che ha proposto un capitolato d’appalto. In questo progetto, questo ruolo è ricoperto dall'azienda \textit{Zucchetti S.p.A.}.}
}

\newglossaryentry{protocollodirete}
{
name={Protocollo di Rete},
description={Definizione formale a priori delle modalità di interazione che due o più apparecchiature elettroniche collegate tra loro devono rispettare per operare particolari funzionalità di elaborazione necessarie all'espletamento di un certo servizio di rete.}
}

\newglossaryentry{rdbms}
{
name={RDBMS},
description={Sistema software basato sul modello relazionale progettato per consentire la creazione e la manipolazione (da parte di un amministratore) e l'interrogazione efficiente (da parte di uno o più utenti) di \textit{database} (ovvero di collezioni di dati strutturati).}
}

\newglossaryentry{realtime}
{
name={Real-Time},
description={Programma per il quale la correttezza del risultato dipende dal tempo di risposta; vi è, quindi, la necessità che risponda ad eventi esterni entro tempi prestabiliti.}
}

\newglossaryentry{rendering}
{
name={Rendering},
description={Generazione di un'immagine a partire da una descrizione matematica di una scena tridimensionale interpretata da algoritmi che definiscono il colore di ogni punto dell'immagine digitale.}
}

\newglossaryentry{repo}
{
name={Repository},
description={Ambiente di un sistema informativo (ad es. di tipo ERP), in cui vengono gestiti i metadati, attraverso tabelle relazionali.},
text={repo}
}

\newglossaryentry{rest}
{
name={REST},
description={Acronimo di \textit{REpresentational State Transfer}, architettura software per i sistemi di ipertesto distribuiti come il World Wide Web; stile architetturale costituito da un insieme coordinato di vincoli architetturali applicati a componenti, connettori ed elementi dati in un sistema ipermediale distribuito. REST ignora i dettagli della sintassi di componenenti di implementazione e protocolli in modo da focalizzarsi sul ruolo di tali componenti, sui vincoli di interazione con altri componenti e sull'interpretazione di elementi dati importanti. 
REST basa su alcuni principi che delineano come le risorse sono definite e indirizzate:
\begin{itemize}
\item \textit{identificazione univoca delle risorse}: ad esempio, nel web devono essere identificate univocamente con un URI;
\item \textit{utilizzo esplicito dei metodi HTTP};
\item \textit{risorse autodescrittive}: è possibile utilizzare virtualmente qualsiasi formato per rappresentare le risorse, ma è opportuno utilizzare formati il più possibile standard in modo da semplificare l'interazione con i client;
\item \textit{collegamenti tra risorse}: una risorsa deve fornire tutte le informazioni riguardo alle risorse ad essa correlate nella sua rappresentazione o mediante collegamenti ipertestuali;
\item \textit{comunicazione stateless}:  nessuna richiesta deve avere relazioni con le richieste precedenti e successive ad essa.
\end{itemize}}
}

\newglossaryentry{restlike}
{
name={REST-like},
description={Architettura software in stile REST, che discosta da questa non vincolandosi all'aderenza di tutti i principi definiti per REST.}
}

\newglossaryentry{runtime}
{
name={Runtime},
description={Momento in cui un programma per computer viene eseguito. Tale termine viene anche indicato per esprimere contrapposizione rispetto a \textit{Compile-time}, ossia tempo di compilazione.}
}

\newglossaryentry{saas}
{
name={SaaS},
description={Acronimo di \textit{Software as a Service}, modello di distribuzione del software applicativo dove un produttore di software sviluppa, opera (direttamente o tramite terze parti) e gestisce un'applicazione web che mette a disposizione dei propri clienti via internet.}
}

\newglossaryentry{salt}
{
name={Salt},
description={\`E un valore molto piccolo, generato casualmente, che permette di rendere l'hashed-value comune di una password un hashed-value più generico. L'intenzione è di ridurre la probabilità che l'hashed value sia scoperto tramite hash-table pre calcolate, questo riduce la probabilità che password molto comuni vengano scoperte. Il valore salt viene applicato semplicemente concatenandolo al valore della password prima che venga generato l'hashed-value.},
first={Salt},
text={salt},
plural={salted}
}

\newglossaryentry{sass}
{
name={Sass},
description={Estensione del linguaggio CSS che permette di utilizzare variabili, di creare funzioni e di organizzare il fogli di stile in più file.}
}

\newglossaryentry{scanner}
{
name={Scanner},
description={Analizzatore lessicale che esamina il codice sorgente in input per individuarne i simboli che lo compongono (token) classificando parole chiave, operatori,identificatori, costanti.},
first={scanner},
text={Scanner}
}

\newglossaryentry{scorecard}
{
name={Scorecard},
description={La scheda di valutazione bilanciata (in inglese balanced scorecard) è uno strumento di supporto nella gestione strategica dell'impresa che permette di tradurre la missione e la strategia dell'impresa in un insieme coerente di misure di performance, facilitandone la misurabilità.},
first={scorecard},
text={Scorecard}
}

\newglossaryentry{sdk}
{
name={SDK},
description={Acronimo di \textit{Software Development Kit}, insieme di strumenti per lo sviluppo e la documentazione di software.},
first={Software Development Kit (SDK)}
}

\newglossaryentry{server}
{
name={Server},
description={Componente o sottosistema informatico di elaborazione che fornisce, a livello logico e a livello fisico, un qualunque tipo di servizio ad altre componenti (client) che ne fanno richiesta attraverso una rete di computer, all'interno di un sistema informatico o direttamente in locale su un computer.}
}

\newglossaryentry{sharing}
{
name={Sharing},
description={Condivisione di file all'interno di una rete di calcolatori.}
}

\newglossaryentry{sistemaoperativo}
{
name={Sistema Operativo},
description={Insieme di componenti software, che consente l'utilizzo di varie apparecchiature informatiche (ad esempio di un computer) da parte di un utente.}
}

\newglossaryentry{slack}
{
name={Slack},
description={Quantità di tempo che un’attività progettuale può essere ritardata senza causare
dei ritardi alle altre attività.}
}

\newglossaryentry{solutionstack}
{
name={Solution Stack},
description={Insieme di sottosistemi o componenti software necessari a creare una piattaforma completa tale da non richiedere software aggiuntivo per supportare un'applicazione.},
text={stack}
}

\newglossaryentry{ssh}
{
name={SSH},
description={Acronimo di \textit{Secure SHell}, protocollo di rete che permette di stabilire una sessione remota cifrata tramite interfaccia a riga di comando con un altro host di una rete informatica.},
first={Secure Shell (SSH)}
}

\newglossaryentry{stack}
{
name={Stack},
description={Tradotto in italiano con \textit{Pila}, tipo di dato astratto che viene usato in diversi contesti per riferirsi a strutture dati, le cui modalità d'accesso ai dati in essa contenuti seguono una modalità \textit{LIFO} (Last In First Out), ovvero tale per cui i dati vengono estratti (letti) in ordine rigorosamente inverso rispetto a quello in cui sono stati inseriti (scritti).},
text={pila}
}

\newglossaryentry{stakeholder}
{
name={Stakeholder},
description={Soggetto (o un gruppo di soggetti) influente nei confronti di un'iniziativa economica, sia essa un'azienda o un progetto; fanno, ad esempio, parte di questo insieme: i clienti, i fornitori, i finanziatori (banche e azionisti), i collaboratori, ma anche gruppi di interesse esterni, come i residenti di aree limitrofe all'azienda o gruppi di interesse locali.},
plural={stakeholders}
}

\newglossaryentry{standardaperto}
{
name={Standard Aperto},
description={Standard disponibile al pubblico con diversi diritti ad esso associati e con diverse proprietà con cui è stato progettato.},
text={open standard}
}

\newglossaryentry{storage}
{
name={Storage},
description={Dispositivi hardware, supporti per la memorizzazione, infrastrutture e software dedicati alla memorizzazione non volatile di grandi quantità di informazioni in formato elettronico.}
}

\newglossaryentry{storytelling}
{
name={Storytelling},
description={Metodologia che usa la narrazione come mezzo creato dalla mente per inquadrare gli eventi della realtà e spiegarli secondo una logica di senso.}
}

\newglossaryentry{stub}
{
name={Stub},
description={Porzione di codice utilizzata in sostituzione di altre funzionalità software. Esso simula temporaneamente il comportamento di codice esistente o che deve essere sviluppato. In particolare può simulare componenti che devono ancora essere integrate al sistema.}
}

\newglossaryentry{svg}
{
name={SVG},
description={Acronimo di \textit{Scalable Vector Graphics}, tecnologia in grado di visualizzare oggetti di grafica vettoriale e, pertanto, di gestire immagini scalabili dimensionalmente.},
first={Scalable Vector Graphics (SVG)}
}

\newglossaryentry{team}
{
name={Team},
description={Insieme di persone che collaborano tra di loro e che hanno lo stesso scopo, in genere lavorativo, scientifico, culturale o sportivo. In questo progetto, questo termine indica i componenti del gruppo \textit{Pragma}.}
}

\newglossaryentry{template}
{
name={Template},
description={In questo progetto, il termine indica uno script \LaTeX contenente le impostazioni stilistiche e i comandi comuni a tutti i documenti.}
}

\newglossaryentry{ticketing}
{
name={Ticketing},
description={Conosciuto anche come \textit{Issue Tracking System}, sistema informatico che fornisce strumenti per la creazione, la modifica e la cancellazione di segnalazioni (ticket).},
text={issue tracking}
}

\newglossaryentry{uml}
{
name={UML},
description={Acronimo di \textit{Unified Modeling Language}, linguaggio di modellazione e specifica basato sul paradigma object-oriented.},
first={Unified Modeling Language (UML)}
}

\newglossaryentry{unicode}
{
name={Unicode},
description={Sistema di codifica che assegna un numero univoco ad ogni carattere usato per la scrittura di testi, in maniera indipendente dalla lingua, dalla piattaforma informatica e dal programma utilizzato.}
}

\newglossaryentry{utfotto}
{
name={UTF-8},
description={Acronimo di \textit{Unicode Transformation Format - 8 bit}, codifica dei caratteri Unicode in sequenze di lunghezza variabile di byte.},
text={utf}
}

\newglossaryentry{versionamento}
{
name={Versionamento},
description={In questo progetto, il termine è usato come sinonimo di \textit{DVCS}.},
text={controllo di versione}
}

\newglossaryentry{view}
{
name={View},
description={Parte del pattern architetturale MVC che si interfaccia con l'utente.},
first={view},
text={View}
}

\newglossaryentry{viewport}
{
name={Viewport},
description={Nei web browser, indica la porzione visibile di un documento; se quest'ultimo è più largo, l'utente può traslare il viewport usando le barre laterali.}
}

\newglossaryentry{walkthrough}
{
name={Walkthrough},
description={Tecnica di analisi statica che ha come obiettivo quello di rilevare la presenza di difetti eseguendo una lettura critica a largo spettro di un documento o del codice di un programma senza assumere particolari presupposti.}
}

\newglossaryentry{webkit}
{
name={WebKit},
description={Motore di rendering per browser web utilizzato per il rendering delle pagine web.}
}

\newglossaryentry{websocket}
{
name={WebSocket},
description={Tecnologia web che fornisce canali di comunicazione full-duplex attraverso una singola connessione TCP; disegnato per essere implementato sia lato browser che lato server, ma può essere utilizzato anche da qualsiasi applicazione client-server.}
}

\newglossaryentry{webstorage}
{
name={Webstorage},
description={Servizio web di storage.}
}

\newglossaryentry{webstorm}
{
name={WebStorm},
description={IDE commerciale per JavaScript, CSS e HTML personalizzato con plugin JavaScript pre-installati (ad esempio Node.js).}
}

\newglossaryentry{wtrec}
{
name={W3C},
description={Organizzazione non governativa internazionale che ha come scopo quello di sviluppare tutte le potenzialità del World Wide Web; la principale attività svolta dal W3C consiste nello stabilire standard tecnici per il World Wide Web.}
}

\newglossaryentry{wtrecrecommendation}
{
name={W3C Recommendation},
description={Riconoscimento di uno standard da parte di W3C; indica che tale standard è, secondo W3C, pronto per essere utilizzato nello sviluppo di altri prodotti.}
}

\newglossaryentry{www}
{
name={World Wide Web},
description={Uno dei principali servizi di Internet che permette di navigare e usufruire di un insieme vastissimo di contenuti (multimediali e non) collegati tra loro e di ulteriori servizi accessibili a tutti o ad una parte selezionata degli utenti di Internet.},
text={web}
}

\newglossaryentry{xml}
{
name={XML},
description={Acronimo di \textit{eXtensible Markup Language}, linguaggio di markup, ovvero linguaggio marcatore basato su un meccanismo sintattico che consente di definire e controllare il significato degli elementi contenuti in un documento o in un testo.},
first={Extensible Markup Language (XML)}
}