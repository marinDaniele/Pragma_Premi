\subsection{Struttura dei verbali}
\paragraph{Problema} I verbali esterni devono evidenziare in modo tracciabile le decisioni prese dal gruppo riguardo ai problemi discussi durante le riunioni.
\paragraph{Decisione} I verbali verranno denominati con un codice identificativo del tipo TN, dove:
\begin{itemize}
\item T rappresenta il tipo del verbale, E per i verbali esterni ed I per quelli interni;
\item N è un numero intero che parte da 1 e  viene incrementato per ogni nuovo verbale.
\end{itemize}
Ogni verbale conterrà una sezione \textquotedbl Problemi e decisioni\textquotedbl, le cui sottosezioni rappresentano i problemi trattati durante la riunione e le decisioni prese dal gruppo per risolverli.
Per riferirsi, da un qualsiasi documento, ad una precisa sezione di un verbale è sufficiente indicare
il suo codice univoco insieme al numero di sezione interessata, utilizzando la notazione TN-S:
\begin{itemize}
\item T ed N identificano il verbale come trattato in precedenza;
\item S rappresenta la sottosezione della sezione \textquotedbl Problemi e decisioni\textquotedbl che riporta il problema e la relativa decisione del gruppo.
\end{itemize}
