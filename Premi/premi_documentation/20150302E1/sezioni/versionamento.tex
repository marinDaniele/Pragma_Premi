\subsection{Versionamento dei documenti}
\paragraph{Problema} Il metodo di \gloxy{versionamento} utilizzato per i documenti presentati nella \RR non segue una logica appropriata.
\paragraph{Decisione} I documenti subiranno un incremento di versione del tipo vX.Y.Z, dove:
\begin{itemize}
\item \textbf{X}: indica il numero di uscite formali del documento e viene incrementato in corrispondenza con l'ultima approvazione del \rRP prima del rilascio. L'incremento di \textbf{X} comporta l'azzeramento sia di \textbf{Y} che di \textbf{Z};
\item \textbf{Y}: indica il numero di modifiche e correzioni effettuate al documento. L'incremento \textbf{Y} comporta l'azzeramento di \textbf{Z};
\item \textbf{\textbf{Z}}: quando vale\begin{itemize}[label=\ding{212}]
\item 0: indica che le ultime modifiche (successive all'ultima verifica) non sono state verificate;
\item 1: indica che le ultime modifiche sono state verificate;
\item 2: indica che l'ultima verifica è stata approvata.
\end{itemize}
\end{itemize}
