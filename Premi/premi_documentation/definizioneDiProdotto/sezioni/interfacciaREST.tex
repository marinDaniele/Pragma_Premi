\colorlet{punct}{red!60!black}
\definecolor{background}{HTML}{EEEEEE}
\definecolor{delim}{RGB}{20,105,176}
\colorlet{numb}{magenta!60!black}
\lstdefinelanguage{json}{
basicstyle=\small\ttfamily,
numbers=left,
numberstyle=\scriptsize,
stepnumber=1,
numbersep=8pt,
showstringspaces=false,
breaklines=true,
frame=lines,
backgroundcolor=\color{background},
literate=
*{0}{{{\color{numb}0}}}{1}
{1}{{{\color{numb}1}}}{1}
{2}{{{\color{numb}2}}}{1}
{3}{{{\color{numb}3}}}{1}
{4}{{{\color{numb}4}}}{1}
{5}{{{\color{numb}5}}}{1}
{6}{{{\color{numb}6}}}{1}
{7}{{{\color{numb}7}}}{1}
{8}{{{\color{numb}8}}}{1}
{9}{{{\color{numb}9}}}{1}
{:}{{{\color{punct}{:}}}}{1}
{,}{{{\color{punct}{,}}}}{1}
{\{}{{{\color{delim}{\{}}}}{1}
{\}}{{{\color{delim}{\}}}}}{1}
{[}{{{\color{delim}{[}}}}{1}
{]}{{{\color{delim}{]}}}}{1},
}
\section{Interfaccia REST}\label{rest}
L'interfaccia per il \gloxy{back-end} del \gloxy{progetto} \progetto è realizzata in stile \gloxy{REST-like}, basato sullo stile \gloxy{REST} ma che non ne segue strettamente tutti i principi.
Questo consente di effettuare l'autenticazione dell'utente salvando un cookie e memorizzando nel \gloxy{server} una sessione associata al \gloxy{client}.
Una volta creata sessione, l'interfaccia con cui si possono accedere e modificare i dati nel database può considerarsi effettivamente \gloxy{REST}.\\
%Questo consente di effettuare l'autenticazione dell'utente mediante cookie, memorizzando una sessione, quindi un contesto \gloxy{client}, sul \gloxy{server}.
%All'interno della sessione, l'interfaccia con cui si possono accedere e modificare i dati nel database può considerarsi effettivamente \gloxy{REST}.\\
I dati scambiati mediante l'interfaccia sono rappresentati con il formato \gloxy{JSON}, che si integra facilmente con le tecnologie ed il linguaggio utilizzati per sviluppare \progetto.\\
Se lo scambio di dati avviene correttamente il \gloxy{server} può fornire in risposta un messaggio di conferma:
\begin{lstlisting}[language=json,firstnumber=1]
{
"status" : "ok"
}
\end{lstlisting}
\subsection{Errori}
In caso di errori il \gloxy{server} risponde con un messaggio d'errore in formato \gloxy{JSON}, definito secondo lo schema:
\begin{lstlisting}[language=json,firstnumber=1]
{
"code": [codice dell'errore],
"title": [titolo dell'errore],
"message": [descrizione testuale dell'errore]
}
\end{lstlisting}
\subsubsection{Errori generici}
Gli errori generici sono gestiti dalla classe \texttt{ErrorHandler} ad eccezione dell'errore \texttt{404} che viene gestito da \texttt{NotFoundHandler}.
In questo modo si ottiene una gestione degli errori più flessibile e modulare, in quanto è possibile inserire in coda allo \gloxy{stack} di chiamate di ogni route la classe \texttt{NotFountHandler}, riducendo così la complessità della classe \texttt{ErrorHandler}.
% in quanto una scelta progettuale più flessibile e modulare prevede, per questa tipologia di errore, una classe specifica da inserire in coda allo \gloxy{stack} delle chiamate di ogni route. In questo modo si riduce la complessità del codice che si andrà a sviluppare per ErrorHandler e allo stesso tempo si ottiene una classe altamente specializzata e facilmente manutenibile(NotFoundHandler).
\subsubsubsection{Errori lato server}
Nel caso si verifichi un errore lato \gloxy{server}, questo risponde con un codice HTTP del tipo \texttt{5xx}, di seguito vengono specificati gli errori che possono essere sollevati:
\begin{itemize}
\item \texttt{500 Errore sconosciuto}: il messaggio descrive un errore generico del \gloxy{server} che avviene quando si verifica una condizione non gestibile e quindi non identificabile con un errore specifico.
\end{itemize}
\subsubsubsection{Errori nelle richieste da parte dei client}
Nel caso si verifichi un errore riguardo le richieste ricevute dal \gloxy{client}, il \gloxy{server} risponde con un codice HTTP del tipo \texttt{4xx}, di seguito vengono specificati gli errori che possono essere sollevati:
\begin{itemize}
\item \texttt{400}: il \gloxy{server} non può gestire la richiesta in seguito ad una generica richiesta errata. Il contenuto del messaggio d'errore varia in base alla tipologia di richiesta ricevuta;
\item \texttt{401}: Accesso non autorizzato;
\item \texttt{404}: Pagina non trovata.
\end{itemize}
\subsubsection{Errori specifici di Premi}
Questi errori rappresentano situazioni specifiche del sistema \progetto, sono quindi stati definiti dei codici personalizzati per questa tipologia di errori basandosi sull'idea dei codici d'errore HTTP standard.
I codici dei messaggi d'errore sono stati assegnati secondo diverse categorie, in modo da poter individuare facilmente quale componente dell'applicazione ha causato l'errore.\\
Ad ogni categoria è stato assegnato successivamente un intervallo di codici possibili, evitato di utilizzare gli intervalli \texttt{4xxx} e \texttt{5xxx} per non creare ambiguità con i codici d'errore standard descritti nella sezione precedente.\\
Le categorie di errori definite sono le seguenti:
\begin{itemize}
\item \texttt{1xxx}: errori di autenticazione e del middleware \texttt{userById};
\item \texttt{2xxx}: errori del middleware \texttt{projectById};
\item \texttt{3xxx}: errori del middleware \texttt{nodeById};
\item \texttt{6xxx}: errori del middleware \texttt{associationById};
\item \texttt{7xxx}: errori del middleware \texttt{pathById};
\item \texttt{8xxx}: errori del \gloxy{controller} \texttt{ProjectManagementController};
\item \texttt{9xxx}: errori del \gloxy{controller} \texttt{NodeController};
\item \texttt{10xxx}: errori del \gloxy{controller} \texttt{PathController};
\item \texttt{11xxx}: errori degli oggetti \texttt{ProjectModel};
\item \texttt{12xxx}: errori degli oggetti \texttt{PathModel}.
\end{itemize}
Gli errori specifici di \progetto vengono gestiti dalla classe \texttt{PremiError} e sono di seguito elencati sotto forma di:
\begin{center}
\texttt{Codice Titolo}: messaggio d'errore
\end{center}
\begin{itemize}
\item \texttt{1000 Utente non trovato}: l'identificativo utente fornito non è un identificativo valido;
\item \texttt{1002 Credenziali non valide}: è necessario fornire un indirizzo email ed una password valide;
\item \texttt{2000 \gloxy{Progetto} non trovato}: l'identificativo del \gloxy{progetto} fornito non è un identificativo valido;
\item \texttt{3000 Nodo non trovato}: l'identificativo del nodo fornito non è un identificativo valido;
\item \texttt{6000 Associazione non trovata}: l'identificativo dell'associazione fornita non è un identificativo valido;
\item \texttt{7000 \gloxy{Percorso} non trovato}: l'identificativo del \gloxy{percorso} fornito non è un identificativo valido;
\item \texttt{8000 \gloxy{Progetto} corrotto}: errore nella ricerca dei campi dati relativi al \gloxy{progetto} indicato;
\item \texttt{8001 Dati non validi}: i dati relativi al \gloxy{progetto} sono vuoti o errati;
\item \texttt{8002 \gloxy{Progetto} corrotto}: errore durante l'eliminazione del \gloxy{progetto};
\item \texttt{9000 Dati non validi}: i dati relativi al contenuto del nodo non sono validi o sono formattati in modo errato;
\item \texttt{9001 Nodi non validi}: gli identificativi dei nodi forniti non sono validi oppure non esistono oppure il nodo entrante coincide con il nodo uscente;
\item \texttt{10000 Dati non validi}: i dati per la modifica del \gloxy{percorso} non sono definiti;
\item \texttt{11001 Nodo del \gloxy{progetto} non trovato}: il nodo riferito nella relazione o nel \gloxy{percorso} non è stato trovato all'interno del \gloxy{progetto} indicato;
\item \texttt{11002 Nodo padre non esistente}: il nodo padre indicato non esiste o non corrisponde ad un nodo valido;
\item \texttt{11003 Nodo non valido}: impossibile eliminare il nodo radice della mappa mentale.
\item \texttt{12000 Nodo del \gloxy{percorso} non trovato}: il nodo riferito non è stato trovato all'interno del \gloxy{percorso} indicato.
\end{itemize}
\subsection{Risorse REST}
In seguito vengono riportate le risorse \gloxy{REST} fornite, associate al tipo di metodo HTTP che è possibile richiedere su di esse e ai permessi necessari per effettuare la richiesta. In particolare, i permessi sono:
\begin{itemize}
\item \textbf{Utente}: la risorsa può essere acceduta da qualsiasi tipo di utente;
\item \textbf{Utente autenticato}: la risorsa può essere acceduta solo dagli utenti che hanno effettuato il login.
\end{itemize}
Inoltre, per ogni risorsa, sono stati specificati i formati per lo scambio dei dati in \gloxy{JSON}:
\begin{itemize}
\item \textbf{Request:} rappresenta l'oggetto \gloxy{JSON} che dovrà essere passato alla risorsa \gloxy{REST};
\item \textbf{Response:} rappresenta l'oggetto \gloxy{JSON} che fornirà in risposta la risorsa \gloxy{REST}.
\end{itemize}
\begin{itemize}
\item \texttt{/signup}
\begin{itemize}
\item \textbf{Method:} POST;
\item \textbf{Livello di Accesso:} utente;
\item \textbf{Descrizione:} crea un nuovo account. L'indirizzo email rappresenta lo username dell'utente e lo identifica univocamente, pertanto un utente non può registrare più account con lo stesso indirizzo di posta elettronica. Restituisce un messaggio di conferma se viene effettuato correttamente, altrimenti un errore;
\item \textbf{Request:} lo scambio dei dati dell'utente avviene attraverso una form che deve avere i seguenti campi:
\begin{lstlisting}[language=json,firstnumber=1]
{
"username" : [username che corrisponde all'indirizzo di posta elettronica con cui ci si vuole registrare];
"password" : [la password per l'account con cui ci si vuole registrare].
}
\end{lstlisting}
\end{itemize}
%%%%%%%%%%%%%%%%%%%
%\item \texttt{/register}
%	\begin{itemize}
%	\item \textbf{Method:} DELETE;
%	\item \textbf{Livello di Accesso:} utente autenticato;
%	\item \textbf{Descrizione:} elimina l'account dell'utente dal database. Questa funzionalità non è attualmente implementata;
%	\item \textbf{Request:}
%	\item \textbf{Response:}
%	\end{itemize}
%%%%%%%%%%%%%%%%%%%
\item \texttt{/login}
\begin{itemize}
\item \textbf{Method:} POST;
\item \textbf{Livello di Accesso:} utente;
\item \textbf{Descrizione:} effettua l'accesso all'account. Restituisce un messaggio di conferma se viene effettuato correttamente, altrimenti un errore;
\item \textbf{Request:}  lo scambio dei dati dell'utente avviene attraverso una form che deve avere i seguenti campi:
\begin{lstlisting}[language=json,firstnumber=1]
{
"username" : [username che corrisponde all'indirizzo di posta elettronica che identifica l'utente];
"password" : [la password dell'account dell'utente].
}
\end{lstlisting}
\end{itemize}
%%%%%%%%%%%%%%%%%%%%%%%
\item \texttt{/logout}
\begin{itemize}
\item \textbf{Method:} GET;
\item \textbf{Livello di Accesso:} utente autenticato;
\item \textbf{Descrizione:} effettua il logout. Restituisce un messaggio di conferma se viene effettuato correttamente, altrimenti un errore;
%\item \textbf{Request:}
%\item \textbf{Response:}
\end{itemize}
%%%%%%%%%%%%%%%%%%%%%%
\item \texttt{/projects}
\begin{itemize}
\item \textbf{Method:} POST;
\item \textbf{Livello di Accesso:} utente autenticato;
\item \textbf{Descrizione:} crea un nuovo \gloxy{progetto} relativo all’utente della sessione corrente e utilizzando il nome specificato nei parametri POST. restituisce un oggetto JSON contenente l'id del progetto, le indicazioni per il layout, le liste di nodi, le relazioni e le associazioni della \gloxy{mappa mentale} del \gloxy{progetto} \texttt{:projectId}, altrimenti un messaggio di errore;
\item \textbf{Request:}
\begin{lstlisting}[language=json,firstnumber=1]
{
"name": [nome del progetto]
}
\end{lstlisting}
\item \textbf{Response:}
\begin{lstlisting}[language=json,firstnumber=1]
{
"_id" : [identificativo del progetto],
"name" : [nome del progetto],
"fontFamily" : [nome della famiglia],
"fontColor" : [nome del colore],
"bkgColor" : [nome del colore],
"relations" : [array JSON di relazioni, con attributi specificati nei diagrammi delle classi],
"nodes" : [array JSON di nodi, con attributi specificati nei diagrammi delle classi]
}
\end{lstlisting}
\end{itemize}
%%%%%%%%%%%%%%%%%%%%%%%%
\item \texttt{/projects}
\begin{itemize}
\item \textbf{Method:} GET;
\item \textbf{Livello di Accesso:} utente autenticato;
\item \textbf{Descrizione:} restituisce  un array di oggetti \gloxy{JSON} che descrive identificativo e nome dei \gloxy{progetti} nel database per l'utente della sessione. Inoltre, per ogni \gloxy{progetto}, fornisce l'elenco dei \gloxy{percorsi di presentazione} associati;
%\item \textbf{Request:}
\item \textbf{Response:}
\begin{lstlisting}[language=json,firstnumber=1]
{
[{
"_id" : [identificativo del progetto],
"name": [nome del progetto],
"paths": [
{
"_id": [identificativo del precorso],
"name" : [nome del percorso]
}
]
}]
}
\end{lstlisting}
\end{itemize}
%%%%%%%%%%%%%%%%%%
\item \texttt{/projects/:projectId}
\begin{itemize}
\item \textbf{Method:} GET;
\item \textbf{Livello di Accesso:} utente autenticato;
\item \textbf{Descrizione:} restituisce un oggetto \gloxy{JSON} contenente le indicazioni per il layout, le liste di nodi, le relazioni e le associazioni della \gloxy{mappa mentale} del \gloxy{progetto} \texttt{:projectId}, altrimenti un messaggio di errore;
%\item \textbf{Request:}
\item \textbf{Response:}
\begin{lstlisting}[language=json,firstnumber=1]
{
"name" : [nome del progetto],
"fontFamily" : [nome della famiglia],
"fontColor" : [nome del colore],
"bkgColor" : [nome del colore],
"relations" : [array JSON di relazioni, con attributi specificati nei diagrammi delle classi],
"nodes" : [array JSON di nodi, con attributi specificati nei diagrammi delle classi]
}
\end{lstlisting}
\end{itemize}
%%%%%%%%%%%%%
\item \texttt{/projects/:projectId}
\begin{itemize}
\item \textbf{Method:} PUT;
\item \textbf{Livello di Accesso:} utente autenticato;
\item \textbf{Descrizione:} modifica le impostazioni del \gloxy{progetto} \texttt{:projectId}, relativo all'utente della sessione corrente. Restituisce un messaggio di conferma se viene effettuato correttamente, altrimenti un errore;
\item \textbf{Request:}
\begin{lstlisting}[language=json,firstnumber=1]
{
"name" : [nome del progetto],
"bkgColor" : [colore dello sfondo],
"fontColor" : [colore per i testi],
"fontFamily" : [famiglia del font per i testi]
}
\end{lstlisting}
%\item \textbf{Response:}
\end{itemize}
%%%%%%%%%%%%%
\item \texttt{/projects/:projectId}
\begin{itemize}
\item \textbf{Method:} DELETE;
\item \textbf{Livello di Accesso:} utente autenticato;
\item \textbf{Descrizione:} rimuove il \gloxy{progetto} con identificativo \texttt{:projectId} relativo all'utente della sessione corrente dal database. Restituisce un messaggio di conferma, altrimenti uno di errore;
%\item \textbf{Request:}
%\item \textbf{Response:}
\end{itemize}
%%%%%%%%%%%%%%%%%%%%%%%%%%%%%%%%
\item \texttt{/projects/:projectId/presentations}
\begin{itemize}
\item \textbf{Method:} GET;
\item \textbf{Livello di Accesso:} utente autenticato;
\item \textbf{Descrizione:} restituisce tutti i nodi della mappa mentale, quindi del \gloxy{progetto} \texttt{projectId} dell'utente della sessione. Restituisce un oggetto \gloxy{JSON} creato ad hoc per eseguire una presentazione nel font-end dell'applicazione. Altrimenti restituisce un messaggio di errore;
%\item \textbf{Request:}
\item \textbf{Response:}
\begin{lstlisting}[language=json,firstnumber=1]
{
"frames" : [
{
"node" : [contenuto e informazioni del nodo, con attributi definiti nei diagrammi delle classi],
"family" : [{
{
"children":[
{
"id" : ["identificativo dei nodi figli"],
"title": ["titolo del nodo"]
}]
"parent" : {
"id" : ["identificativo del nodo padre"],
"title": ["titolo del nodo"]
}
}],
"associations" : [
{
"id" : [identificativo dei nodi con cui esiste una associazione],
"title": ["titolo del nodo"]
}
]
}
]
}
\end{lstlisting}
\end{itemize}
%%%%%%%%%%%%%%%%%%%%%%%
\item \texttt{/projects/:projectId/nodes/:nodeId}
\begin{itemize}
\item \textbf{Method:} POST;
\item \textbf{Livello di Accesso:} utente autenticato;
\item \textbf{Descrizione:} inserisce nel database un nuovo nodo vuoto come figlio del nodo \texttt{:nodeId}, relativo al \gloxy{progetto} \texttt{:projectId}. Restituisce i dati del nuovo nodo oppure un messaggio di errore;
\item \textbf{Response:}
\begin{lstlisting}[language=json,firstnumber=1]
{
"nodeId" : [identificativo del nodo appena creato]
"contents" : [
{
"content" : [testo o URL],
"x" : [coordinata sull'asse x dell'elemento nel frame],
"y" : [coordinata sull'asse y dell'elemento nel frame],
"height" : [altezza dell'elemento],
"width" : [larghezza dell'elemento],
"class" : [title, text o imgURL]
}
]
}
\end{lstlisting}
\end{itemize}
%%%%%%%%%%%%%%%%%%%%%%%
\item \texttt{/projects/:projectId/nodes/:nodeId}
\begin{itemize}
\item \textbf{Method:} PUT;
\item \textbf{Livello di Accesso:} utente autenticato;
\item \textbf{Descrizione:} aggiorna il contenuto del nodo \texttt{:nodeId}, nel \gloxy{progetto} \texttt{:projectId}, relativo all'utente della sessione. Restituisce un messaggio di conferma oppure un messaggio di errore;
%\item \textbf{Request:}
\item \textbf{Request:}
\begin{lstlisting}[language=json,firstnumber=1]
{
"contents" : [
{
"content" : [testo o URL],
"x" : [coordinata sull'asse x dell'elemento nel frame],
"y" : [coordinata sull'asse y dell'elemento nel frame],
"height" : [altezza dell'elemento],
"width" : [larghezza dell'elemento],
"class" : [title, text o imgURL]
}
]
}
\end{lstlisting}
\end{itemize}
%%%%%%%%%%%%%%%%%%%%%%%%%%%
\item \texttt{/projects/:projectId/nodes/:nodeId}
\begin{itemize}
\item \textbf{Method:} DELETE;
\item \textbf{Livello di Accesso:} utente autenticato;
\item \textbf{Descrizione:} elimina dal database il nodo \texttt{:nodeId}, nel \gloxy{progetto} \texttt{:projectId}, relativo all'utente della sessione. Oltre ad esso viene eliminato l'eventuale sotto-albero di cui \texttt{:nodeId} è padre, le associazioni che coinvolgono i nodi del sotto-albero e vengono aggiornati i \gloxy{percorsi} in cui il nodo compariva. Restituisce la nuova mappa mentale, con nodi e relazioni aggiornati oppure un messaggio di errore;
%\item \textbf{Request:}
\item \textbf{Response:}
\begin{lstlisting}[language=json,firstnumber=1]
{
"nodes" : [array JSON di nodi, con attributi specificati nei diagrammi delle classi],
"relations" : [array JSON di relazioni, con attributi specificati nei diagrammi delle classi]
}
\end{lstlisting}
\end{itemize}
%%%%%%%%%%%%%%%%%%%%%%%%%%%%%%%%
\item \texttt{/projects/:projectId/associations}
\begin{itemize}
\item \textbf{Method:} POST;
\item \textbf{Livello di Accesso:} utente autenticato;
\item \textbf{Descrizione:} crea una relazione di tipo associazione nel \gloxy{progetto} \texttt{:projectId}, relativo all'utente della sessione. Richiede gli identificativi dei nodi che formano l'associazione. Restituisce un messaggio di conferma oppure un messaggio di errore;
\item \textbf{Request:}
\begin{lstlisting}[language=json,firstnumber=1]
{
"sourceId" : [identificativo del nodo sorgente della relazione],
"destinationId" : [identificativo del nodo destinazione della relazione]
}
\end{lstlisting}
\item \textbf{Response:}
\begin{lstlisting}[language=json,firstnumber=1]
{
"associationId" : [identificativo dell'associazione creata]
}
\end{lstlisting}
\end{itemize}
%%%%%%%%%%%%%%%%%%%%%%%%%%%%%%%
\item \texttt{/projects/:projectId/associations/:associationId}
\begin{itemize}
\item \textbf{Method:} DELETE;
\item \textbf{Livello di Accesso:} utente autenticato;
\item \textbf{Descrizione:} elimina una relazione di tipo associazione nel \gloxy{progetto} \texttt{:projectId}, relativo all'utente della sessione. Richiede gli identificativi dei nodi che formano l'associazione. Restituisce un messaggio di conferma oppure un messaggio di errore;
\item \textbf{Request:}
\begin{lstlisting}[language=json,firstnumber=1]
{
"associationId" : [identificativo dell'associazione creata]
}
\end{lstlisting}
\end{itemize}
%%%%%%%%%%%%%%%%%%%%%%%%%%%%%%%
\item \texttt{/projects/:projectId/paths}
\begin{itemize}
\item \textbf{Method:} POST;
\item \textbf{Livello di Accesso:} utente autenticato;
\item \textbf{Descrizione:} aggiunge un \gloxy{percorso di presentazione} vuoto al \gloxy{progetto} \texttt{:projectId} relativo all'utente della sessione. Richiede che venga specificato un nome per il \gloxy{percorso}. Restituisce l'identificativo del nuovo \gloxy{percorso}, generato automaticamente dal database o un messaggio di errore;
\item \textbf{Request:}
\begin{lstlisting}[language=json,firstnumber=1]
{
"name" : [nome per il percorso]
}
\end{lstlisting}
\item \textbf{Response:}
\begin{lstlisting}[language=json,firstnumber=1]
{
"pathId" : [identificativo del percorso],
"name" : [nome del percorso]
}
\end{lstlisting}
\end{itemize}
%%%%%%%%%%%%%%%%%%%%%%%%%%%%
\item \texttt{/projects/:projectId/paths}
\begin{itemize}
\item \textbf{Method:} GET;
\item \textbf{Livello di Accesso:} utente autenticato;
\item \textbf{Descrizione:} restituisce i \gloxy{percorsi di presentazione} del \gloxy{progetto} \texttt{:projectId} relativo all'utente della sessione. Altrimenti restituisce un messaggio di errore;
%\item \textbf{Request:}
\item \textbf{Response:}
\begin{lstlisting}[language=json,firstnumber=1]
{
"paths" : [{
"id" : [identificativo del percorso],
"name" : [nome del percorso]
}]
}
\end{lstlisting}
\end{itemize}
%%%%%%%%%%%%%%%%%%%%%%%%%%%%%%%%%%
\item \texttt{/projects/:projectId/paths/:pathId/:nodeId}
\begin{itemize}
\item \textbf{Method:} POST;
\item \textbf{Livello di Accesso:} utente autenticato;
\item \textbf{Descrizione:} aggiunge il nodo \texttt{:nodeId} al \gloxy{percorso} \texttt{:pathId} relativo al \gloxy{progetto} \texttt{projectId} dell'utente della sessione. Richiede che venga specificato l'identificativo del nodo da aggiungere. Restituisce tutti i nodi del \gloxy{percorso} \texttt{:pathId} relativi al \gloxy{progetto} \texttt{projectId} dell'utente della sessione oppure un messaggio di errore;
%\item \textbf{Request:}
%\begin{lstlisting}[language=json,firstnumber=1]
%{
%"nodeId" : [identificativo del nodo da aggiungere al percorso]
%}
%\end{lstlisting}
\item \textbf{Response:}
\begin{lstlisting}[language=json,firstnumber=1]
{
"name": [nome del percorso]
"nodes" : [array JSON di nodi, con attributi specificati nei diagrammi delle classi]
}
\end{lstlisting}
\end{itemize}
%%%%%%%%%%%%%%%%%%%%%%%%%%%%%%%%%
\item \texttt{/projects/:projectId/paths/:pathId}
\begin{itemize}
\item \textbf{Method:} GET;
\item \textbf{Livello di Accesso:} utente autenticato;
\item \textbf{Descrizione:} restituisce tutti i nodi del \gloxy{percorso} \texttt{:pathId} relativi al \gloxy{progetto} \texttt{projectId} dell'utente della sessione oppure un messaggio di errore;
%\item \textbf{Request:}
\item \textbf{Response:}
\begin{lstlisting}[language=json,firstnumber=1]
{
"name": [nome del percorso]
"nodes" : [array \gloxy{JSON} di nodi, con attributi specificati nei diagrammi delle classi]
}
\end{lstlisting}
\end{itemize}
%%%%%%%%%%%%%%%%%%%%%%%%%%%%
\item \texttt{/projects/:projectId/paths/:pathId}
\begin{itemize}
\item \textbf{Method:} PUT;
\item \textbf{Livello di Accesso:} utente autenticato;
\item \textbf{Descrizione:} aggiorna le informazioni del \gloxy{percorso} \texttt{:pathId} nel \gloxy{progetto} \texttt{:projectId}, relativo all'utente della sessione, in particolare consente di rinominarlo. Richiede che venga specificato un nuovo nome per il \gloxy{percorso}. Restituisce un messaggio di conferma o un messaggio di errore;
\item \textbf{Request:}
\begin{lstlisting}[language=json,firstnumber=1]
{
"name" : [nome per il percorso]
}
\end{lstlisting}
%\item \textbf{Response:}
\end{itemize}
%%%%%%%%%%%%%%%%%%%%%%%%%%%%%%%%%
\item \texttt{/projects/:projectId/paths/:pathId}
\begin{itemize}
\item \textbf{Method:} DELETE;
\item \textbf{Livello di Accesso:} utente autenticato;
\item \textbf{Descrizione:} rimuove il \gloxy{percorso di presentazione} \texttt{:pathId} dal \gloxy{progetto} \texttt{:projectId}, relativo all'utente della sessione. Restituisce un messaggio di conferma oppure un messaggio di errore;
%\item \textbf{Request:}
%\item \textbf{Response:}
\end{itemize}
%%%%%%%%%%%%%%%%%%%%%%%%%%%%%%%%%%
\item \texttt{/projects/:projectId/paths/:pathId/:nodeId}
\begin{itemize}
\item \textbf{Method:} DELETE;
\item \textbf{Livello di Accesso:} utente autenticato;
\item \textbf{Descrizione:} rimuove il nodo \texttt{:nodeId} dal \gloxy{percorso} \texttt{:pathId}, relativo al \gloxy{progetto} \texttt{projectId} dell'utente della sessione. Restituisce tutti i nodi del \gloxy{percorso} \texttt{:pathId} relativi al \gloxy{progetto} \texttt{projectId} dell'utente della sessione oppure un messaggio di errore;
%\item \textbf{Request:}
\item \textbf{Response:}
\begin{lstlisting}[language=json,firstnumber=1]
{
"name": [nome del percorso]
"nodes" : [array \gloxy{JSON} di nodi, con attributi specificati nei diagrammi delle classi]
}
\end{lstlisting}
\end{itemize}
%%%%%%%%%%%%%%%%%%%%%%%%%%%%%%%%%%%%%%%%%%%%%%%%%%%%%%
\item \texttt{/statics/userManual}
\begin{itemize}
\item \textbf{Method:} GET;
\item \textbf{Livello di Accesso:} utente;
\item \textbf{Descrizione:} fornisce al chiamante l'URL della pagina \gloxy{web} contenente il manuale utente dell'applicativo;
%\item \textbf{Request:}
%\item \textbf{Response:}
\end{itemize}
\end{itemize}
