\section{Introduzione} \label{intro}
\subsection{Scopo del documento}
Il presente documento ha lo scopo di definire in dettaglio la struttura e il funzionamento delle componenti del \gloxy{progetto} \progetto. Questo documento servirà come guida per i \rps del gruppo \gruppo fornendo direttive e vincoli per la realizzazione del \gloxy{progetto}.
\subsection{Scopo del prodotto}
\scopoProdotto
\subsection{Glossario}
\descrizioneGlossario
\subsection{Riferimenti}
\subsubsection{Normativi}
\begin{itemize}
\item \normeDiProgetto
\item \analisiDeiRequisiti
\item \textbf{Verbale esterno:} \eV.
\end{itemize}
\subsubsection{Informativi}
\begin{itemize}
\item\textbf{Ingegneria del software - Ian Sommerville - 8a edizione (2007):} \\ Parte terza: Progettazione, Capitolo 11: Progettazione architetturale, Capitolo 14: Progettazione orientata agli oggetti;
\item \textbf{Design Patterns} - Erich Gamma, Richard Helm, Ralph Johnson, John Vlissides - 1a edizione italiana (2008);
\item \textbf{Slide dell'insegnamento - Design patterns:} \\
\begin{itemize}
\item Introduzione: \\
\url{http://www.math.unipd.it/~tullio/IS-1/2014/Dispense/E4.pdf};
\item Strutturali: \\
\url{http://www.math.unipd.it/~tullio/IS-1/2014/Dispense/E6.pdf};
\item Creazionali: \\
\url{http://www.math.unipd.it/~tullio/IS-1/2014/Dispense/E7.pdf};
\item Comportamentali:  \\
\url{http://www.math.unipd.it/~tullio/IS-1/2014/Dispense/E8.pdf};
\item Architetturali: \\
\url{http://www.math.unipd.it/~tullio/IS-1/2014/Dispense/E9.pdf};
\url{http://www.math.unipd.it/~rcardin/pdf/Design%20Pattern%20Architetturali%20-%20Model%20View%20Controller_4x4.pdf};
\end{itemize}
\item \textbf{Martin Fowler - \gloxy{UML} Distilled} - 2nd edition;
\item\textbf{Slide dell'insegnamento - Diagrammi delle classi}:
\url{http://www.math.unipd.it/~tullio/IS-1/2014/Dispense/E2a.pdf}
\item\textbf{Slide dell'insegnamento - Diagrammi dei package}:
\url{http://www.math.unipd.it/~tullio/IS-1/2014/Dispense/E2b.pdf}
\item\textbf{Slide dell'insegnamento - Diagrammi di sequenza}:
\url{http://www.math.unipd.it/~tullio/IS-1/2014/Dispense/E3a.pdf}
\item \textbf{Documentazione del \gloxy{framework} \gloxy{MEAN}.js}: \\ \url{http://meanjs.org/docs.html};
\item \textbf{Documentazione della \gloxy{piattaforma} Node.js}: \\ \url{https://nodejs.org/documentation};
\item \textbf{Breve guida ai middleware Express}: \\ \url{http://stephensugden.com/middleware_guide/};
\item \textbf{Guida all'utilizzo dei middleware Express}: \\ \url{http://expressjs.com/guide/using-middleware.html};
\item \textbf{Guida all'utilizzo dei middleware Passport}: \\ \url{http://passportjs.org/guide};
\item \textbf{Manuale del database \gloxy{MongoDB}}: \\ \url{https://docs.mongodb.org/manual};
\item \textbf{Documentazione del \gloxy{framework} \gloxy{AngularJS}}:
\begin{itemize}
\item \textit{Documentazione generica}: \url{https://docs.angularjs.org/guide};
\item \textit{Documentazione servizio \$http}: \url{https://docs.angularjs.org/api/ng/service/$http}; %$stranamente dentro \url non serve il \ per fare il $ $
\item \textit{Documentazione servizio \$location}: \url{https://docs.angularjs.org/api/ng/service/$location}; %$
\item \textit{Documentazione servizio \$window}: \url{https://docs.angularjs.org/api/ng/service/$window}; %$
\item \textit{Documentazione servizio \$routeParams}: \url{https://docs.angularjs.org/api/ngRoute/service/$routeParams}; %$
\item \textit{Documentazione servizio \$q}: \url{https://docs.angularjs.org/api/ng/service/$q}.%$
\end{itemize}
\item \textbf{Documentazione del \gloxy{framework} \nogloxy{Material for Angular}}:\\ \url{https://material.angularjs.org/latest/#/}.
\end{itemize}
