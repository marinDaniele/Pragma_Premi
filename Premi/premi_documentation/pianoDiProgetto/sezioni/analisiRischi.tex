\section{Analisi dei rischi}\label{adr}
%le probabilità di occorrenza sono state commentate nell'attuazione delle modifiche in seguito a E1-6
Per garantire il corretto avanzamento del \gloxy{progetto} è stata effettuata un'attenta analisi dei rischi applicando la  procedura di rilevazione indicata in \normeDiProgetto. \\
La procedura descritta ha portato all'identificazione di vari rischi e alla predisposizione di contromisure per ridurne l'impatto sul \gloxy{progetto}. \\
I rischi saranno quindi costantemente monitorati e, nel caso si verificassero, saranno mitigati dalle contromisure decise proattivamente durante la procedura di rilevazione.\\
Infine il loro riscontro effettivo nel corso del \gloxy{progetto} sarà descritto nel presente documento ed evidenziato in §\ref{rplg}.\\ Quest'ultimo sarà lo strumento principale utilizzato dal \rRP per avere una visione chiara e semplice dello stato attuale in cui si trova il \gloxy{team} rispetto ai rischi individuati. \\
Ogni rischio individuato è descritto con:
\begin{itemize}
%\item \textbf{Probabilità di occorrenza:} valutazione della probabilità di occorrenza del rischio, espressa in: Alta, Media o Bassa;
\item \textbf{Pericolosità:} valutazione della pericolosità del rischio e dell'impatto che ha sulla pianificazione del \gloxy{progetto}, espressa in: Alta, Media e Bassa;
\item \textbf{Descrizione:} breve descrizione del rischio;
\item \textbf{Strategie di rilevazione:} descrizione di come viene individuata la situazione rischiosa e definizione delle responsabilità di rilevamento;
\item \textbf{Contromisure:} descrizione delle contromisure che saranno adottate per ridurre l'impatto del rischio nel caso che si verifichi.
\item \textbf{Riscontro:} descrizione di come e se il rischio è stato effettivamente riscontrato.
\end{itemize}
\uline{I parametri appena indicati saranno costantemente aggiornati} per avere una visione sempre attualizzata della situazione del \gloxy{team} e permetteranno al \rRP di agire tempestivamente, applicando le contromisure adeguate.
\subsection{Rischi individuati}\label{ri}
% intro
%per ogni rischio:
%Occorrenza, Pericolosità, Descrizione, Rilevazione, Contromisure
\subsubsection{Livello tecnologico} \label{lt}
%Mancanza di conoscenza riguardo le tecnologie
\subsubsubsection{Problemi legati alle tecnologie}\label{r_tec}
\begin{itemize}
%\item \textbf{Possibilità di occorrenza:} Media;
\item \textbf{Pericolosità:} Alta;
\item \textbf{Descrizione:} Le tecnologie imposte dal capitolato sono note a buona parte del gruppo. Tuttavia non è da escludere la possibilità che i membri del gruppo incontrino problemi nell'utilizzo di queste;
\item \textbf{Strategie di rilevazione:} Sarà compito del \rRP verificare le conoscenze dei componenti monitorando costantemente la loro preparazione riguardo le tecnologie adottate;
\item \textbf{Contromisure:}
\begin{itemize}
\item Il materiale necessario per documentarsi sulle tecnologie sarà fornito dall'\rAP;
\item La pianificazione verrà attualizzata tenendo conto del rischio e calcolando nel dettaglio, per ogni componente, il tempo necessario affinché egli possa documentarsi adeguatamente riguardo le tecnologie adottate.
\end{itemize}
\item \textbf{Riscontro:} Non è stato finora riscontrato nessun rischio.
\end{itemize}
%Framework inadeguato
\subsubsubsection{Problemi legati al framework adottato}\label{r_framework}
\begin{itemize}
%\item \textbf{Possibilità di occorrenza:} Alta;
\item \textbf{Pericolosità:} Alta;
\item \textbf{Descrizione:} La scelta del \gloxy{framework} per la realizzazione del \gloxy{progetto} è a discrezione del gruppo. Data la grande quantità di framework attualmente disponibili, esiste la possibilità che il gruppo scelga un \gloxy{framework} non adatto allo sviluppo del \gloxy{progetto};
\item \textbf{Strategie di rilevazione:} I \rPs dovranno porre molta attenzione nella scelta del \gloxy{framework}, documentandosi adeguatamente riguardo gli ambiti di possibile utilizzo dello stesso. Dovranno inoltre effettuare dei test preliminari per verificare che il \gloxy{framework} soddisfi le aspettative attese;
\item \textbf{Contromisure:}
\begin{itemize}
\item Se questo tipo di problema viene rilevato prima dell'inizio dello sviluppo, i \rPs, in collaborazione con l'\rAP, dovranno cercare un nuovo \gloxy{framework} oppure trovare una soluzione alternativa.
\end{itemize}
\item \textbf{Riscontro:} In seguito ad un incremento dei requisiti si sono dovuti studiare e valutare una serie di \gloxy{framework} necessari per realizzare un sistema che ben si adattasse a risolvere il nuovo tipo di problema. \`E stato inoltre necessario creare dei prototipi per progettare correttamente le componenti di dettaglio dell'architettura.
\end{itemize}
%Rotture \gloxy{HW}
\subsubsubsection{Problemi legati alle risorse hardware}\label{r_hw}
\begin{itemize}
%\item \textbf{Possibilità di occorrenza:} Bassa;
\item \textbf{Pericolosità:} Bassa;
\item \textbf{Descrizione:} I computer usati dai vari componenti potrebbero rompersi o avere malfunzionamenti;
\item \textbf{Strategie di rilevazione:} Qualora il computer di un componente del gruppo diventi inutilizzabile, sarà compito del proprietario avvisare il \rRP, il quale provvederà a trovare una soluzione;
\item \textbf{Contromisure:}
\begin{itemize}
\item Alcuni membri del gruppo hanno messo a disposizione un secondo computer da prestare per diminuire la perdita di tempo;
\item Sarà possibile utilizzare i computer presenti nei laboratori dell'università. \\
\item La facilità di installazione e la portabilità dei programmi scelti per lo sviluppo del \gloxy{progetto} renderanno possibile ritornare al lavoro in breve tempo;
\end{itemize}
\item \textbf{Riscontro:} Non è stato finora riscontrato nessun rischio.
\end{itemize}
%Problemi dovuti all'utilizzo di servizi \gloxy{Cloud}
\subsubsubsection{Problemi legati alle applicazioni web}\label{r_appweb}
\begin{itemize}
%\item \textbf{Possibilità di occorrenza:} Alta;
\item \textbf{Pericolosità:} Bassa;
\item \textbf{Descrizione:} Il \gloxy{repository}, il sistema di \gloxy{ticketing} e i programmi per la gestione dei requisiti e del glossario sono eseguiti su \gloxy{server} terzi esterni al gruppo;
\item \textbf{Strategie di rilevazione:} Non è possibile rilevare anticipatamente questo tipo di problematiche;
\item \textbf{Contromisure:}
\begin{itemize}
\item Sono stati scelti fornitori ritenuti affidabili e che forniscono questo tipo di servizi da molto tempo;
\item L'\rAP effettuerà periodicamente dei back-up dei \gloxy{server} sul suo computer locale;
\item grazie all'uso di \gloxy{Git} tutto il contenuto del \gloxy{repository} è salvato anche in locale, sarà compito dei componenti del gruppo avere salvata in locale la versione più aggiornata.
\end{itemize}
\item \textbf{Riscontro:} Si è verificata questa problematica durante la \fPD in quanto \pragmadb è stato indisponibile per svariate ore impedendo il lavoro di tracciamento da parte dei membri del gruppo.
\end{itemize}
\subsubsection{Livello personale}\label{lp}
%Mancanza dei componenti del gruppo per malattia o problemi personali
\subsubsubsection{Problemi dei componenti del gruppo}\label{r_comp}
\begin{itemize}
%\item \textbf{Possibilità di occorrenza:} Media;
\item \textbf{Pericolosità:} Alta;
\item \textbf{Descrizione:} Può capitare che durante la realizzazione del \gloxy{progetto} alcuni componenti del gruppo si ammalino o si ritrovino ad avere altri impegni universitari o extra universitari e che per un determinato periodo di tempo non possano lavorare al \gloxy{progetto};
\item \textbf{Strategie di rilevazione:} I componenti del gruppo che non possono lavorare in un determinato periodo devono segnalarlo sul calendario condiviso e comunicarlo tempestivamente al \rRP, in modo che quest'ultimo possa avere una visione complessiva delle indisponibilità;
\item \textbf{Contromisure:}
\begin{itemize}
\item Il \rRP dovrà effettuare una nuova pianificazione delle attività per il periodo di tempo interessato, tenendo conto delle assenze in modo da evitare o limitare i ritardi e non compromettere il limite imposto dalle \gloxy{milestone} fissate;
\item L'interessato dovrà preoccuparsi di riallinearsi il prima possibile con il resto del \gloxy{team}.
\end{itemize}
\item \textbf{Riscontro:}
Si sono verificati due casi di assenza imprevista di un componente per motivi lavorativi, entrambe le assenze si sono verificate in corrispondenza di riunioni interne al \gloxy{team}. \\ In questo caso il componente si è allineato la sera stessa con le decisioni prese  dal resto del \gloxy{team} consultando il verbale redatto dai presenti e verificandone la correttezza attraverso la registrazione audio della riunione svolta.
%Liti tra componenti del gruppo
\end{itemize}
\subsubsubsection{Problemi tra componenti del gruppo}\label{r_liti}
\begin{itemize}
%\item \textbf{Possibilità di occorrenza:} Bassa;
\item \textbf{Pericolosità:} Alta;
\item \textbf{Descrizione:} Essendo il \gloxy{team} composto da persone che prima del \gloxy{progetto} non avevano mai lavorato in un gruppo così numeroso, è probabile che nascano divergenze o incomprensioni;
\item \textbf{Strategie di rilevazione:}
\begin{itemize}
\item Il \rRP dovrà controllare l'evoluzione dei rapporti tra i componenti;
\item \`E compito di ogni componente segnalare al \rRP la nascita di eventuali problemi, qualora si verifichino.
\end{itemize}
\item \textbf{Contromisure:}
\begin{itemize}
\item Sarà compito del \rRP fare da mediatore tra le parti;
\item Nei casi più gravi dovrà pianificare nuovamente le attività, evitando che i componenti che hanno avuto divergenze si trovino a lavorare assieme.
\end{itemize}
\item \textbf{Riscontro:} Si è verificato un caso di lite tra componenti del gruppo impegnati nell'attività di Progettazione.
Il \rRP, non essendo presente al momento dell'accaduto, è stato notificato da tutti e cinque i componenti presenti.
Il litigio è stato ritenuto grave dal \rRP in quanto ha portato a cinque ore di lavoro inconcludenti in una fase cruciale per il \gloxy{progetto} per il modello di ciclo di vita adottato. Il \rRP dopo aver attentamente ascoltato la versione riferita da ogni componente presente ed aver effettivamente riscontrato l'impossibilità di lavorare assieme per i due componenti coinvolti,  ha valutato l'accaduto e adottato le seguenti misure correttive:
\begin{itemize}
\item Le ore da \rP per uno dei due componenti coinvolti sono state convertite in ore da \rV, in questo modo non potrà interagire direttamente con il resto dei \rPs ma potrà comunque verificare che il lavoro svolto sia corretto ed avanzare eventuali dubbi in merito, giustificandoli opportunamente;
\item Sono state ripianificate le attività delle fasi successive, dove necessario, in modo che i due componenti coinvolti non si trovino a lavorare assieme.
\end{itemize}
\end{itemize}
%Inesperienza del gruppo
\subsubsubsection{Problemi legati all'inesperienza del gruppo}\label{r_inesp}
\begin{itemize}
%\item \textbf{Possibilità di occorrenza:} Alta;
\item \textbf{Pericolosità:} Alta;
\item \textbf{Descrizione:} Questo metodo di lavoro risulta nuovo e richiede capacità di pianificazione e analisi che si sviluppano con l'esperienza, cosa che la maggior parte dei componenti del gruppo non possiede. Inoltre viene richiesto l'uso di software che nessun membro ha mai utilizzato;
\item \textbf{Strategie di rilevazione:} Nel caso un componente del gruppo non riesca a trovare il materiale su cui studiare o non riesca a risolvere un determinato problema, dovrà chiedere aiuto usando la \gloxy{mailing list} o la \gloxy{chat};
\item \textbf{Contromisure:}
\begin{itemize}
\item Ogni componente del gruppo si impegna a studiare il materiale necessario per poter affrontare il \gloxy{progetto} in modo ottimale;
\item In caso di eventuali dubbi emersi durante lo studio il componente dovrà notificare il \rRP in modo da predisporre delle domande volte al chiarimento dei dubbi da presentare al più presto al docente.
\end{itemize}
\item \textbf{Riscontro:}
Sono stati riscontrati alcuni dubbi riguardo gli argomenti studiati durante il corso. \\
Questi sono stati prontamente chiariti dai docenti attraverso mail oppure di persona. Sono stati riscontrati dubbi riguardo lo studio dei \gloxy{framework} e delle tecnologie scelte. Questi sono stati risolti internamente al \gloxy{team} grazie ad una collaborazione continua tra i vari \rPs, che ha permesso di chiarire ogni dubbio rilevato.
\end{itemize}
\subsubsection{Livello organizzativo}\label{lo}
%Errata stima delle ore necessarie
\subsubsubsection{Problemi legati alla stima delle ore}\label{r_stima}
\begin{itemize}
%\item \textbf{Possibilità di occorrenza:} Media;
\item \textbf{Pericolosità:} Alta;
\item \textbf{Descrizione:} I tempi stimati per l'esecuzione di un'attività possono essere sottostimati, provocando un aumento dei costi e ritardi nella consegna;
\item \textbf{Strategie di rilevazione:} Il sistema di \gloxy{ticketing} adottato permette al \rRP di avere sotto controllo lo stato di avanzamento del \gloxy{progetto} e in particolar modo evidenzia le attività che sono in ritardo;
\item \textbf{Contromisure:}
\begin{itemize}
\item Per ogni fase e per tutta la durata del \gloxy{progetto} sono state programmate delle attività volte all'incremento dei documenti in modo che, se si dovessero verificare dei cambiamenti imprevisti, siano già pianificate le tempistiche entro cui attuarli;
\item Quando possibile sono stati inseriti giorni di \gloxy{slack} tra le varie attività, in modo da evitare che un eventuale ritardo influenzi la durata del \gloxy{progetto};
\item Le attività riguardanti le consegne per una revisione sono state pianificate in modo che terminino circa 8 giorni prima della data effettiva di revisione e qualche giorno prima rispetto alla data di consegna effettiva. In questo modo se la data effettiva di consegna viene anticipata rispetto alla data stabilita, non sarà necessaria una nuova pianificazione. Mentre se la stima della data di consegna si rivelerà pessimistica ci saranno più giorni di \gloxy{slack} a disposizione.
\end{itemize}
\item \textbf{Riscontro:}
Si è verificato un errore di pianificazione riguardo la fase di \fAD, in quanto è stata sottostimata. Questo ha portato ad utilizzare tutto lo \gloxy{slack} disponibile ed ha comportato inoltre una compressione della fase successiva per potersi riallineare con la pianificazione e le \gloxy{milestone} fissate dal \rRP.
\end{itemize}
%Errori legati ripianificazione bilancio
\subsubsubsection{Problemi legati alla ripianificazione del bilancio}\label{r_bilancio}
\begin{itemize}
%\item \textbf{Possibilità di occorrenza:} Media;
\item \textbf{Pericolosità:} Alta;
\item \textbf{Descrizione:} a seguito del taglio delle ore previsto dopo la revisione di bilancio effettuata in \fC, sono state ridotte le ore totali di lavoro. Di conseguenza le nuove ore stimate potrebbero essere non sufficienti per il quantitativo di lavoro necessario in particolare per la progettazione e la codifica;
\item \textbf{Strategie di rilevazione:} al \rRP viene notificato un quantitativo maggiore di ore richieste all'adempimento delle attività assegnate da parte dei membri del \gloxy{team}.
Questo può avvenire tramite \gloxy{ticketing} o comunicazione diretta:
\item \textbf{Contromisure:}
\begin{itemize}
\item Viene intensificato il sistema di \textit{\gloxy{ticketing}} con lo scopo di avere una visione dettagliata delle attività e dello stato dei lavoro dei singolo membri del gruppo. In questo modo il lavoro del \rRP viene ulteriormente semplificato e la sua visione delle tempistiche risulterà essere più chiara e precisa.
\item I membri del gruppo si impegnano ulteriormente e assiduamente a lavorare in maniera disciplinata, sistematica e quantificabile al fine di rendere il flusso dei lavori il più fluido possibile e privo di imprevisti;
\item L'occorrenza di questo rischio è direttamente correlata con l'occorrenza degli altri preventivati. La prevenzione di questi ultimi porta come conseguenza diretta la diminuzione della probabilità che si verifichi.
\end{itemize}
\item \textbf{Riscontro:}
Questo rischio e stato riscontrato a seguito della ripianificazione del bilancio e delle ore preventivate per i ruoli durante la \fC.
\end{itemize}
\subsubsection{Livello dei requisiti}\label{lr}
%Analisi dei requisiti fatta male
\subsubsubsection{Problemi legati all'identificazione dei requisiti}\label{r_req}
\begin{itemize}
%\item \textbf{Possibilità di occorrenza:} Alta;
\item \textbf{Pericolosità:} Alta;
\item \textbf{Descrizione:} Durante l'analisi del capitolato è possibile che qualche aspetto del problema non venga colto oppure venga mal interpretato da parte degli \rAs, provocando divergenze tra le aspettative del \gloxy{Proponente} e l'idea del gruppo sul prodotto;
\item \textbf{Strategie di rilevazione:} Durante la fase di \fA e di \fAD si terranno delle riunioni con il \gloxy{Proponente}. Inoltre tutti i documenti prodotti verranno consegnati e valutati dal \gloxy{Committente} ad ogni revisione;
\item \textbf{Contromisure:}
\begin{itemize}
\item Si cercherà di scrivere un documento di \AR che si presti ad essere incrementato, anche sostanzialmente, nelle fasi successive ad \fA. \\ In questo modo si riuscirà ad adattarsi alla tipologia di capitolato scelto;
\item Pianificazione delle riunioni con il \gloxy{Proponente};
\item Pianificazione della fase di \fAD con lo scopo di incrementare i documenti prodotti, in particolare l'\AR, sfruttando le indicazioni del \gloxy{Committente}.
\end{itemize}
\item \textbf{Riscontro:}
Si è verificato una variazione dei requisiti durante la fase di \fPA, in seguito ad una riunione con il \gloxy{Proponente}. \\
Il gruppo è riuscito ad integrare tempestivamente le aggiunte al documento \AR, in quanto esso era stato precedentemente scritto in modo da potersi adattare ad eventi simili, evidentemente insiti del capitolato scelto.
\end{itemize}
\subsubsection{Riepilogo}\label{rplg}
\begin{table}[h]
\begin{center}
\begin{tabular}{|l|c|}
\hline Rischio & Pericolosità \\
\hline
\ref{r_bilancio} Problemi legati alla ripianificazione del bilancio & \textcolor{red}{Alta}\\
\ref{r_framework} Problemi legati ai \gloxy{framework} & {Alta} \\
\ref{r_inesp} Problemi legati all'inesperienza &  {Alta} \\
\ref{r_liti} Problemi tra i componenti del gruppo & Alta \\
\ref{r_stima} Problemi legati alla stima delle ore & Alta \\
\ref{r_tec} Problemi legati alle tecnologie & Alta \\
\ref{r_comp} Problemi legati ai componenti del gruppo & {Media} \\
\ref{r_req} Problemi legati ai requisiti & {Bassa} \\
\ref{r_appweb} Problemi legati ai servizi \gloxy{web} & Bassa \\
\ref{r_hw} Problemi legati alle risorse \gloxy{HW} & Bassa \\
\hline
\end{tabular}
\caption{Riepilogo dei rischi, ordinati per pericolosità}
\end{center}
\end{table}
\FloatBarrier
Evidenziati i rischi verificatisi recentemente o che hanno subito una variazione di ``Pericolosita''.
\subsection{Previsione rischi}
%struttura report:
%\subsubsection{Report fase X}
%\subsubsubsection{Resoconto fase precedente}
%\subsubsubsection{Nuovi rischi individuati}
%Devono essere aggiunti in nella sezione §5.1 e qui deve essere inserito un riferimento e le motivazioni che hanno portato all'individuazione del rischio
%\subsubsubsection{Previsioni incidenza}
%tabella + commenti cambiamenti
\subsubsection{Fase di \fAt}
\subsubsubsection{Resoconto fase precedente}
Essendo la fase iniziale del \gloxy{progetto} non è possibile effettuare un resoconto sull'incidenza o meno dei vari rischi identificati.
\subsubsubsection{Nuovi rischi individuati}
Sono state individuate varie tipologie di rischi, principalmente legati all'inesperienza dei componenti del gruppo.\\
In particolare ne sono state individuate 4 tipologie:
\begin{itemize}
\item \textbf{\ref{lt} Livello tecnologico}: rischi legati ad un'errata scelta delle tecnologie adottate per realizzare il \gloxy{progetto};
\item \textbf{\ref{lp} Livello personale}: rischi legati ai componenti del gruppo e ai rapporti interni;
\item \textbf{\ref{lo} Livello organizzativo}: rischi legati ad una pianificazione errata delle attività;
\item \textbf{\ref{lr} Livello dei requisiti}: rischi legati ad un'analisi dei requisiti errata o non sufficientemente approfondita.
\end{itemize}
\subsubsubsection{Previsioni incidenza}
\begin{table}[h]
\begin{center}
\begin{tabular}{|l|c|}
\hline Rischio & Probabilità \\
\hline
\ref{r_framework} Problemi legati ai \gloxy{framework} & Alta \\
\ref{r_inesp} Problemi legati all'inesperienza & Alta \\
\ref{r_req} Problemi legati ai requisiti & Alta \\
\ref{r_tec} Problemi legati alle tecnologie & Media \\
\ref{r_stima} Problemi legati alla stima delle ore & Media \\
\ref{r_comp} Problemi legati ai componenti del gruppo & Media \\
\ref{r_appweb} Problemi legati ai servizi \gloxy{web} & Media \\
\ref{r_liti} Problemi tra i componenti del gruppo & Bassa \\
\ref{r_hw} Problemi legati alle risorse \gloxy{HW} & Bassa \\
\hline
\end{tabular}
\caption{Probabilità d'incidenza dei rischi, fase di \fAt}
\end{center}
\end{table}
\FloatBarrier
\subsubsubsection{Mitigazione rischi}\label{mRisk1}
Di seguito viene spiegato come verranno mitigati i rischi più critici previsti e saranno elencate le strategie atte a limitarne i possibili danni.
\begin{itemize}
\item{Problemi legati ai requisiti:} vista la fase in cui ci si trova è auspicabile che il problema non sia ancora pienamente compreso. Al fine di limitare questa possibilità è necessario che vengano effettuati incontri con il \gloxy{Proponente} in modo che siano ben delineate le funzionalità dell'applicativo e non si verifichino rischi di questa tipologia.
\item{Problemi legati ai Framework:} può rivelarsi utile la creazione di piccoli prototipi, che permettano di comprendere il funzionamento di alcuni aspetti del \gloxy{framework}. Il tempo speso complessivamente nella creazione di tali prototipi risulta essere decisamente minore rispetto a quello richiesto per un cambo di \gloxy{framework}.
\end{itemize}
\subsubsection{Fase di \fADt}
\subsubsubsection{Resoconto fase precedente}
Durante la fase precedente si sono verificate delle discussioni tra i vari \rAs del gruppo, risolte mediante votazioni che hanno permesso di giungere a delle conclusioni in modo efficace. \\
\subsubsubsection{Nuovi rischi individuati}
Non sono stati individuati altri rischi poiché le attività svolte in questa fase sono simili a quelle svolte nella fase precedente.
\subsubsubsection{Previsioni incidenza}
\begin{table}[h]
\begin{center}
\begin{tabular}{|l|c|}
\hline Rischio & Probabilità \\
\hline
\ref{r_framework} Problemi legati ai \gloxy{framework} & Alta \\
\ref{r_inesp} Problemi legati all'inesperienza & Alta \\
\ref{r_liti} Problemi tra i componenti del gruppo & \textcolor{YellowOrange}{Media} \\
\ref{r_tec} Problemi legati alle tecnologie & Media \\
\ref{r_stima} Problemi legati alla stima delle ore & Media \\
\ref{r_comp} Problemi legati ai componenti del gruppo & Media \\
\ref{r_appweb} Problemi legati ai servizi \gloxy{web} & Media \\
\ref{r_req} Problemi legati ai requisiti & \textcolor{OliveGreen}{Bassa} \\
\ref{r_hw} Problemi legati alle risorse \gloxy{HW} & Bassa \\
\hline
\end{tabular}
\caption{Probabilità d'incidenza dei rischi, fase di \fADt}
\end{center}
\end{table}
\FloatBarrier
\begin{itemize}
\item Essendosi verificate delle discussioni nella fase precedente, la probabilità che si verifichino ulteriori discussioni e divergenze tra i vari componenti del gruppo aumenta;
\item Gli esiti della \RR, riguardanti il documento \AR, sono stati positivi. Di conseguenza la probabilità che le future attività di analisi siano svolte in modo errato o non sufficientemente approfondito diminuisce.
\end{itemize}
\subsubsubsection{Mitigazione rischi}\label{mRisk2}
Di seguito viene spiegato come verranno mitigati i rischi più critici previsti verranno elencate le strategie atte a limitarne i possibili danni.
\begin{itemize}
\item{Problemi legati ai Framework:} può rivelarsi utile la creazione di piccoli prototipi, che permettano di comprendere il funzionamento di alcuni aspetti del \gloxy{framework}. Il tempo speso complessivamente nella creazione di tali prototipi risulta essere decisamente minore rispetto a quello richiesto per un cambo di \gloxy{framework}.
\item{Problemi legati all'inesperienza:} è necessario investire più tempo personale in ore di studio mirato a colmare le lacune che i vari membri del gruppo presentano in alcuni campi.
\item{Problemi tra i componenti del gruppo:} i rischi relativi alle problematiche del gruppo sono stati risolti con discreti risultati, cercando un punto d'incontro tra le parti coinvolte.
\end{itemize}
\subsubsection{Fase di \fPAt}\label{ppa}
\subsubsubsection{Resoconto fase precedente}
Durante la fase precedente si sono verificati degli errori nella stima delle ore per il completamento delle attività pianificate. Ciò ha provocato l'esaurimento del tempo di \gloxy{slack} a disposizione per quella fase, oltre che una compressione dei tempi previsti per la fase successiva. Al fine di rispettare le ore preventivate, i ritardi causati dalla fase di \fAD hanno comportato un maggiore carico di lavoro giornaliero nella fase di \fPA. Si sono verificati dei rischi legati all'inesperienza dei componenti in alcune attività legate ai processi di supporto, e la gestione di tali rischi è stata agevolata grazie al supporto del \gloxy{committente}.
\subsubsubsection{Nuovi rischi individuati}
In questa fase non sono stati individuati nuovi rischi.
\subsubsubsection{Previsioni incidenza}
\begin{table}[h]
\begin{center}
\begin{tabular}{|l|c|}
\hline Rischio & Probabilità \\
\hline
\ref{r_framework} Problemi legati ai \gloxy{framework} & Alta \\
\ref{r_inesp} Problemi legati all'inesperienza & \textcolor{red} {Alta} \\
\ref{r_stima} Problemi legati alla stima delle ore & \textcolor{red}{Alta} \\
\ref{r_liti} Problemi tra i componenti del gruppo & Media \\
\ref{r_tec} Problemi legati alle tecnologie & Media \\
\ref{r_comp} Problemi legati ai componenti del gruppo & Media \\
\ref{r_appweb} Problemi legati ai servizi \gloxy{web} & Media \\
\ref{r_req} Problemi legati ai requisiti & Bassa \\
\ref{r_hw} Problemi legati alle risorse \gloxy{HW} & Bassa \\
\hline
\end{tabular}
\caption{Probabilità d'incidenza dei rischi, fase di \fPAt}
\end{center}
\end{table}
\FloatBarrier
\begin{itemize}
\item Rischi legati all'inesperienza potrebbero verificarsi anche in futuro, seppur con una minor incidenza sulle attività legate ai processi di supporto poiché i dubbi più incidenti sono stati chiariti;
\item Rischi legati alla stima delle ore di pianificazione per una determinata attività potrebbero verificarsi in futuro. Questi rischi vengono quindi valutati come altamente pericolosi in quanto potrebbero avere effetti critici, influendo negativamente sul \gloxy{progetto}.
\end{itemize}
\subsubsubsection{Mitigazione rischi}\label{mRisk3}
Di seguito viene spiegato come verranno mitigati i rischi più critici previsti e saranno elencate le strategie atte a limitarne i possibili danni.
\begin{itemize}
\item{Problemi relativi ai framework:} da mitigare con l'investimento di più ore di studio personali. Si presuppone che il \gloxy{team} abbia acquisito un discreto livello di conoscenza del \gloxy{framework}, poiché sono stati creati svariati prototipi per comprenderne meglio il funzionamento.
\item{Problemi relativi alla stima delle ore:} da mitigare intensificando l'uso del sistema di \gloxy{ticketing}. Ciò garantisce al \rRP un maggiore controllo dell'avanzamento del lavoro, che permette una migliore gestione dello stesso, e comporta una diminuzione della probabilità di ritardi non preventivati.
\end{itemize}
\subsubsection{Fase di \fPDt}\label{ppd}
\subsubsubsection{Resoconto fase precedente}
Durante la fase precedente si sono verificati diversi rischi, tra cui litigi interni che sono stati opportunamente gestiti dal \rRP.
In seguito all'assenza di un componente per due riunioni interne si è scelto di fissare un limite superiore di al massimo due assenze, giustificabili solo se si tratta di casi eccezionali. Il tempo impiegato nello studio dei \gloxy{framework} ha comportato uno slittamento in avanti nella pianificazione di alcune attività. In questo caso si è utilizzato lo \gloxy{slack} disponibile evitando quindi ritardi. Sono stati riscontrati alcuni problemi legati alla variazione dei requisiti in seguito ad un incontro con il \gloxy{proponente}: in questo caso, il rischio è stato mitigato in modo efficace riuscendo a modificare tempestivamente il documento di \AR, senza procurare ritardi ad altre attività.
\subsubsubsection{Nuovi rischi individuati}
In questa fase non sono stati individuati nuovi rischi.
\subsubsubsection{Previsioni incidenza}
\begin{table}[h]
\begin{center}
\begin{tabular}{|l|c|}
\hline Rischio & Pericolosità \\
\hline
\ref{r_liti} Problemi tra i componenti del gruppo & \textcolor{red} {Alta} \\
\ref{r_comp} Problemi legati ai componenti del gruppo & \textcolor{red} {Alta} \\
\ref{r_framework} Problemi legati ai \gloxy{framework} & \textcolor{red} {Alta} \\
\ref{r_stima} Problemi legati alla stima delle ore & Alta \\
\ref{r_inesp} Problemi legati all'inesperienza & Alta \\
\ref{r_tec} Problemi legati alle tecnologie & Alta \\
\ref{r_req} Problemi legati ai requisiti & \textcolor{YellowOrange} {Media} \\
\ref{r_appweb} Problemi legati ai servizi \gloxy{web} & \textcolor{OliveGreen}{Bassa} \\
\ref{r_hw} Problemi legati alle risorse \gloxy{HW} & Bassa \\
\hline
\end{tabular}
\caption{Probabilità di incidenza dei rischi, fase di \fPDt}
\end{center}
\end{table}
\FloatBarrier
\begin{itemize}
\item Rischi legati ai litigi interni potrebbero verificarsi in futuro ed avere effetti disastrosi. Infatti, se si perdesse l'affiatamento creato, non si riuscirebbe ad avanzare come previsto nei livelli di miglioramento descritti da SEMAT. Di conseguenza non si riuscirebbe a migliorare, in particolare nelle dimensioni ``\gloxy{team}'' e ``way of working''. L'aspetto di relazioni di fiducia e di rispetto che si instaurano all'interno di un gruppo è stato valutato con un grado di importanza alto dal \rRP. Esso permette ai componenti di lavorare  in modo unito e quindi di focalizzare le proprie energie su un obbiettivo comune, aumentando di molto le probabilità di successo nel lungo periodo;
\item La pericolosità di problemi relativi a singoli componenti del gruppo è stata aggiornata ad alta. Se si riverificasse nuovamente un rischio simile in una situazione più critica (ad esempio a ridosso di una \gloxy{milestone} o di una consegna), in cui le tempistiche non permettessero margini di azione da parte del \rRP, verrebbe inevitabilmente compromesso il \gloxy{progetto}, portando a conseguenze catastrofiche per l'intero \gloxy{team};
\item Rischi legati ai \gloxy{framework} possono presentarsi in questa fase nel caso fossero stati commessi errori di valutazione nella fase precedente;
\item Rischi legati alla variazione dei requisiti non dovrebbero di norma riverificarsi. Nel caso si avverassero, verrebbero trattati come nella fase precedente, cercando di aggiornare subito il documento e modellando una soluzione semplice per il problema.
\item I servizi \gloxy{web} scelti hanno risposto bene alle esigenze del gruppo senza provocare alcun tipo di problema. Per questo motivo la pericolosità del rischio è stata aggiornata a bassa.
\end{itemize}
\subsubsubsection{Mitigazione rischi}\label{mRisk4}
Di seguito viene spiegato come verranno mitigati i rischi più critici previsti e saranno elencate le strategie atte a limitarne i possibili danni.
\begin{itemize}
\item{Problemi tra i componenti del gruppo:} considerando le esperienze precedenti, il rischio che si verifichino discussioni tra i componenti del gruppo è molto alto. A tale proposito, si cercherà di trovare dei punti d'incontro tra le parti coinvolte, in modo da appianare le divergenze. Nel caso di situazioni gravi, il \rRP potrà decidere di allontanare i membri problematici dalle rispettive aree di lavoro, in modo da diminuirne l'interazione.
\item{Problemi legati ai componenti del gruppo:} visto quanto accaduto nella fase precedente, per tale rischio è stato alzato il livello di occorrenza. La strategia adottata fino a questo momento è stata molto buona. Dunque, in caso di nuova incidenza, si procederà ad una redistribuzione del carico di lavoro tra i membri del \gloxy{team}, cercando di mantenere equità nelle ore di lavoro.
\end{itemize}
\subsubsection{Fase di \fCt}\label{ppd}
\subsubsubsection{Resoconto fase precedente}
Durante la fase precedente si sono verificati rischi legati principalmente allo studio dei \gloxy{framework} e all'inesperienza dei componenti del \gloxy{team}. La conseguente creazione di prototipi ha portato a consumare lo \gloxy{slack} ricavato dalla fase precedente, oltre a quello pianificato per questa fase. Si è scelto infatti di consegnare in ingresso a \RP una progettazione contenente le varie componenti descritte nel dettaglio. Per fare ciò è stato necessario uno studio impegnativo dei \gloxy{framework} scelti. Questo rischio era stato preventivato durante la pianificazione di \fA e discusso adeguatamente tra i membri del gruppo. Si è scelto comunque di consegnare RPmax in quanto si è cercato diminuire il carico di lavoro della fase successiva, anche in vista della correzione dei documenti(sottostimata in fase di \fAD).
Si è rivelata cruciale la scelta attuata nella precedente fase riguardo i rischi interni al \gloxy{team}: in questa fase sono stati risolti i problemi precedenti ed i vari componenti del \gloxy{team} sono riusciti a lavorare assieme, evitando litigi inutili.
\subsubsubsection{Nuovi rischi individuati}
In questa fase non sono stati individuati nuovi rischi.
\subsubsubsection{Previsioni incidenza}
\begin{table}[h]
\begin{center}
\begin{tabular}{|l|c|}
\hline Rischio & Pericolosità \\
\hline
\ref{r_inesp} Problemi legati all'inesperienza & \textcolor{red} {Alta} \\
\ref{r_framework} Problemi legati ai \gloxy{framework} & \textcolor{red} {Alta} \\
\ref{r_liti} Problemi tra i componenti del gruppo & Alta \\
\ref{r_stima} Problemi legati alla stima delle ore & Alta \\
\ref{r_tec} Problemi legati alle tecnologie & Alta \\
\ref{r_comp} Problemi legati ai componenti del gruppo & \textcolor{YellowOrange} {Media} \\
\ref{r_req} Problemi legati ai requisiti & \textcolor{OliveGreen}{Bassa} \\
\ref{r_appweb} Problemi legati ai servizi \gloxy{web} & Bassa \\
\ref{r_hw} Problemi legati alle risorse \gloxy{HW} & Bassa \\
\hline
\end{tabular}
\caption{Probabilità di incidenza dei rischi, fase di \fCt}
\end{center}
\end{table}
\FloatBarrier
\begin{itemize}
\item Rischi legati ai litigi interni potrebbero comunque verificarsi in futuro, nonostante siano stati mitigati;
\item La pericolosità dei problemi relativi a singoli componenti del gruppo è stata aggiornata a ``media''. In questa fase è stata riscontrata un'ottima risposta da parte di tutti i componenti: ognuno ha dimostrato la piena disponibilità ed un notevole impegno nel portare a termine le attività assegnategli;
\item Rischi legati ai \gloxy{framework} si sono concretizzati durante la fase precedente e sono stati mitigati grazie alla collaborazione tra i \rPs, che sono riusciti a trovare una soluzione per ogni nuovo problema o dubbio emerso;
\item Non si sono verificati rischi legati alla variazione dei requisiti, in quanto la comunicazione con \proponente ha chiarito qualsiasi possibile incomprensione o dubbio riscontrato. La pericolosità dei problemi legati ai requisiti è stata aggiornata a ``bassa''.
\end{itemize}
\subsubsubsection{Mitigazione rischi}\label{mRisk5}
Di seguito viene spiegato come verranno mitigati i rischi più critici previsti e saranno elencate le strategie atte a limitarne i possibili danni.
\begin{itemize}
\item{Problemi tra i componenti del gruppo:} considerando le esperienze precedenti, il rischio che si verifichino discussioni tra i componenti del gruppo è molto alto. A tale proposito, si cercherà di trovare dei punti d'incontro tra le parti coinvolte, in modo da appianare le divergenze. Nel caso di situazioni gravi, il \rRP potrà decidere di allontanare i membri problematici dalle rispettive aree di lavoro, in modo da diminuirne l'interazione.
\item{Problemi legati ai componenti del gruppo:} La strategia adottata fino a questo momento è stata molto buona. Dunque, in caso di nuova incidenza, si procederà ad una redistribuzione del carico di lavoro tra i membri del \gloxy{team}, cercando di mantenere equità nelle ore di lavoro.
\end{itemize}
\subsubsection{Fase di \fVVt}\label{pvv}
\subsubsubsection{Resoconto fase precedente}
Durante la fase precedente si sono verificati alcuni dei rischi che erano stati preventivati in \fCt.
In particolare andando in ordine di gravità sono state rilevate diverse problematiche, ognuna delle quali è stata gestita secondo quanto descritto nella sezione di Analisi dei Rischi.
\begin{itemize}
\item \textbf{Problematiche relative alla ripianificazione di bilancio (vedi: \ref{r_bilancio})}: a seguito della ripianificazione di bilancio per rientrare nei costi concordati col \gloxy{Committente} è stata individuata questa nuova tipologia di rischio.
In particolare per la fase di \fC non si è verificata nessuna problematica grazie ad alcuni accorgimenti.
I tagli effettuati alle ore preventivate hanno riguardato principalmente l'Amministratore, l'Analista e il \rRP.
Sebbene i tagli siano stati sostanziali, i nuovi tempi stabiliti per il lavoro sono stati sufficienti a coprire il carico di attività dettato, sottostimando leggermente le ore da \rRP.
Dato il fatto che la maggior parte delle ore lavorative riguardava la codifica e progettazione, è stato scelto di non diminuire le ore da Programmatore e Progettista, stimando in maniera corretta il quantitativo di lavoro.
\item \textbf{Problematiche tra membri del \gloxy{team}:} hanno costituito sicuramente il rischio con gli effetti più gravi tra quelli preventivati. Lo scontrarsi tra i membri del gruppo in alcune situazione ha generato un clima poco affine al lavoro di gruppo e tensioni interne tra i componenti. In alcuni casi le situazioni si sono risolte tramite il semplice intervento di una parte mediatrice ma nei casi più gravi si è dovuto ricorrere alla separazione tra i componenti assegnando loro aree di lavoro diverse in modo da diminuire l'interazione possibile. Questo ha a volte generato ritardi sul flusso di lavoro che è comunque stato gestito in maniera ottimale grazie all'esperienza acquisita nelle revisioni precedenti e grazie alla pianificazione generale che il gruppo sta adottando per avere un tracciamento dei lavori in corso.
Questo tipo di problematica è la causa principale che ha portato ad uno stallo per quanto riguarda la progressione del SEMAT in ``Way of Working''.
\item \textbf{Problematiche Servizi \gloxy{Web}}: alcune problematiche che sono occorse con l'utilizzo di \pragmadb. Nello specifico cause esterne hanno reso impossibile l'accesso per diverse ore impedendo il lavoro di alcuni componenti del gruppo durante l'attività di tracciamento. Le misure adottate in questo caso per risolvere il problema si sono concretizzate in una semplice segnalazione all'ente specifico, esterno al gruppo.
\item \textbf{Problematiche dei singoli componenti}: rispetto alle previsioni fatte son occorse meno del previsto.
Tuttavia si sono verificati in alcuni casi delle situazioni in cui i problemi personali di un singolo componente hanno costretto una ridistribuzione del carico dei lavoro tra i membri in modo da venire incontro alle esigenze personali di ognuno. In ogni caso si è proceduto con una distribuzione omogenea delle attività lavorative ed il \rRP si è assicurato che non ci fosse alcun tipo di squilibrio sotto ogni aspetto della pianificazione.
\end{itemize}
\subsubsubsection{Nuovi rischi individuati}
È stato individuato un nuovo possibile rischio:
\begin{itemize}
\item Rischio legato alla ripianificazione di bilancio (vedi: \ref{r_bilancio}).
\end{itemize}
\subsubsubsection{Previsioni incidenza}\label{prevRisk}
\begin{table}[h]
\begin{center}
\begin{tabular}{|l|c|}
\hline Rischio & Probabilità \\
\hline
\ref{r_bilancio} Problemi legati alla ripianificazione del bilancio & \textcolor{red}{Alta}\\
\ref{r_liti} Problemi tra i componenti del gruppo & \textcolor{red} {Alta} \\
\ref{r_inesp} Problemi legati all'inesperienza & {Alta} \\
\ref{r_stima} Problemi legati alla stima delle ore & {Alta} \\
\ref{r_framework} Problemi legati ai \gloxy{framework} & \textcolor{YellowOrange} {Media} \\
\ref{r_appweb} Problemi legati ai servizi \gloxy{web} & \textcolor{YellowOrange} {Media} \\
\ref{r_comp} Problemi legati ai componenti del gruppo & Media \\
\ref{r_tec} Problemi legati alle tecnologie & Media \\
\ref{r_req} Problemi legati ai requisiti & Bassa \\
\ref{r_hw} Problemi legati alle risorse \gloxy{HW} & Bassa \\
\hline
\end{tabular}
\caption{Probabilità d'incidenza dei rischi, fase di \fVVt}
\end{center}
\end{table}
\FloatBarrier
\begin{itemize}
\item La ripianificazione delle ore di lavoro con il loro conseguente taglio al fine di rientrare nel budget prestabilito porta il rischio che nella fase successiva non ci sia abbastanza tempo per svolgere tutte le attività richieste a seguito di eventuali ritardi sui lavori.
La probabilità di occorrenza viene quindi impostata su ``alta'' in quanto non prevedere il verificarsi di alcun rischio è altamente improbabile.
Sarà quindi cruciale la gestione delle attività e come verranno mitigate la situazione anomale che verranno a verificarsi.
\item I rischi legati all'interazione tra i membri del gruppo continuano a rimanere su pericolosità ``alta''.
In previsione del prossimo periodo di lavoro si stima che i litigi interni continueranno a risultare una problematica da affrontare. Per quanto si sia cercato di mitigare questo tipo di situazioni e siano sensibilmente migliorate il rischio che si verifichino è molto alto.
\item Le problematiche relative ai servizi \gloxy{web} ed in particolare al funzionamento di \pragmadb sono state aggiornate a ``media''. Durante il periodo di lavoro precendente l'impossibilità di accedere a \pragmadb ha portato a ritarti, seppur leggeri, sui lavori. In vista della prossima revisione l'utilizzo del database interno sarà intenso, pertanto la possibilità del verificarsi di problematiche potrebbe aumentare.
\item Le problematiche riguardanti  i \gloxy{framework} utilizzati è scesa a ``media''. Superato lo scoglio iniziale dove si è entrati in contatto per la prima volta con le tecnologie utilizzate, i componenti stanno cominciando ad acquisire una buona manualità e confidenza con quanto utilizzato. Di conseguenza si prevede una leggera diminuzione delle problematiche relative a questo aspetto, tenendo comunque in considerazione il fatto che anche se si sone verificate precedentemente sono sempre state mitigate riducendo l'impatto sui lavori.
\end{itemize}
\subsubsubsection{Mitigazione rischi}\label{mRisk6}
Di seguito viene spiegato come verranno mitigati i rischi più critici previsti e saranno elencate le strategie atte a limitarne i possibili danni.
\begin{itemize}
\item {Problematiche derivate dalla revisione di bilancio:} i rischi introdotti dalla revisione di bilancio sono molto concreti e auspicabili nella prossima fase. Nel caso si dovessero verificare si sono pensati ad alcuni accorgimenti per mitigare gli effetti. Una possibile taglio alle ore non necessarie nella prossima fase può essere attuato nella necessità che qualche ruolo abbia bisogno di più ore lavorative per poter compiere quanto assegnato. Il margine comunque rimane molto lieve in quanto sono già stati fatti onerosi tagli. La buona gestione attuata durante la \fC e i risparmi in termini di \euro che sono stati ottenuti costituiscono un fondo dal quale attingere nel caso fosse necessario aggiungere qualche ora per completare i lavori. Si stima che questo sia possibile per quanto riguarda in particolare la Codifica.
\item Per quanto riguarda gli altri rischi non si suono individuate nuove strategie di mitigazione differenti da quelle utilizzate e raffinate fino a questo momento. SI considera buona la gestione dei rischi nel caso occorrano e i loro danni vengono limitati quanto più possibile, ove possibile.
\end{itemize}

