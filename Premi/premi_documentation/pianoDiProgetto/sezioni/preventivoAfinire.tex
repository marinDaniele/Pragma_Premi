\section{Preventivo a finire}\label{CaF}
Di seguito verranno riportati i preventivi a finire che si sono generati al termine delle varie fasi considerando i relativi consuntivi.
\subsection{Fase di \fPAt}
In questa fase il bilancio risulta \textbf{in positivo} con un risparmio complessivo di \textbf{30\euro} rispetto a ciò che era stato precedentemente preventivato. Questo risparmio potrà essere utilizzato nelle fasi successive per coprire ulteriori ore di verifica.
\subsection{Fase di \fPDt}
In questa fase il bilancio risulta \textbf{in negativo} con un deficit complessivo di \textbf{4\euro} rispetto a ciò che era stato precedentemente preventivato. Il bilancio complessivo risulta comunque \textbf{in positivo} di \textbf{26\euro}, grazie ai \textbf{30\euro} risparmiati durante la fase precedente. Questo risparmio potrà essere investito nuovamente nelle fasi successive, ove necessario.
\subsection{Fase di \fCt}
Alla fine della fase di \fC il bilancio risulta \textbf{in positivo} di \textbf{\euro81} rispetto a quanto previsto.
Questo risparmio deriva dal consuntivo della fase di \fC, al quale va aggiunto il residuo della fase precedente.
In un primo momento si era pensato di reinvestire nell'immediato i soldi risparmiati in attività di verifica ma, poiché per questa fase la sua percentuale risulta buona, si è deciso di conservarli in ottica della fase di \fVV con le motivazione indicate di seguito. \newline
In relazione a quanto scritto riguardo i rischi preventivati, è possibile che siano richieste più ore da \rRP per mitigare i contrasti interni fra i membri del \gloxy{team}.\newline
Inoltre, si assume che possano essere necessarie più ore di codifica per implementare le funzionalità finali dell'applicativo, correggere eventuali \gloxy{bug} e implementare i test mancanti.\newline
In generale, sarà fondamentale mitigare i rischi nella prossima fase al fine di rientrare nelle ore di lavoro preventivate.
Dunque, si ritiene necessario lasciare dei margini per coprire eventuali ore utili al completamento del \gloxy{progetto}.
Nel caso tali ipotesi non si verificassero è comunque lecito assumere che le ore vengano investite in verifica e di codifica per soddisfare ulteriori requisiti individuati. Tutto ciò al fine di ottenere un prodotto che rispetti gli standard di qualità prefissati, e possa offrire all'utente finale la migliore esperienza possibile.
