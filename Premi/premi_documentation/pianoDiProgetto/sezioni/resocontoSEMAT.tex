\section{Resoconto SEMAT}\label{resocontoSEMAT}
\subsection{\fPAt}\label{resocontoSEMAT1}
\subsubsection{Obbiettivi preventivati}
\begin{itemize}
\item \textit{Opportunity:} Value Established;
\item \textit{\gloxy{Stakeholders}:} Involved;
\item \textit{Requirements:} Acceptable;
\item \textit{Software system:} Demonstrable;
\item \textit{\gloxy{Team}:} Collaborating;
\item \textit{Work:} Under Control;
\item \textit{Way of working:} In Place;
\end{itemize}
\subsubsection{Obbiettivi raggiunti}
\begin{itemize}
\item \textit{Opportunity:} è stato raggiunto un livello collocabile tra ``Value Established'' e ``Viable'' in quanto è stata individuata una soluzione ed il \gloxy{team}, in vista della fase di \fC, ha individuato il suo costo. Tuttavia le linee guida date dalla soluzione individuata non sono ancora così specifiche da permettere lo sviluppo e la consegna del prodotto al cliente. Infine i rischi verificatisi sono stati gestiti adeguatamente dal \gloxy{team}, grazie ad una pianificazione preventiva prevista per far fronte ad essi;
\item \gloxy{Stakeholders}: tale livello è stato raggiunto successivamente agli incontri avvenuti con il \gloxy{proponente} che ha aiutato il \gloxy{team} nel ricercare la tipologia di soluzione più adatta per il problema individuato. Questo ha portato ad una variazione dei requisiti , come indicato in §\ref{ppd};
%Scrivere dopo incontro con Zucchetti
\item \textit{Requirements:} il livello è stato raggiunto con successo dal \gloxy{team}. La possibilità di variare i requisiti diviene molto bassa, in quanto essi sono stati discussi adeguatamente con il \gloxy{proponente} e confermati dal \gloxy{committente};
\item \textit{Software system:} il \gloxy{team} non ha raggiunto questo livello in quanto è stata definita l'architettura dell'applicativo solamente ad alto livello. \`E stato quindi raggiunto il livello ``Architecture Selected'';
%Scrivere dopo incontro con Zucchetti
\item \gloxy{Team}: Purtroppo a causa di litigi interni il livello preventivato non è stato raggiunto. Il livello attuale del \gloxy{team} risulta collocabile tra ``Formed'' e ``Collaborating''.
%Il \rRP è riuscito a coordinare efficacemente le varie attività in modo che le perdite di tempo fossero completamente rimosse.
\item \textit{Work:} è stato possibile gestire i rischi che si sono presentati. Grazie ad un'attenta riformulazione del \PQ è stato possibile fissare degli obbiettivi di qualità chiari e misurabili. In questo modo il \gloxy{team} ha potuto stimare il grado di progresso raggiunto rispetto agli obbiettivi prefissati nella presente fase. Infine è stato incrementato il \PP permettendo al \rRP di avere un controllo maggiore su tutte le attività ed una visione attualizzata della situazione corrente, soprattutto riguardo ai rischi;
\item \textit{Way of working:} questo obbiettivo non è stato totalmente raggiunto in quanto alcuni membri non aderiscono sempre alle norme decise e fissate nel documento \NP. Il livello raggiunto è quindi ``In Use''.
\end{itemize}
\subsection{\fPDt}\label{resocontoSEMAT2}
\subsubsection{Obbiettivi preventivati}
\begin{itemize}
\item \textit{Opportunity:} Value Established;
\item \gloxy{Stakeholders}: In Agreement;
\item \textit{Requirements:} Acceptable;
\item \textit{Software system:} Usable;
\item \gloxy{Team}: Collaborating;
\item \textit{Work:} Under Control;
\item \textit{Way of working:} In Place;
\end{itemize}
\subsubsection{Obbiettivi raggiunti}
\begin{itemize}
\item \textit{Opportunity:} è stato raggiunto un livello collocabile tra ``Value Established'' e ``Viable'' in quanto è stata individuata una soluzione ed il \gloxy{team}, in vista della fase di \fC, ha individuato il suo costo. Tuttavia le linee guida date dalla soluzione individuata non sono ancora così specifiche da permettere lo sviluppo e la consegna del prodotto al cliente.
\item \gloxy{Stakeholders}:  tale livello è stato raggiunto successivamente agli incontri avvenuti con il \gloxy{proponente} dove sono stati accordati i requisiti minimi e prioritari che l'applicativo prodotto dovrà rispettare;
\item \textit{Requirements:}  il livello è stato raggiunto con successo dal \gloxy{team}. La possibilità di variare i requisiti da questa fase in poi diviene molto bassa, in quanto essi sono stati discussi adeguatamente con il \gloxy{proponente} e confermati dal \gloxy{committente};
\item \textit{Software system:}il \gloxy{team} ha raggiunto questo livello in quanto è stata definita l'architettura dell'applicativo sia ad alto livello che nel dettaglio, grazie anche agli incontri con il \gloxy{proponente}. Tuttavia l'applicativo non è ancora stato implementato, impedendo il raggiungimento del livello successivo;
\item \gloxy{Team}: dopo gli eventi accaduti durante la fase di \fPA il \gloxy{team} è diventato più coeso. La migliorata interazione tra i membri del gruppo ha portato ad una maggiore produttività e si è effettivamente raggiunto il livello preventivato;
\item \textit{Work:} è stato possibile gestire i rischi che si sono presentati. Anche in questa fase il livello risulta ``Under Control'';
\item \textit{Way of working:} questo livello è stato raggiunto grazie ad una migliore interazione tra i membri del gruppo e ad una comune decisione di seguire pedissequamente le norme e regole fissate nel documento \NP.
\end{itemize}
\subsection{\fCt}\label{resocontoSEMAT3}
\subsubsection{Obbiettivi preventivati}
\begin{itemize}
\item \textit{Opportunity:} Viable;
\item \gloxy{Stakeholders}: Satisfied For Deployment;
\item \textit{Requirements:} Fullfilled;
\item \textit{Software system:} Usable;
\item \gloxy{Team}: Collaborating;
\item \textit{Work:} Under Control;
\item \textit{Way of working:} Working Well.
\end{itemize}
\subsubsection{Obbiettivi raggiunti}
\begin{itemize}
\item \textit{Opportunity:} è stato raggiunto il livello ``Viable'' in quanto è stata individuata una soluzione ed il suo costo. Le linee guida date dalla soluzione individuata hanno permesso un primo sviluppo del prodotto.
La mancata presenza di nuovi rischi e il consolidamento di quelli preventivati ha fatto in modo che quelli verificatisi siano gestibili senza particolari problematiche.
\item \gloxy{Stakeholders}:  tale livello è stato raggiunto solamente in parte successivamente agli incontri avvenuti con il \gloxy{proponente} dove sono stati accordati i requisiti minimi e prioritari che l'applicativo prodotto dovrà rispettare.
I requisiti da soddisfare sono ben chiari e definiti.
Non è stato possibile raggiungere completamente il livello successivo in quanto manca il riscontro del \gloxy{proponente} sulle prime versioni prodotte dell'applicativo;
\item \textit{Requirements:}  il livello è stato raggiunto con successo dal \gloxy{team}. La possibilità di variare i requisiti da questa fase in poi diviene molto bassa, in quanto essi sono stati discussi adeguatamente con il \gloxy{proponente} e confermati dal \gloxy{committente};
\item \textit{Software system:} il \gloxy{team} ha raggiunto un livello intermedio tra ``Demostrable'' e ``Usable''.
L'architettura è stata definita e consolidata sia ad alto che a basso livello e questo ha permesso una prima realizzazione del prodotto, il quale soddisfa le funzionalità più basilari.
Tuttavia, quanto realizzato non permette di raggiungere il livello ``Usable''.
Caratteristiche necessarie al fine del raggiungimento di tale livello sono che il prodotto sia usabile e abbia delle determinate caratteristiche di qualità precedentemente fissate.
In questa prima versione prodotta, il livello di test effettuati copre solamente il corretto funzionamento dell'applicativo. Ciò è insufficiente pel determinare che il prodotto soddisfi tutte le funzionalità richieste e risulti usabile da un utente esterno;
\item \gloxy{Team}: il \gloxy{team} risulta essersi stabilizzato sul livello ``Collaborating''.
La produttività è ottima anche grazie all'efficacia del \rRP nel gestire le incomprensioni e le discussioni che potrebbero portare a eventuali ritardi e perdite di tempo.
La comunicazione è molto buona sotto due aspetti importanti:
\begin{itemize}
\item a livello di coordinazione, grazie ad un utilizzo costante e intenso del sistema di \gloxy{ticketing}
\item a livello di discussioni e ricerca delle soluzioni ad eventuali problematiche che possono sorgere durante il lavoro di gruppo.
\end{itemize}
Tuttavia non si riesce ad avere un incremento per quanto riguarda questo livello per la seguente motivazione: sporadicamente si verificano situazioni dove non si riesce ad anteporre gli obbiettivi e il bene del gruppo rispetto a convinzioni o pareri personali, anche a fronte di tutte le misure adottate per prevenire rischi di questa tipologia.
\item \textit{Work:} il livello raggiunto dal \gloxy{team} risulta collocato in ``Under Control''. Il flusso di lavoro è costante e ben organizzato. Non sono stati individuati nuovi rischi rispetto la fase precedente e l'esperienza delle revisioni passate ha consentito la gestione di eventuali imprevisti o ritardi senza un aggravarsi eccessivo del carico di lavoro sui singoli membri.
Le attività e il lavoro prodotto sono costantemente tracciati e permettono di fissare \texttt{\gloxy{milestone}} che risultano attendibili nelle tempistiche date.
\item \textit{Way of working:} questo livello è stato raggiunto grazie ad una migliore interazione tra i membri del gruppo e ad una comune decisione di seguire pedissequamente le norme e regole fissate nel documento \NP.
\end{itemize}
\subsection{\fVVt}\label{resocontoSEMAT4}
\subsubsection{Obbiettivi preventivati}
\begin{itemize}
\item \textit{Opportunity:} Addressed;
\item \gloxy{Stakeholders}: Satisfied For Deployment;
\item \textit{Requirements:} Fullfilled;
\item \textit{Software system:} Operational;
\item \gloxy{Team}: Performing;
\item \textit{Work:} Concluded;
\item \textit{Way of working:} Retired;
\end{itemize}
\subsubsection{Obbiettivi raggiunti}
\begin{itemize}
\item \textit{Opportunity:} è stata prodotta una prima versione del prodotto. Questo risulta funzionante e soddisfa tutti i requisiti obbligatori richiesti. Inoltre sono stati effettuati svariati test affinché l'applicazione rispetti gli standard di qualità prefissati. A seguito di un incontro con il \gloxy{proponente} si è rilevato che il prodotto realizzato soddisfa quanto richiesto ed è pronto per essere consegnato.
\item \gloxy{Stakeholders}: il \gloxy{proponente} si è dimostrato soddisfatto di quanto prodotto. I requisiti richiesti sono stati pienamente soddisfatti. Inoltre il prodotto è stato testato da degli utenti selezionati come tester per avere feedback riguardo l'esperienza d'uso ed eventuali \gloxy{bug}.
\item \textit{Requirements:} tutti i requisiti e le funzionalità obbligatorie sono state implementate e sono funzionanti.
\item \textit{Software system:} il sistema è perfettamente funzionante. L'ambiente è stato configurato affinché l'applicativo sia fruibile sulle piattaforme desiderate.
\item \gloxy{Team:} il team lavora in maniera coesa e disciplinata. Riesce ad autogestirsi per alcuni aspetti e la comunicazione risulta ottima. La perdita di tempo è minima e le scadenze vengono rispettate puntualmente garantendo efficienza ed efficacia. \`{E} ben chiaro l'obbiettivo comune del gruppo e tutti lavoro per il suo raggiungimento.
\item \textit{Work:} il lavoro per il prodotto è stato portato a termine con successo. Il \gloxy{Proponente} è soddisfatto con quanto prodotto e ha espresso ottimi feedback per quanto riguarda i risultati ottenuti.
\item \textit{Way of working:} l'esperienza lavorativa per il \gloxy{progetto} si è conclusa e i membri del tema hanno fatto tesoro dell'esperienza avuta.
\end{itemize}
