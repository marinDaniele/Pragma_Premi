\section{Organigramma}
\subsection{Redazione}
\begin{table}[h]
\begin{center}
\begin{tabularx}{\textwidth}{|X|c|X|}
\hline Nome & Data & Firma \\
\hline \gma & 2014-12-10 & ... \\
\hline
\end{tabularx}
\caption{Redazione}
\end{center}
\end{table}
\subsection{Approvazione}
\begin{table}[h]
\begin{center}
\begin{tabularx}{\textwidth}{|X|c|X|}
\hline Nome & Data & Firma \\
\hline \gma & 2014-12-10 & ... \\
\hline \committente &  & ... \\
\hline
\end{tabularx}
\caption{Approvazione}
\end{center}
\end{table}
\subsection{Accettazione dei componenti}
\begin{table}[h]
\begin{center}
\begin{tabularx}{\textwidth}{|X|c|X|}
\hline Nome & Data & Firma \\
\hline \mb & 2014-12-10 & ... \\
\hline \gma & 2014-12-10 & ... \\
\hline \dm & 2014-12-10 & ... \\
\hline \gmi & 2014-12-10 & ... \\
\hline \sm & 2014-12-10 & ... \\
\hline \ao & 2014-12-10 & ... \\
\hline \fv & 2014-12-10 & ... \\
\hline
\end{tabularx}
\caption{Accettazione}
\end{center}
\end{table}
\pagebreak
\subsection{Componenti}
\begin{table}[h]
\begin{center}
\begin{tabularx}{\textwidth}{|X|c|m{7cm}|}
\hline Nome & Matricola & Indirizzo e-mail \\
\hline \mb & 584257 & massimiliano.baruffato@studenti.unipd.it \\
\hline \gma & 1049820 & giacomo.manzoli@studenti.unipd.it \\
\hline \dm & 523148 & daniele.marin@studenti.unipd.it \\
\hline \gmi & 1026084 & gianmarco.midena@studenti.unipd.it \\
\hline \sm & 1031243 & stefano.munari.1@studenti.unipd.it \\
\hline \ao & 1005923 & andrea.ongaro.1@studenti.unipd.it \\
\hline \fv & 1029029 & fabio.vedovato@studenti.unipd.it \\
\hline
\end{tabularx}
\caption{Componenti}
\end{center}
\end{table}
\subsection{Ruoli}
Durante lo sviluppo del \gloxy{progetto} ogni componente del gruppo \gruppo ricoprirà almeno una volta ognuno dei ruoli\footnote{Vengono considerate anche le ore a carico del \gloxy{fornitore} delle fasi di \fAt e \fADt}. L'organizzazione di questa alternanza sarà compito del \rRP il quale dovrà assicurarsi che non ci siano conflitti d'interesse tra i ruoli ricoperti dai membri.\footnote{Come specificato nel regolamento dell'organigramma \url{http://www.math.unipd.it/~tullio/IS-1/2014/Progetto/PD01b.html}} \\
Il \rRP avrà anche il compito di garantire che il carico di lavoro sia equo per ogni membro del gruppo. \\
Ogni ruolo ha un costo orario diverso, come riportato in tabella \ref{tabellacosti}.
\begin{table}[h]
\begin{center}
\begin{tabular}{|c|c|}
\hline Ruolo & Costo \\
\hline \rRP & 30\euro \\
\rAP & 20\euro \\
\rA & 25\euro \\
\rP & 22\euro \\
\rV & 15\euro \\
\rp & 15\euro \\
\hline
\end{tabular}
\caption{Costi orari per ruolo}\label{tabellacosti}
\end{center}
\end{table}
