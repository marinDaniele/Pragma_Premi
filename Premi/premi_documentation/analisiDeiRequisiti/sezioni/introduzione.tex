\section{Introduzione}
\subsection{Scopo del documento}
Questo documento ha lo scopo di definire i requisiti del prodotto emersi durante l'analisi del capitolato C4 e successivamente all'incontro con il \proponente.
\subsection{Scopo del prodotto}\label{scopoProdotto}
\scopoProdotto
\subsection{Glossario}
\descrizioneGlossario
\subsection{Riferimenti}
\subsubsection{Normativi}
\begin{itemize}
\item \textbf{Norme di \nogloxy{Progetto}}: \normeDiProgetto;
\item \textbf{Capitolato d'appalto C4}: \progetto: Software di presentazione \textit{better than Prezi}. Reperibile all'indirizzo: \url{http://www.math.unipd.it/~tullio/IS-1/2014/Progetto/C4.pdf};
\item \textbf{Verbale Esterno}: \eII.
\end{itemize}
\subsubsection{Informativi}
%http://books.google.it/books?id=h-hCKFMbqNMC&pg=PA137&hl=it&source=gbs_toc_r&cad=4#v=onepage&q&f=false
\begin{itemize}
\item \textbf{\SF}: \studioDiFattibilita;
\item \textbf{Ingegneria del Software - Ian Sommerville - Ottava edizione:}
\begin{itemize}
\item Capitolo 6: Requisiti del software;
\item Capitolo 7: Processi di ingegneria dei requisiti.
\end{itemize}
\item \textbf{Slide dell'insegnamento - Diagrammi dei casi d'uso:} \url{http://www.math.unipd.it/~tullio/IS-1/2014/Dispense/E2a.pdf};
\item \textbf{Verbali Interni}:
\begin{itemize}
\item \iI;
\item \iII;
\end{itemize}
\end{itemize}
