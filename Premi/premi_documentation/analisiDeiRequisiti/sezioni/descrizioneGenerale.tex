\section{Descrizione generale}
\subsection{Contesto d'uso del prodotto}
Il prodotto dovrà essere utilizzabile da desktop su un qualsiasi \gloxy{browser} che sia compatibile con la tecnologia \gloxy{HTML5} e CSS3. In aggiunta la funzionalità di presentazione dovrà essere usabile anche da mobile sotto medesime condizioni.\\
Ci si aspetta quindi che l'usabilità del prodotto non venga preclusa dal \gloxy{sistema operativo} che si sta utilizzando e che esso sia utilizzabile nei più recenti dispositivi mobile dotati in \gloxy{browser} per la navigazione.
\subsection{Funzioni del prodotto}\label{funzioniProdotto}
Il prodotto sarà composto da un applicativo \gloxy{web} per la creazione e l'esecuzione di presentazioni.
La \emph{modalità desktop} includerà entrambe queste funzionalità, mentre quella \emph{mobile} consentirà la sola esecuzione di presentazioni.
In fase di \emph{creazione}, l'utente dovrà costruire una mappa mentale, sulla quale potrà successivamente definire uno o più \gloxy{percorsi} di presentazione. Durante la costruzione della mappa mentale, il sistema creerà automaticamente un \emph{\gloxy{percorso} di default}, seguendo l'ordine di inserimento dei nodi nella mappa.
In fase di \emph{esecuzione}, l'utente dovrà scegliere un \gloxy{percorso di presentazione} definito in fase di creazione. L'utente potrà decidere di avanzare linearmente lungo tale \gloxy{percorso} oppure visualizzare il contenuto di un qualsiasi nodo della mappa mentale.
Sarà possibile \emph{salvare} un \gloxy{progetto}, \emph{esportare in \gloxy{PDF}} e \emph{stampare} la \gloxy{mappa mentale} e le presentazioni relative ad un \gloxy{progetto}, ed \emph{esportare} una \emph{pagina \gloxy{web}} con tutti i \gloxy{percorsi di presentazione} di un \gloxy{progetto}.
Verrà infine reso disponibile un \MU, contenente una descrizione dei comandi e delle funzionalità relative all'applicazione.
\subsection{Caratteristiche degli utenti}
La categoria di utenti che utilizzerà questo prodotto non dovrà possedere competenze particolari.\\
Per questo motivo il prodotto sarà dotato di un'interfaccia utente il più intuitiva possibile che permetterà la creazione sia di \gloxy{progetti} semplici sia di \gloxy{progetti} più articolati, in base alle necessità dell'utilizzatore.\\
Allegato all'applicativo sarà fornito anche un \MU con tutte le indicazioni necessarie per consentire un utilizzo corretto ed efficace del prodotto.
\subsection{Vincoli generali}
I \gloxy{progetti} potranno essere salvati sotto un formato definito dall'applicazione stessa e non saranno quindi compatibili con altri prodotti in commercio.\\
Di conseguenza per modificare o presentare i propri \gloxy{progetti} l'utente è vincolato all'utilizzo dell'applicazione oggetto dell'appalto.\\
In caso in cui l'utente voglia visualizzare il suo \gloxy{progetto} al di fuori dell'applicativo dovrà esportalo in un dei formati previsti.
\subsection{Assunzione dipendenze}
Per il corretto funzionamento dell’applicazione sarà necessario l’utilizzo di un \gloxy{browser} che sia compatibile con lo standard \gloxy{HTML5} e CSS3.
