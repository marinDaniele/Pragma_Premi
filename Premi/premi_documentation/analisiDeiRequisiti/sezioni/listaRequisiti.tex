\section{Requisiti}
\label{tabellona}
Di seguito vengono riportati tutti i requisiti individuati dal \gloxy{team} durante l'analisi del capitolato, dei casi d'uso, dall'incontro con il \gloxy{Proponente} oppure da necessità interne.
Per facilitare la consultazione, i requisiti saranno separati su più tabelle in base alla loro categoria.
I requisiti saranno classificati per tipo e importanza e utilizzeranno la seguente sintassi:
\begin{center}
R[Importanza][Tipo][Codice]
\end{center}
\begin{itemize}
\item \textbf{Importanza}: può assumere sono uno fra i seguenti valori:
\begin{itemize}
\item \textit{O}: requisito obbligatorio;
\item \textit{D}: requisito desiderabile;
\item \textit{F}: requisito facoltativo.
\end{itemize}
\item \textbf{Tipo}: può assumere solo uno fra i seguenti valori:
\begin{itemize}
\item \textit{F}: funzionale;
\item \textit{P}: prestazionale;
\item \textit{Q}: qualità;
\item \textit{V}: vincolo.
\end{itemize}
\item \textbf{Codice}: è il codice gerarchico univoco di ogni requisito, espresso in numeri.
\end{itemize}
\noindent Per ogni requisito sarà anche specificato:
\begin{itemize}
\item \textbf{Descrizione}: una breve ma completa descrizione del requisito, che deve essere il meno ambigua possibile;
\item \textbf{Fonte}: che può essere una o più delle seguenti:
\begin{itemize}
\item \hypertarget{Capitolato}{\textit{Capitolato}}: si tratta di un requisito derivato direttamente dal capitolato;
\item \hypertarget{Verbale 2014-12-19}{\textit{Verbale 2014-12-19}}: si tratta di un requisito derivato direttamente dal verbale \iI;
\item \hypertarget{Interno}{\textit{Interno}}: si tratta di un requisito identificato dagli \rAs;
\item \hypertarget{Verbale 2015-03-13}{\textit{Verbale 2015-03-13}}: si tratta di un requisito derivato direttamente dal verbale \iII;
\item \textit{Codice di un caso d'uso}: si tratta di un requisito emerso da un caso d'uso, in questo caso viene riportato l'identificativo del caso d'uso associato.
\end{itemize}
\end{itemize}
\begingroup
\let\clearpage\relax
\input{sezioni/tabellaRequisiti.tex}
\subsection{Tracciamento Requisiti-Fonti}
\normalsize
\begin{longtable}{|>{\centering}m{5cm}|m{5cm}<{\centering}|}
\hline
\textbf{Id Requisito} & \textbf{Fonti}\\
\hline
\endhead
\hyperlink{RFO1}{RFO1} & \hyperlink{Capitolato}{Capitolato}\\
& \hyperref[UC1.1]{UC1.1}\\ \hline
\hyperlink{RFO1.1}{RFO1.1} & \hyperlink{Interno}{Interno}\\
& \hyperref[UC1.1]{UC1.1}\\
& \hyperref[UC1.1.1]{UC1.1.1}\\ \hline
\hyperlink{RFO1.2}{RFO1.2} & \hyperlink{Interno}{Interno}\\
& \hyperref[UC1.1]{UC1.1}\\
& \hyperref[UC1.1.2]{UC1.1.2}\\ \hline
\hyperlink{RFO1.2.1}{RFO1.2.1} & \hyperlink{Interno}{Interno}\\
& \hyperref[UC1.1.2]{UC1.1.2}\\ \hline
\hyperlink{RFO1.2.1.1}{RFO1.2.1.1} & \hyperlink{Interno}{Interno}\\
& \hyperref[UC1.1.2]{UC1.1.2}\\ \hline
\hyperlink{RFO2}{RFO2} & \hyperlink{Capitolato}{Capitolato}\\
& \hyperref[UC1.1]{UC1.1}\\ \hline
\hyperlink{RFD3}{RFD3} & \hyperlink{Verbale 2014-12-19}{Verbale 2014-12-19}\\
& \hyperref[UC1.1]{UC1.1}\\ \hline
\hyperlink{RFO4}{RFO4} & \hyperlink{Capitolato}{Capitolato}\\
& \hyperref[UC1.2]{UC1.2}\\ \hline
\hyperlink{RFD4.1}{RFD4.1} & \hyperlink{Interno}{Interno}\\
& \hyperref[UC1.2.1]{UC1.2.1}\\ \hline
\hyperlink{RFO4.2}{RFO4.2} & \hyperlink{Interno}{Interno}\\
& \hyperref[UC1.2]{UC1.2}\\
& \hyperref[UC1.2.2]{UC1.2.2}\\ \hline
\hyperlink{RFO4.2.1}{RFO4.2.1} & \hyperlink{Interno}{Interno}\\
& \hyperref[UC1.2.2]{UC1.2.2}\\
& \hyperref[UC1.2.2.1]{UC1.2.2.1}\\ \hline
\hyperlink{RFO4.2.2}{RFO4.2.2} & \hyperlink{Interno}{Interno}\\
& \hyperref[UC1.2.2]{UC1.2.2}\\
& \hyperref[UC1.2.2.2]{UC1.2.2.2}\\ \hline
\hyperlink{RFO4.2.3}{RFO4.2.3} & \hyperlink{Interno}{Interno}\\
& \hyperref[UC1.2.2]{UC1.2.2}\\
& \hyperref[UC1.2.2.3]{UC1.2.2.3}\\ \hline
\hyperlink{RFO4.2.3.1}{RFO4.2.3.1} & \hyperlink{Interno}{Interno}\\
& \hyperref[UC1.2.2.3]{UC1.2.2.3}\\
& \hyperref[UC1.2.2.3.1]{UC1.2.2.3.1}\\ \hline
\hyperlink{RFO4.2.3.2}{RFO4.2.3.2} & \hyperlink{Interno}{Interno}\\
& \hyperref[UC1.2.2.3]{UC1.2.2.3}\\
& \hyperref[UC1.2.2.3.2]{UC1.2.2.3.2}\\ \hline
\hyperlink{RFF4.2.3.3}{RFF4.2.3.3} & \hyperlink{Interno}{Interno}\\
& \hyperref[UC1.2.2.3]{UC1.2.2.3}\\
& \hyperref[UC1.2.2.3.3]{UC1.2.2.3.3}\\ \hline
\hyperlink{RFO4.2.3.4}{RFO4.2.3.4} & \hyperlink{Interno}{Interno}\\
& \hyperref[UC1.2.2.3]{UC1.2.2.3}\\
& \hyperref[UC1.2.2.3.4]{UC1.2.2.3.4}\\ \hline
\hyperlink{RFO4.2.3.5}{RFO4.2.3.5} & \hyperlink{Interno}{Interno}\\
& \hyperref[UC1.2.2.3]{UC1.2.2.3}\\
& \hyperref[UC1.2.2.3.5]{UC1.2.2.3.5}\\ \hline
\hyperlink{RFO4.2.3.5.1}{RFO4.2.3.5.1} & \hyperlink{Interno}{Interno}\\
& \hyperref[UC1.2.2.3.5]{UC1.2.2.3.5}\\
& \hyperref[UC1.2.2.3.5.2]{UC1.2.2.3.5.2}\\ \hline
\hyperlink{RFO4.2.3.5.2}{RFO4.2.3.5.2} & \hyperlink{Interno}{Interno}\\
& \hyperref[UC1.2.2.3.5]{UC1.2.2.3.5}\\
& \hyperref[UC1.2.2.3.5.1]{UC1.2.2.3.5.1}\\ \hline
\hyperlink{RFF4.2.3.6}{RFF4.2.3.6} & \hyperlink{Interno}{Interno}\\
& \hyperref[UC1.2.2.3]{UC1.2.2.3}\\
& \hyperref[UC1.2.2.3.6]{UC1.2.2.3.6}\\ \hline
\hyperlink{RFF4.2.3.6.1}{RFF4.2.3.6.1} & \hyperlink{Interno}{Interno}\\
& \hyperref[UC1.2.2.3.6]{UC1.2.2.3.6}\\
& \hyperref[UC1.2.2.3.6.1]{UC1.2.2.3.6.1}\\ \hline
\hyperlink{RFF4.2.3.6.2}{RFF4.2.3.6.2} & \hyperlink{Interno}{Interno}\\
& \hyperref[UC1.2.2.3.6]{UC1.2.2.3.6}\\
& \hyperref[UC1.2.2.3.6.1]{UC1.2.2.3.6.1}\\ \hline
\hyperlink{RFF4.2.3.6.3}{RFF4.2.3.6.3} & \hyperlink{Interno}{Interno}\\
& \hyperref[UC1.2.2.3.6]{UC1.2.2.3.6}\\
& \hyperref[UC1.2.2.3.6.2]{UC1.2.2.3.6.2}\\ \hline
\hyperlink{RFO4.2.3.7}{RFO4.2.3.7} & \hyperlink{Interno}{Interno}\\
& \hyperref[UC1.2.2.3]{UC1.2.2.3}\\
& \hyperref[UC1.2.2.3.7]{UC1.2.2.3.7}\\ \hline
\hyperlink{RFD4.2.3.8}{RFD4.2.3.8} & \hyperlink{Interno}{Interno}\\
& \hyperref[UC1.2.2.3]{UC1.2.2.3}\\
& \hyperref[UC1.2.2.3.8]{UC1.2.2.3.8}\\ \hline
\hyperlink{RFD4.2.3.8.1}{RFD4.2.3.8.1} & \hyperlink{Interno}{Interno}\\
& \hyperref[UC1.2.2.3.8]{UC1.2.2.3.8}\\
& \hyperref[UC1.2.2.3.8.1]{UC1.2.2.3.8.1}\\ \hline
\hyperlink{RFD4.2.3.8.2}{RFD4.2.3.8.2} & \hyperlink{Interno}{Interno}\\
& \hyperref[UC1.2.2.3.8]{UC1.2.2.3.8}\\
& \hyperref[UC1.2.2.3.8.2]{UC1.2.2.3.8.2}\\ \hline
\hyperlink{RFD4.2.3.9}{RFD4.2.3.9} & \hyperlink{Interno}{Interno}\\
& \hyperref[UC1.2.2.3]{UC1.2.2.3}\\
& \hyperref[UC1.2.2.3.9]{UC1.2.2.3.9}\\ \hline
\hyperlink{RFD4.2.3.10}{RFD4.2.3.10} & \hyperlink{Interno}{Interno}\\
& \hyperref[UC1.2.2.3]{UC1.2.2.3}\\
& \hyperref[UC1.2.2.3.10]{UC1.2.2.3.10}\\ \hline
\hyperlink{RFF4.2.3.11}{RFF4.2.3.11} & \hyperlink{Interno}{Interno}\\
& \hyperref[UC1.2.2.3.11]{UC1.2.2.3.11}\\ \hline
\hyperlink{RFO4.2.3.12}{RFO4.2.3.12} & \hyperlink{Interno}{Interno}\\
& \hyperref[UC1.2.2.3]{UC1.2.2.3}\\
& \hyperref[UC1.2.2.3.12]{UC1.2.2.3.12}\\ \hline
\hyperlink{RFO4.2.3.13}{RFO4.2.3.13} & \hyperlink{Interno}{Interno}\\
& \hyperref[UC1.2.2.3]{UC1.2.2.3}\\
& \hyperref[UC1.2.2.3.13]{UC1.2.2.3.13}\\ \hline
\hyperlink{RFF4.2.3.14}{RFF4.2.3.14} & \hyperlink{Interno}{Interno}\\
& \hyperref[UC1.2.2.3]{UC1.2.2.3}\\
& \hyperref[UC1.2.2.3.14]{UC1.2.2.3.14}\\ \hline
\hyperlink{RFD4.2.3.15}{RFD4.2.3.15} & \hyperlink{Interno}{Interno}\\
& \hyperref[UC1.2.2.3]{UC1.2.2.3}\\ \hline
\hyperlink{RFD4.2.3.16}{RFD4.2.3.16} & \hyperlink{Interno}{Interno}\\
& \hyperref[UC1.2.2.3]{UC1.2.2.3}\\ \hline
\hyperlink{RFD4.2.3.17}{RFD4.2.3.17} & \hyperlink{Interno}{Interno}\\
& \hyperref[UC1.2.2.3]{UC1.2.2.3}\\ \hline
\hyperlink{RFD4.2.3.18}{RFD4.2.3.18} & \hyperlink{Interno}{Interno}\\
& \hyperref[UC1.2.2.3]{UC1.2.2.3}\\ \hline
\hyperlink{RFD4.2.3.19}{RFD4.2.3.19} & \hyperlink{Interno}{Interno}\\
& \hyperref[UC1.2.2.3]{UC1.2.2.3}\\ \hline
\hyperlink{RFD4.2.3.20}{RFD4.2.3.20} & \hyperlink{Interno}{Interno}\\
& \hyperref[UC1.2.2.3]{UC1.2.2.3}\\ \hline
\hyperlink{RFD4.2.4}{RFD4.2.4} & \hyperlink{Interno}{Interno}\\
& \hyperref[UC1.2.2]{UC1.2.2}\\
& \hyperref[UC1.2.2.4]{UC1.2.2.4}\\ \hline
\hyperlink{RFO4.2.5}{RFO4.2.5} & \hyperlink{Interno}{Interno}\\
& \hyperref[UC1.2.2]{UC1.2.2}\\
& \hyperref[UC1.2.2.5]{UC1.2.2.5}\\ \hline
\hyperlink{RFO4.2.6}{RFO4.2.6} & \hyperlink{Interno}{Interno}\\
& \hyperref[UC1.2.2]{UC1.2.2}\\
& \hyperref[UC1.2.2.6]{UC1.2.2.6}\\ \hline
\hyperlink{RFO4.2.7}{RFO4.2.7} & \hyperlink{Interno}{Interno}\\
& \hyperref[UC1.2.2]{UC1.2.2}\\
& \hyperref[UC1.2.2.7]{UC1.2.2.7}\\ \hline
\hyperlink{RFO4.2.8}{RFO4.2.8} & \hyperlink{Interno}{Interno}\\
& \hyperref[UC1.2.2]{UC1.2.2}\\
& \hyperref[UC1.2.2.8]{UC1.2.2.8}\\ \hline
\hyperlink{RFD4.2.9}{RFD4.2.9} & \hyperlink{Interno}{Interno}\\
& \hyperref[UC1.2.2]{UC1.2.2}\\
& \hyperref[UC1.2.2.9]{UC1.2.2.9}\\ \hline
\hyperlink{RFD4.3}{RFD4.3} & \hyperlink{Interno}{Interno}\\
& \hyperref[UC1.2]{UC1.2}\\
& \hyperref[UC1.2.3]{UC1.2.3}\\ \hline
\hyperlink{RFD4.3.1}{RFD4.3.1} & \hyperlink{Interno}{Interno}\\
& \hyperref[UC1.2.3]{UC1.2.3}\\
& \hyperref[UC1.2.3.1]{UC1.2.3.1}\\ \hline
\hyperlink{RFD4.3.2}{RFD4.3.2} & \hyperlink{Interno}{Interno}\\
& \hyperref[UC1.2.3]{UC1.2.3}\\
& \hyperref[UC1.2.3.2]{UC1.2.3.2}\\ \hline
\hyperlink{RFD4.3.3}{RFD4.3.3} & \hyperlink{Interno}{Interno}\\
& \hyperref[UC1.2.3]{UC1.2.3}\\
& \hyperref[UC1.2.3.3]{UC1.2.3.3}\\ \hline
\hyperlink{RFD4.4}{RFD4.4} & \hyperlink{Interno}{Interno}\\
& \hyperref[UC1.2]{UC1.2}\\
& \hyperref[UC1.2.4]{UC1.2.4}\\ \hline
\hyperlink{RFD4.4.1}{RFD4.4.1} & \hyperlink{Interno}{Interno}\\
& \hyperref[UC1.2.4]{UC1.2.4}\\
& \hyperref[UC1.2.4.1]{UC1.2.4.1}\\ \hline
\hyperlink{RFD4.4.2}{RFD4.4.2} & \hyperlink{Interno}{Interno}\\
& \hyperref[UC1.2.4]{UC1.2.4}\\
& \hyperref[UC1.2.4.2]{UC1.2.4.2}\\ \hline
\hyperlink{RFD4.4.3}{RFD4.4.3} & \hyperlink{Interno}{Interno}\\
& \hyperref[UC1.2.4]{UC1.2.4}\\
& \hyperref[UC1.2.4.3]{UC1.2.4.3}\\ \hline
\hyperlink{RFD4.4.4}{RFD4.4.4} & \hyperlink{Interno}{Interno}\\
& \hyperref[UC1.2.4]{UC1.2.4}\\
& \hyperref[UC1.2.4.4]{UC1.2.4.4}\\ \hline
\hyperlink{RFD4.4.5}{RFD4.4.5} & \hyperlink{Interno}{Interno}\\
& \hyperref[UC1.2.4]{UC1.2.4}\\ \hline
\hyperlink{RFD4.5}{RFD4.5} & \hyperlink{Interno}{Interno}\\
& \hyperref[UC1.2.5]{UC1.2.5}\\ \hline
\hyperlink{RFD4.5.1}{RFD4.5.1} & \hyperlink{Interno}{Interno}\\
& \hyperref[UC1.2.5.1]{UC1.2.5.1}\\ \hline
\hyperlink{RFF4.6}{RFF4.6} & \hyperlink{Interno}{Interno}\\
& \hyperref[UC1.2]{UC1.2}\\
& \hyperref[UC1.2.6]{UC1.2.6}\\ \hline
\hyperlink{RFF4.6.1}{RFF4.6.1} & \hyperlink{Interno}{Interno}\\
& \hyperref[UC1.2.6]{UC1.2.6}\\
& \hyperref[UC1.2.6.1]{UC1.2.6.1}\\ \hline
\hyperlink{RFF4.6.1.1}{RFF4.6.1.1} & \hyperlink{Interno}{Interno}\\
& \hyperref[UC1.2.6.1]{UC1.2.6.1}\\
& \hyperref[UC1.2.6.1.1]{UC1.2.6.1.1}\\ \hline
\hyperlink{RFF4.6.1.2}{RFF4.6.1.2} & \hyperlink{Interno}{Interno}\\
& \hyperref[UC1.2.6.1]{UC1.2.6.1}\\
& \hyperref[UC1.2.6.1.2]{UC1.2.6.1.2}\\ \hline
\hyperlink{RFF4.6.2}{RFF4.6.2} & \hyperlink{Interno}{Interno}\\
& \hyperref[UC1.2.6]{UC1.2.6}\\
& \hyperref[UC1.2.6.2]{UC1.2.6.2}\\ \hline
\hyperlink{RFF4.6.3}{RFF4.6.3} & \hyperlink{Interno}{Interno}\\
& \hyperref[UC1.2.6]{UC1.2.6}\\
& \hyperref[UC1.2.6.3]{UC1.2.6.3}\\ \hline
\hyperlink{RFD4.7}{RFD4.7} & \hyperlink{Interno}{Interno}\\
& \hyperref[UC1.2.7]{UC1.2.7}\\ \hline
\hyperlink{RFO5}{RFO5} & \hyperlink{Capitolato}{Capitolato}\\
& \hyperref[UC1.2]{UC1.2}\\ \hline
\hyperlink{RFD6}{RFD6} & \hyperlink{Verbale 2014-12-19}{Verbale 2014-12-19}\\
& \hyperref[UC1.2]{UC1.2}\\ \hline
\hyperlink{RFO7}{RFO7} & \hyperlink{Capitolato}{Capitolato}\\
& \hyperref[UC1.3]{UC1.3}\\ \hline
\hyperlink{RFD7.1}{RFD7.1} & \hyperlink{Interno}{Interno}\\
& \hyperref[UC1.3]{UC1.3}\\
& \hyperref[UC1.3.1]{UC1.3.1}\\ \hline
\hyperlink{RFO7.2}{RFO7.2} & \hyperlink{Interno}{Interno}\\
& \hyperref[UC1.3]{UC1.3}\\
& \hyperref[UC1.3.2]{UC1.3.2}\\ \hline
\hyperlink{RFO7.3}{RFO7.3} & \hyperlink{Interno}{Interno}\\
& \hyperref[UC1.3]{UC1.3}\\
& \hyperref[UC1.3.3]{UC1.3.3}\\ \hline
\hyperlink{RFD7.4}{RFD7.4} & \hyperlink{Interno}{Interno}\\
& \hyperref[UC1.3]{UC1.3}\\
& \hyperref[UC1.3.4]{UC1.3.4}\\ \hline
\hyperlink{RFD7.4.1}{RFD7.4.1} & \hyperlink{Interno}{Interno}\\
& \hyperref[UC1.3.4]{UC1.3.4}\\
& \hyperref[UC1.3.4.1]{UC1.3.4.1}\\ \hline
\hyperlink{RFD7.4.2}{RFD7.4.2} & \hyperlink{Interno}{Interno}\\
& \hyperref[UC1.3.4]{UC1.3.4}\\
& \hyperref[UC1.3.4.2]{UC1.3.4.2}\\ \hline
\hyperlink{RFD7.5}{RFD7.5} & \hyperlink{Interno}{Interno}\\
& \hyperref[UC1.3]{UC1.3}\\
& \hyperref[UC1.3.5]{UC1.3.5}\\ \hline
\hyperlink{RFD7.6}{RFD7.6} & \hyperlink{Interno}{Interno}\\
& \hyperref[UC1.3]{UC1.3}\\
& \hyperref[UC1.3.6]{UC1.3.6}\\ \hline
\hyperlink{RFD7.6.1}{RFD7.6.1} & \hyperlink{Interno}{Interno}\\
& \hyperref[UC1.3.6]{UC1.3.6}\\
& \hyperref[UC1.3.6.1]{UC1.3.6.1}\\ \hline
\hyperlink{RFD7.6.2}{RFD7.6.2} & \hyperlink{Interno}{Interno}\\
& \hyperref[UC1.3.6]{UC1.3.6}\\
& \hyperref[UC1.3.6.2]{UC1.3.6.2}\\ \hline
\hyperlink{RFO7.7}{RFO7.7} & \hyperlink{Interno}{Interno}\\
& \hyperref[UC1.3]{UC1.3}\\ \hline
\hyperlink{RFO8}{RFO8} & \hyperlink{Capitolato}{Capitolato}\\
& \hyperref[UC1.3]{UC1.3}\\ \hline
\hyperlink{RFO9}{RFO9} & \hyperlink{Capitolato}{Capitolato}\\
& \hyperref[UC1.3]{UC1.3}\\ \hline
\hyperlink{RFO10}{RFO10} & \hyperlink{Interno}{Interno}\\
& \hyperref[UC1.4]{UC1.4}\\ \hline
\hyperlink{RFO10.1}{RFO10.1} & \hyperlink{Interno}{Interno}\\
& \hyperref[UC1.4.1]{UC1.4.1}\\ \hline
\hyperlink{RFO10.2}{RFO10.2} & \hyperlink{Interno}{Interno}\\
& \hyperref[UC1.4.2]{UC1.4.2}\\ \hline
\hyperlink{RFO11}{RFO11} & \hyperlink{Interno}{Interno}\\
& \hyperref[UC1.5]{UC1.5}\\ \hline
\hyperlink{RFO11.1}{RFO11.1} & \hyperlink{Interno}{Interno}\\
& \hyperref[UC1.5]{UC1.5}\\
& \hyperref[UC1.5.3]{UC1.5.3}\\ \hline
\hyperlink{RFO11.2}{RFO11.2} & \hyperlink{Interno}{Interno}\\
& \hyperref[UC1.5]{UC1.5}\\
& \hyperref[UC1.5.2]{UC1.5.2}\\ \hline
\hyperlink{RFO11.3}{RFO11.3} & \hyperlink{Interno}{Interno}\\
& \hyperref[UC1.5]{UC1.5}\\
& \hyperref[UC1.5.1]{UC1.5.1}\\ \hline
\hyperlink{RFD12}{RFD12} & \hyperlink{Interno}{Interno}\\
& \hyperref[UC1.6]{UC1.6}\\
& \hyperref[UC1.6.1]{UC1.6.1}\\ \hline
\hyperlink{RFD12.1}{RFD12.1} & \hyperlink{Interno}{Interno}\\
& \hyperref[UC1.6.1]{UC1.6.1}\\
& \hyperref[UC1.6.1.1]{UC1.6.1.1}\\ \hline
\hyperlink{RFD12.2}{RFD12.2} & \hyperlink{Interno}{Interno}\\
& \hyperref[UC1.6.1]{UC1.6.1}\\
& \hyperref[UC1.6.1.2]{UC1.6.1.2}\\ \hline
\hyperlink{RFD12.3}{RFD12.3} & \hyperlink{Interno}{Interno}\\
& \hyperref[UC1.6.1]{UC1.6.1}\\
& \hyperref[UC1.6.1.3]{UC1.6.1.3}\\ \hline
\hyperlink{RFF13}{RFF13} & \hyperlink{Capitolato}{Capitolato}\\
& \hyperref[UC1.6]{UC1.6}\\
& \hyperref[UC1.6.2]{UC1.6.2}\\ \hline
\hyperlink{RFF13.1}{RFF13.1} & \hyperlink{Capitolato}{Capitolato}\\
& \hyperref[UC1.6.2]{UC1.6.2}\\
& \hyperref[UC1.6.2.1]{UC1.6.2.1}\\ \hline
\hyperlink{RFF13.2}{RFF13.2} & \hyperlink{Capitolato}{Capitolato}\\
& \hyperref[UC1.6.2]{UC1.6.2}\\
& \hyperref[UC1.6.2.2]{UC1.6.2.2}\\ \hline
\hyperlink{RFF13.3}{RFF13.3} & \hyperlink{Capitolato}{Capitolato}\\
& \hyperref[UC1.6.2]{UC1.6.2}\\
& \hyperref[UC1.6.2.3]{UC1.6.2.3}\\ \hline
\hyperlink{RFD14}{RFD14} & \hyperlink{Interno}{Interno}\\
& \hyperref[UC1.6]{UC1.6}\\
& \hyperref[UC1.6.3]{UC1.6.3}\\ \hline
\hyperlink{RFD14.1}{RFD14.1} & \hyperlink{Interno}{Interno}\\
& \hyperref[UC1.6.3]{UC1.6.3}\\
& \hyperref[UC1.6.3.4]{UC1.6.3.4}\\ \hline
\hyperlink{RFD14.2}{RFD14.2} & \hyperlink{Interno}{Interno}\\
& \hyperref[UC1.6.3]{UC1.6.3}\\
& \hyperref[UC1.6.3.2]{UC1.6.3.2}\\ \hline
\hyperlink{RFD14.3}{RFD14.3} & \hyperlink{Interno}{Interno}\\
& \hyperref[UC1.6.3]{UC1.6.3}\\
& \hyperref[UC1.6.3.3]{UC1.6.3.3}\\ \hline
\hyperlink{RFD14.4}{RFD14.4} & \hyperlink{Interno}{Interno}\\
& \hyperref[UC1.6.3]{UC1.6.3}\\
& \hyperref[UC1.6.3.1]{UC1.6.3.1}\\ \hline
\hyperlink{RFD15}{RFD15} & \hyperlink{Interno}{Interno}\\
& \hyperref[UC1.7]{UC1.7}\\
& \hyperref[UC1.7.1]{UC1.7.1}\\ \hline
\hyperlink{RFF15.1}{RFF15.1} & \hyperlink{Capitolato}{Capitolato}\\
& \hyperref[UC1.7.1]{UC1.7.1}\\
& \hyperref[UC1.7.1.1]{UC1.7.1.1}\\ \hline
\hyperlink{RFF15.2}{RFF15.2} & \hyperlink{Capitolato}{Capitolato}\\
& \hyperref[UC1.7.1]{UC1.7.1}\\
& \hyperref[UC1.7.1.2]{UC1.7.1.2}\\ \hline
\hyperlink{RFD15.3}{RFD15.3} & \hyperlink{Interno}{Interno}\\
& \hyperref[UC1.7.1]{UC1.7.1}\\
& \hyperref[UC1.7.1.3]{UC1.7.1.3}\\ \hline
\hyperlink{RFO16}{RFO16} & \hyperlink{Capitolato}{Capitolato}\\
& \hyperref[UC1.7]{UC1.7}\\
& \hyperref[UC1.7.2]{UC1.7.2}\\ \hline
\hyperlink{RFO16.1}{RFO16.1} & \hyperlink{Interno}{Interno}\\
& \hyperref[UC1.7.2]{UC1.7.2}\\
& \hyperref[UC1.7.2.4]{UC1.7.2.4}\\ \hline
\hyperlink{RFF16.2}{RFF16.2} & \hyperlink{Capitolato}{Capitolato}\\
& \hyperref[UC1.7.2]{UC1.7.2}\\
& \hyperref[UC1.7.2.2]{UC1.7.2.2}\\ \hline
\hyperlink{RFF16.3}{RFF16.3} & \hyperlink{Capitolato}{Capitolato}\\
& \hyperref[UC1.7.2]{UC1.7.2}\\
& \hyperref[UC1.7.2.3]{UC1.7.2.3}\\ \hline
\hyperlink{RFF16.3.1}{RFF16.3.1} & \hyperlink{Interno}{Interno}\\
& \hyperref[UC1.7.2.3]{UC1.7.2.3}\\ \hline
\hyperlink{RFD16.4}{RFD16.4} & \hyperlink{Interno}{Interno}\\
& \hyperref[UC1.7.2]{UC1.7.2}\\
& \hyperref[UC1.7.2.1]{UC1.7.2.1}\\ \hline
\hyperlink{RFD17}{RFD17} & \hyperlink{Capitolato}{Capitolato}\\ \hline
\hyperlink{RFO18}{RFO18} & \hyperlink{Interno}{Interno}\\
& \hyperref[UC1.1.2]{UC1.1.2}\\ \hline
\hyperlink{RFD19}{RFD19} & \hyperlink{Verbale 2014-12-19}{Verbale 2014-12-19}\\
& \hyperref[UC1.2.4]{UC1.2.4}\\
& \hyperref[UC1.2.4.5]{UC1.2.4.5}\\ \hline
\hyperlink{RFD19.1}{RFD19.1} & \hyperlink{Verbale 2014-12-19}{Verbale 2014-12-19}\\
& \hyperref[UC1.2.4.5]{UC1.2.4.5}\\ \hline
\hyperlink{RFD20}{RFD20} & \hyperlink{Capitolato}{Capitolato}\\ \hline
\hyperlink{RFF21}{RFF21} & \hyperlink{Verbale 2014-12-19}{Verbale 2014-12-19}\\ \hline
\hyperlink{RFD22}{RFD22} & \hyperlink{Capitolato}{Capitolato}\\
& \hyperref[UC1.1.2]{UC1.1.2}\\ \hline
\hyperlink{RFD23}{RFD23} & \hyperlink{Capitolato}{Capitolato}\\ \hline
\hyperlink{RFD24}{RFD24} & \hyperlink{Capitolato}{Capitolato}\\ \hline
\hyperlink{RFF25}{RFF25} & \hyperlink{Interno}{Interno}\\
& \hyperref[UC1.8]{UC1.8}\\
& \hyperref[UC1.8.1]{UC1.8.1}\\ \hline
\hyperlink{RFF25.1}{RFF25.1} & \hyperlink{Interno}{Interno}\\
& \hyperref[UC1.8]{UC1.8}\\
& \hyperref[UC1.8.1]{UC1.8.1}\\ \hline
\hyperlink{RFD26}{RFD26} & \hyperlink{Interno}{Interno}\\
& \hyperref[UC1.9]{UC1.9}\\ \hline
\hyperlink{RFO27}{RFO27} & \hyperlink{Capitolato}{Capitolato}\\
& \hyperref[UC1]{UC1}\\ \hline
\hyperlink{RFF28}{RFF28} & \hyperlink{Interno}{Interno}\\
& \hyperref[UC1.2]{UC1.2}\\ \hline
\hyperlink{RFF29}{RFF29} & \hyperlink{Interno}{Interno}\\
& \hyperref[UC1.2]{UC1.2}\\ \hline
\hyperlink{RFO30}{RFO30} & \hyperlink{Verbale 2015-03-13}{Verbale 2015-03-13}\\
& \hyperref[UC2]{UC2}\\ \hline
\hyperlink{RFO30.1}{RFO30.1} & \hyperlink{Verbale 2015-03-13}{Verbale 2015-03-13}\\
& \hyperref[UC2]{UC2}\\
& \hyperref[UC2.1]{UC2.1}\\ \hline
\hyperlink{RFO30.1.1}{RFO30.1.1} & \hyperlink{Verbale 2015-03-13}{Verbale 2015-03-13}\\
& \hyperref[UC2.1]{UC2.1}\\
& \hyperref[UC2.1.1]{UC2.1.1}\\ \hline
\hyperlink{RFO30.1.2}{RFO30.1.2} & \hyperlink{Verbale 2015-03-13}{Verbale 2015-03-13}\\
& \hyperref[UC2.1]{UC2.1}\\
& \hyperref[UC2.1.2]{UC2.1.2}\\ \hline
\hyperlink{RFO30.1.3}{RFO30.1.3} & \hyperlink{Verbale 2015-03-13}{Verbale 2015-03-13}\\
& \hyperref[UC2.1]{UC2.1}\\
& \hyperref[UC2.1.3]{UC2.1.3}\\ \hline
\hyperlink{RFO30.2}{RFO30.2} & \hyperlink{Verbale 2015-03-13}{Verbale 2015-03-13}\\
& \hyperref[UC2]{UC2}\\
& \hyperref[UC2.2]{UC2.2}\\ \hline
\hyperlink{RFO30.2.1}{RFO30.2.1} & \hyperlink{Verbale 2015-03-13}{Verbale 2015-03-13}\\
& \hyperref[UC2.2]{UC2.2}\\
& \hyperref[UC2.2.1]{UC2.2.1}\\ \hline
\hyperlink{RFO30.2.2}{RFO30.2.2} & \hyperlink{Verbale 2015-03-13}{Verbale 2015-03-13}\\
& \hyperref[UC2.2]{UC2.2}\\
& \hyperref[UC2.2.2]{UC2.2.2}\\ \hline
\hyperlink{RFO30.2.3}{RFO30.2.3} & \hyperlink{Verbale 2015-03-13}{Verbale 2015-03-13}\\
& \hyperref[UC2.2]{UC2.2}\\
& \hyperref[UC2.2.3]{UC2.2.3}\\ \hline
\hyperlink{RFO30.3}{RFO30.3} & \hyperlink{Verbale 2015-03-13}{Verbale 2015-03-13}\\
& \hyperref[UC2]{UC2}\\
& \hyperref[UC2.3]{UC2.3}\\ \hline
\hyperlink{RFO30.4}{RFO30.4} & \hyperlink{Verbale 2015-03-13}{Verbale 2015-03-13}\\
& \hyperref[UC2]{UC2}\\
& \hyperref[UC2.4]{UC2.4}\\ \hline
\hyperlink{RFD31}{RFD31} & \hyperlink{Verbale 2015-03-13}{Verbale 2015-03-13}\\
& \hyperref[UC1]{UC1}\\
& \hyperref[UC1.10]{UC1.10}\\ \hline
\hyperlink{RFD31.1}{RFD31.1} & \hyperlink{Verbale 2015-03-13}{Verbale 2015-03-13}\\
& \hyperref[UC1.10]{UC1.10}\\
& \hyperref[UC1.10.1.1]{UC1.10.1.1}\\ \hline
\hyperlink{RFD31.1.1}{RFD31.1.1} & \hyperlink{Verbale 2015-03-13}{Verbale 2015-03-13}\\
& \hyperref[UC1.10.1]{UC1.10.1}\\
& \hyperref[UC1.10.1.1]{UC1.10.1.1}\\ \hline
\hyperlink{RFD31.1.2}{RFD31.1.2} & \hyperlink{Verbale 2015-03-13}{Verbale 2015-03-13}\\
& \hyperref[UC1.10.1]{UC1.10.1}\\
& \hyperref[UC1.10.1.2]{UC1.10.1.2}\\ \hline
\hyperlink{RFD31.1.3}{RFD31.1.3} & \hyperlink{Verbale 2015-03-13}{Verbale 2015-03-13}\\
& \hyperref[UC1.10.1]{UC1.10.1}\\
& \hyperref[UC1.10.1.3]{UC1.10.1.3}\\ \hline
\hyperlink{RFF31.2}{RFF31.2} & \hyperlink{Verbale 2015-03-13}{Verbale 2015-03-13}\\
& \hyperref[UC1.10]{UC1.10}\\
& \hyperref[UC1.10.2]{UC1.10.2}\\ \hline
\hyperlink{RFF31.2.1}{RFF31.2.1} & \hyperlink{Verbale 2015-03-13}{Verbale 2015-03-13}\\
& \hyperref[UC1.10.2]{UC1.10.2}\\
& \hyperref[UC1.10.2.1]{UC1.10.2.1}\\ \hline
\hyperlink{RFO32}{RFO32} & \hyperlink{Verbale 2015-03-13}{Verbale 2015-03-13}\\
& \hyperref[UC1]{UC1}\\
& \hyperref[UC1.11]{UC1.11}\\ \hline
\hyperlink{RFD33}{RFD33} & \hyperlink{Interno}{Interno}\\
& \hyperref[UC1.1]{UC1.1}\\
& \hyperref[UC1.1.3]{UC1.1.3}\\
& \hyperref[UC1.2]{UC1.2}\\ \hline
\hyperlink{RFD34}{RFD34} & \hyperlink{Interno}{Interno}\\
& \hyperref[UC3]{UC3}\\ \hline
\hyperlink{RFD34.1}{RFD34.1} & \hyperlink{Interno}{Interno}\\
& \hyperref[UC3]{UC3}\\
& \hyperref[UC3.5]{UC3.5}\\ \hline
\hyperlink{RFD34.1.1}{RFD34.1.1} & \hyperlink{Interno}{Interno}\\
& \hyperref[UC3.5]{UC3.5}\\
& \hyperref[UC3.5.1]{UC3.5.1}\\ \hline
\hyperlink{RFD34.1.2}{RFD34.1.2} & \hyperlink{Interno}{Interno}\\
& \hyperref[UC3.5]{UC3.5}\\
& \hyperref[UC3.5.2]{UC3.5.2}\\ \hline
\hyperlink{RFD34.2}{RFD34.2} & \hyperlink{Interno}{Interno}\\
& \hyperref[UC3]{UC3}\\
& \hyperref[UC3.2]{UC3.2}\\ \hline
\hyperlink{RFD34.3}{RFD34.3} & \hyperlink{Interno}{Interno}\\
& \hyperref[UC3]{UC3}\\
& \hyperref[UC3.3]{UC3.3}\\ \hline
\hyperlink{RFD34.4}{RFD34.4} & \hyperlink{Interno}{Interno}\\
& \hyperref[UC3]{UC3}\\
& \hyperref[UC3.4]{UC3.4}\\ \hline
\hyperlink{RFD34.4.1}{RFD34.4.1} & \hyperlink{Interno}{Interno}\\
& \hyperref[UC3.4]{UC3.4}\\
& \hyperref[UC3.4.1]{UC3.4.1}\\ \hline
\hyperlink{RFD34.4.2}{RFD34.4.2} & \hyperlink{Interno}{Interno}\\
& \hyperref[UC3.4]{UC3.4}\\
& \hyperref[UC3.4.2]{UC3.4.2}\\ \hline
\hyperlink{RFD34.5}{RFD34.5} & \hyperlink{Interno}{Interno}\\
& \hyperref[UC3]{UC3}\\
& \hyperref[UC3.1]{UC3.1}\\ \hline
\hyperlink{RFD35}{RFD35} & \hyperlink{Verbale 2015-03-13}{Verbale 2015-03-13}\\ \hline
\hyperlink{RFO36}{RFO36} & \hyperlink{Capitolato}{Capitolato}\\ \hline
\hyperlink{RQO1}{RQO1} & \hyperlink{Capitolato}{Capitolato}\\ \hline
\hyperlink{RQO1.1}{RQO1.1} & \hyperlink{Interno}{Interno}\\ \hline
\hyperlink{RQO1.2}{RQO1.2} & \hyperlink{Interno}{Interno}\\ \hline
\hyperlink{RQO1.3}{RQO1.3} & \hyperlink{Interno}{Interno}\\ \hline
\hyperlink{RQO1.4}{RQO1.4} & \hyperlink{Interno}{Interno}\\ \hline
\hyperlink{RQD2}{RQD2} & \hyperlink{Capitolato}{Capitolato}\\ \hline
\hyperlink{RQD2.1}{RQD2.1} & \hyperlink{Interno}{Interno}\\ \hline
\hyperlink{RQF3}{RQF3} & \hyperlink{Interno}{Interno}\\ \hline
\hyperlink{RQO4}{RQO4} & \hyperlink{Interno}{Interno}\\ \hline
\hyperlink{RQO5}{RQO5} & \hyperlink{Interno}{Interno}\\ \hline
\hyperlink{RQF6}{RQF6} & \hyperlink{Interno}{Interno}\\ \hline
\hyperlink{RQO7}{RQO7} & \hyperlink{Interno}{Interno}\\ \hline
\hyperlink{RVO1}{RVO1} & \hyperlink{Capitolato}{Capitolato}\\ \hline
\hyperlink{RVO2}{RVO2} & \hyperlink{Interno}{Interno}\\ \hline
\hyperlink{RVD3}{RVD3} & \hyperlink{Interno}{Interno}\\ \hline
\hyperlink{RVO4}{RVO4} & \hyperlink{Interno}{Interno}\\ \hline
\hyperlink{RVO5}{RVO5} & \hyperlink{Interno}{Interno}\\ \hline
\hyperlink{RVD6}{RVD6} & \hyperlink{Interno}{Interno}\\ \hline
\hyperlink{RVD7}{RVD7} & \hyperlink{Interno}{Interno}\\ \hline
\hyperlink{RVD8}{RVD8} & \hyperlink{Interno}{Interno}\\ \hline
\hyperlink{RVD9}{RVD9} & \hyperlink{Interno}{Interno}\\ \hline
\hyperlink{RVO10}{RVO10} & \hyperlink{Interno}{Interno}\\ \hline
\hyperlink{RVF11}{RVF11} & \hyperlink{Interno}{Interno}\\ \hline
\hyperlink{RVO12}{RVO12} & \hyperlink{Interno}{Interno}\\ \hline
\hyperlink{RVO13}{RVO13} & \hyperlink{Verbale 2015-03-13}{Verbale 2015-03-13}\\ \hline
\caption[Tracciamento Requisiti-Fonti]{Tracciamento Requisiti-Fonti}
\label{tabella:requi-fonti}
\end{longtable}
\clearpage

\subsection{Tracciamento Fonti-Requisiti}
\normalsize
\begin{longtable}{|>{\centering}m{5cm}|m{5cm}<{\centering}|}
\hline
\textbf{Fonte} & \textbf{Id Requisiti}\\
\hline
\endhead
\hyperlink{Capitolato}{Capitolato} & \hyperlink{RFO1}{RFO1}\\
& \hyperlink{RFO2}{RFO2}\\
& \hyperlink{RFO4}{RFO4}\\
& \hyperlink{RFO5}{RFO5}\\
& \hyperlink{RFO7}{RFO7}\\
& \hyperlink{RFO8}{RFO8}\\
& \hyperlink{RFO9}{RFO9}\\
& \hyperlink{RFF13}{RFF13}\\
& \hyperlink{RFF13.1}{RFF13.1}\\
& \hyperlink{RFF13.2}{RFF13.2}\\
& \hyperlink{RFF13.3}{RFF13.3}\\
& \hyperlink{RFF15.1}{RFF15.1}\\
& \hyperlink{RFF15.2}{RFF15.2}\\
& \hyperlink{RFO16}{RFO16}\\
& \hyperlink{RFF16.2}{RFF16.2}\\
& \hyperlink{RFF16.3}{RFF16.3}\\
& \hyperlink{RFD17}{RFD17}\\
& \hyperlink{RFD20}{RFD20}\\
& \hyperlink{RFD22}{RFD22}\\
& \hyperlink{RFD23}{RFD23}\\
& \hyperlink{RFD24}{RFD24}\\
& \hyperlink{RFO27}{RFO27}\\
& \hyperlink{RFO36}{RFO36}\\
& \hyperlink{RQO1}{RQO1}\\
& \hyperlink{RQD2}{RQD2}\\
& \hyperlink{RVO1}{RVO1}\\ \hline
\hyperlink{Interno}{Interno} & \hyperlink{RFO1.1}{RFO1.1}\\
& \hyperlink{RFO1.2}{RFO1.2}\\
& \hyperlink{RFO1.2.1}{RFO1.2.1}\\
& \hyperlink{RFO1.2.1.1}{RFO1.2.1.1}\\
& \hyperlink{RFD4.1}{RFD4.1}\\
& \hyperlink{RFO4.2}{RFO4.2}\\
& \hyperlink{RFO4.2.1}{RFO4.2.1}\\
& \hyperlink{RFO4.2.2}{RFO4.2.2}\\
& \hyperlink{RFO4.2.3}{RFO4.2.3}\\
& \hyperlink{RFO4.2.3.1}{RFO4.2.3.1}\\
& \hyperlink{RFO4.2.3.2}{RFO4.2.3.2}\\
& \hyperlink{RFF4.2.3.3}{RFF4.2.3.3}\\
& \hyperlink{RFO4.2.3.4}{RFO4.2.3.4}\\
& \hyperlink{RFO4.2.3.5}{RFO4.2.3.5}\\
& \hyperlink{RFO4.2.3.5.1}{RFO4.2.3.5.1}\\
& \hyperlink{RFO4.2.3.5.2}{RFO4.2.3.5.2}\\
& \hyperlink{RFF4.2.3.6}{RFF4.2.3.6}\\
& \hyperlink{RFF4.2.3.6.1}{RFF4.2.3.6.1}\\
& \hyperlink{RFF4.2.3.6.2}{RFF4.2.3.6.2}\\
& \hyperlink{RFF4.2.3.6.3}{RFF4.2.3.6.3}\\
& \hyperlink{RFO4.2.3.7}{RFO4.2.3.7}\\
& \hyperlink{RFD4.2.3.8}{RFD4.2.3.8}\\
& \hyperlink{RFD4.2.3.8.1}{RFD4.2.3.8.1}\\
& \hyperlink{RFD4.2.3.8.2}{RFD4.2.3.8.2}\\
& \hyperlink{RFD4.2.3.9}{RFD4.2.3.9}\\
& \hyperlink{RFD4.2.3.10}{RFD4.2.3.10}\\
& \hyperlink{RFF4.2.3.11}{RFF4.2.3.11}\\
& \hyperlink{RFO4.2.3.12}{RFO4.2.3.12}\\
& \hyperlink{RFO4.2.3.13}{RFO4.2.3.13}\\
& \hyperlink{RFF4.2.3.14}{RFF4.2.3.14}\\
& \hyperlink{RFD4.2.3.15}{RFD4.2.3.15}\\
& \hyperlink{RFD4.2.3.16}{RFD4.2.3.16}\\
& \hyperlink{RFD4.2.3.17}{RFD4.2.3.17}\\
& \hyperlink{RFD4.2.3.18}{RFD4.2.3.18}\\
& \hyperlink{RFD4.2.3.19}{RFD4.2.3.19}\\
& \hyperlink{RFD4.2.3.20}{RFD4.2.3.20}\\
& \hyperlink{RFD4.2.4}{RFD4.2.4}\\
& \hyperlink{RFO4.2.5}{RFO4.2.5}\\
& \hyperlink{RFO4.2.6}{RFO4.2.6}\\
& \hyperlink{RFO4.2.7}{RFO4.2.7}\\
& \hyperlink{RFO4.2.8}{RFO4.2.8}\\
& \hyperlink{RFD4.2.9}{RFD4.2.9}\\
& \hyperlink{RFD4.3}{RFD4.3}\\
& \hyperlink{RFD4.3.1}{RFD4.3.1}\\
& \hyperlink{RFD4.3.2}{RFD4.3.2}\\
& \hyperlink{RFD4.3.3}{RFD4.3.3}\\
& \hyperlink{RFD4.4}{RFD4.4}\\
& \hyperlink{RFD4.4.1}{RFD4.4.1}\\
& \hyperlink{RFD4.4.2}{RFD4.4.2}\\
& \hyperlink{RFD4.4.3}{RFD4.4.3}\\
& \hyperlink{RFD4.4.4}{RFD4.4.4}\\
& \hyperlink{RFD4.4.5}{RFD4.4.5}\\
& \hyperlink{RFD4.5}{RFD4.5}\\
& \hyperlink{RFD4.5.1}{RFD4.5.1}\\
& \hyperlink{RFF4.6}{RFF4.6}\\
& \hyperlink{RFF4.6.1}{RFF4.6.1}\\
& \hyperlink{RFF4.6.1.1}{RFF4.6.1.1}\\
& \hyperlink{RFF4.6.1.2}{RFF4.6.1.2}\\
& \hyperlink{RFF4.6.2}{RFF4.6.2}\\
& \hyperlink{RFF4.6.3}{RFF4.6.3}\\
& \hyperlink{RFD4.7}{RFD4.7}\\
& \hyperlink{RFD7.1}{RFD7.1}\\
& \hyperlink{RFO7.2}{RFO7.2}\\
& \hyperlink{RFO7.3}{RFO7.3}\\
& \hyperlink{RFD7.4}{RFD7.4}\\
& \hyperlink{RFD7.4.1}{RFD7.4.1}\\
& \hyperlink{RFD7.4.2}{RFD7.4.2}\\
& \hyperlink{RFD7.5}{RFD7.5}\\
& \hyperlink{RFD7.6}{RFD7.6}\\
& \hyperlink{RFD7.6.1}{RFD7.6.1}\\
& \hyperlink{RFD7.6.2}{RFD7.6.2}\\
& \hyperlink{RFO7.7}{RFO7.7}\\
& \hyperlink{RFO10}{RFO10}\\
& \hyperlink{RFO10.1}{RFO10.1}\\
& \hyperlink{RFO10.2}{RFO10.2}\\
& \hyperlink{RFO11}{RFO11}\\
& \hyperlink{RFO11.1}{RFO11.1}\\
& \hyperlink{RFO11.2}{RFO11.2}\\
& \hyperlink{RFO11.3}{RFO11.3}\\
& \hyperlink{RFD12}{RFD12}\\
& \hyperlink{RFD12.1}{RFD12.1}\\
& \hyperlink{RFD12.2}{RFD12.2}\\
& \hyperlink{RFD12.3}{RFD12.3}\\
& \hyperlink{RFD14}{RFD14}\\
& \hyperlink{RFD14.1}{RFD14.1}\\
& \hyperlink{RFD14.2}{RFD14.2}\\
& \hyperlink{RFD14.3}{RFD14.3}\\
& \hyperlink{RFD14.4}{RFD14.4}\\
& \hyperlink{RFD15}{RFD15}\\
& \hyperlink{RFD15.3}{RFD15.3}\\
& \hyperlink{RFO16.1}{RFO16.1}\\
& \hyperlink{RFF16.3.1}{RFF16.3.1}\\
& \hyperlink{RFD16.4}{RFD16.4}\\
& \hyperlink{RFO18}{RFO18}\\
& \hyperlink{RFF25}{RFF25}\\
& \hyperlink{RFF25.1}{RFF25.1}\\
& \hyperlink{RFD26}{RFD26}\\
& \hyperlink{RFF28}{RFF28}\\
& \hyperlink{RFF29}{RFF29}\\
& \hyperlink{RFD33}{RFD33}\\
& \hyperlink{RFD34}{RFD34}\\
& \hyperlink{RFD34.1}{RFD34.1}\\
& \hyperlink{RFD34.1.1}{RFD34.1.1}\\
& \hyperlink{RFD34.1.2}{RFD34.1.2}\\
& \hyperlink{RFD34.2}{RFD34.2}\\
& \hyperlink{RFD34.3}{RFD34.3}\\
& \hyperlink{RFD34.4}{RFD34.4}\\
& \hyperlink{RFD34.4.1}{RFD34.4.1}\\
& \hyperlink{RFD34.4.2}{RFD34.4.2}\\
& \hyperlink{RFD34.5}{RFD34.5}\\
& \hyperlink{RQO1.1}{RQO1.1}\\
& \hyperlink{RQO1.2}{RQO1.2}\\
& \hyperlink{RQO1.3}{RQO1.3}\\
& \hyperlink{RQO1.4}{RQO1.4}\\
& \hyperlink{RQD2.1}{RQD2.1}\\
& \hyperlink{RQF3}{RQF3}\\
& \hyperlink{RQO4}{RQO4}\\
& \hyperlink{RQO5}{RQO5}\\
& \hyperlink{RQF6}{RQF6}\\
& \hyperlink{RQO7}{RQO7}\\
& \hyperlink{RVO2}{RVO2}\\
& \hyperlink{RVD3}{RVD3}\\
& \hyperlink{RVO4}{RVO4}\\
& \hyperlink{RVO5}{RVO5}\\
& \hyperlink{RVD6}{RVD6}\\
& \hyperlink{RVD7}{RVD7}\\
& \hyperlink{RVD8}{RVD8}\\
& \hyperlink{RVD9}{RVD9}\\
& \hyperlink{RVO10}{RVO10}\\
& \hyperlink{RVF11}{RVF11}\\
& \hyperlink{RVO12}{RVO12}\\ \hline
\hyperlink{Verbale 2014-12-19}{Verbale 2014-12-19} & \hyperlink{RFD3}{RFD3}\\
& \hyperlink{RFD6}{RFD6}\\
& \hyperlink{RFD19}{RFD19}\\
& \hyperlink{RFD19.1}{RFD19.1}\\
& \hyperlink{RFF21}{RFF21}\\ \hline
\hyperlink{Verbale 2015-03-13}{Verbale 2015-03-13} & \hyperlink{RFO30}{RFO30}\\
& \hyperlink{RFO30.1}{RFO30.1}\\
& \hyperlink{RFO30.1.1}{RFO30.1.1}\\
& \hyperlink{RFO30.1.2}{RFO30.1.2}\\
& \hyperlink{RFO30.1.3}{RFO30.1.3}\\
& \hyperlink{RFO30.2}{RFO30.2}\\
& \hyperlink{RFO30.2.1}{RFO30.2.1}\\
& \hyperlink{RFO30.2.2}{RFO30.2.2}\\
& \hyperlink{RFO30.2.3}{RFO30.2.3}\\
& \hyperlink{RFO30.3}{RFO30.3}\\
& \hyperlink{RFO30.4}{RFO30.4}\\
& \hyperlink{RFD31}{RFD31}\\
& \hyperlink{RFD31.1}{RFD31.1}\\
& \hyperlink{RFD31.1.1}{RFD31.1.1}\\
& \hyperlink{RFD31.1.2}{RFD31.1.2}\\
& \hyperlink{RFD31.1.3}{RFD31.1.3}\\
& \hyperlink{RFF31.2}{RFF31.2}\\
& \hyperlink{RFF31.2.1}{RFF31.2.1}\\
& \hyperlink{RFO32}{RFO32}\\
& \hyperlink{RFD35}{RFD35}\\
& \hyperlink{RVO13}{RVO13}\\ \hline
\hyperref[UC1]{UC1} & \hyperlink{RFO27}{RFO27}\\
& \hyperlink{RFD31}{RFD31}\\
& \hyperlink{RFO32}{RFO32}\\ \hline
\hyperref[UC1.1]{UC1.1} & \hyperlink{RFO1}{RFO1}\\
& \hyperlink{RFO1.1}{RFO1.1}\\
& \hyperlink{RFO1.2}{RFO1.2}\\
& \hyperlink{RFO2}{RFO2}\\
& \hyperlink{RFD3}{RFD3}\\
& \hyperlink{RFD33}{RFD33}\\ \hline
\hyperref[UC1.1.1]{UC1.1.1} & \hyperlink{RFO1.1}{RFO1.1}\\ \hline
\hyperref[UC1.1.2]{UC1.1.2} & \hyperlink{RFO1.2}{RFO1.2}\\
& \hyperlink{RFO1.2.1}{RFO1.2.1}\\
& \hyperlink{RFO1.2.1.1}{RFO1.2.1.1}\\
& \hyperlink{RFO18}{RFO18}\\
& \hyperlink{RFD22}{RFD22}\\ \hline
\hyperref[UC1.1.3]{UC1.1.3} & \hyperlink{RFD33}{RFD33}\\ \hline
\hyperref[UC1.2]{UC1.2} & \hyperlink{RFO4}{RFO4}\\
& \hyperlink{RFO4.2}{RFO4.2}\\
& \hyperlink{RFD4.3}{RFD4.3}\\
& \hyperlink{RFD4.4}{RFD4.4}\\
& \hyperlink{RFF4.6}{RFF4.6}\\
& \hyperlink{RFO5}{RFO5}\\
& \hyperlink{RFD6}{RFD6}\\
& \hyperlink{RFF28}{RFF28}\\
& \hyperlink{RFF29}{RFF29}\\
& \hyperlink{RFD33}{RFD33}\\ \hline
\hyperref[UC1.2.1]{UC1.2.1} & \hyperlink{RFD4.1}{RFD4.1}\\ \hline
\hyperref[UC1.2.2]{UC1.2.2} & \hyperlink{RFO4.2}{RFO4.2}\\
& \hyperlink{RFO4.2.1}{RFO4.2.1}\\
& \hyperlink{RFO4.2.2}{RFO4.2.2}\\
& \hyperlink{RFO4.2.3}{RFO4.2.3}\\
& \hyperlink{RFD4.2.4}{RFD4.2.4}\\
& \hyperlink{RFO4.2.5}{RFO4.2.5}\\
& \hyperlink{RFO4.2.6}{RFO4.2.6}\\
& \hyperlink{RFO4.2.7}{RFO4.2.7}\\
& \hyperlink{RFO4.2.8}{RFO4.2.8}\\
& \hyperlink{RFD4.2.9}{RFD4.2.9}\\ \hline
\hyperref[UC1.2.2.1]{UC1.2.2.1} & \hyperlink{RFO4.2.1}{RFO4.2.1}\\ \hline
\hyperref[UC1.2.2.2]{UC1.2.2.2} & \hyperlink{RFO4.2.2}{RFO4.2.2}\\ \hline
\hyperref[UC1.2.2.3]{UC1.2.2.3} & \hyperlink{RFO4.2.3}{RFO4.2.3}\\
& \hyperlink{RFO4.2.3.1}{RFO4.2.3.1}\\
& \hyperlink{RFO4.2.3.2}{RFO4.2.3.2}\\
& \hyperlink{RFF4.2.3.3}{RFF4.2.3.3}\\
& \hyperlink{RFO4.2.3.4}{RFO4.2.3.4}\\
& \hyperlink{RFO4.2.3.5}{RFO4.2.3.5}\\
& \hyperlink{RFF4.2.3.6}{RFF4.2.3.6}\\
& \hyperlink{RFO4.2.3.7}{RFO4.2.3.7}\\
& \hyperlink{RFD4.2.3.8}{RFD4.2.3.8}\\
& \hyperlink{RFD4.2.3.9}{RFD4.2.3.9}\\
& \hyperlink{RFD4.2.3.10}{RFD4.2.3.10}\\
& \hyperlink{RFO4.2.3.12}{RFO4.2.3.12}\\
& \hyperlink{RFO4.2.3.13}{RFO4.2.3.13}\\
& \hyperlink{RFF4.2.3.14}{RFF4.2.3.14}\\
& \hyperlink{RFD4.2.3.15}{RFD4.2.3.15}\\
& \hyperlink{RFD4.2.3.16}{RFD4.2.3.16}\\
& \hyperlink{RFD4.2.3.17}{RFD4.2.3.17}\\
& \hyperlink{RFD4.2.3.18}{RFD4.2.3.18}\\
& \hyperlink{RFD4.2.3.19}{RFD4.2.3.19}\\
& \hyperlink{RFD4.2.3.20}{RFD4.2.3.20}\\ \hline
\hyperref[UC1.2.2.3.1]{UC1.2.2.3.1} & \hyperlink{RFO4.2.3.1}{RFO4.2.3.1}\\ \hline
\hyperref[UC1.2.2.3.2]{UC1.2.2.3.2} & \hyperlink{RFO4.2.3.2}{RFO4.2.3.2}\\ \hline
\hyperref[UC1.2.2.3.3]{UC1.2.2.3.3} & \hyperlink{RFF4.2.3.3}{RFF4.2.3.3}\\ \hline
\hyperref[UC1.2.2.3.4]{UC1.2.2.3.4} & \hyperlink{RFO4.2.3.4}{RFO4.2.3.4}\\ \hline
\hyperref[UC1.2.2.3.5]{UC1.2.2.3.5} & \hyperlink{RFO4.2.3.5}{RFO4.2.3.5}\\
& \hyperlink{RFO4.2.3.5.1}{RFO4.2.3.5.1}\\
& \hyperlink{RFO4.2.3.5.2}{RFO4.2.3.5.2}\\ \hline
\hyperref[UC1.2.2.3.5.1]{UC1.2.2.3.5.1} & \hyperlink{RFO4.2.3.5.2}{RFO4.2.3.5.2}\\ \hline
\hyperref[UC1.2.2.3.5.2]{UC1.2.2.3.5.2} & \hyperlink{RFO4.2.3.5.1}{RFO4.2.3.5.1}\\ \hline
\hyperref[UC1.2.2.3.6]{UC1.2.2.3.6} & \hyperlink{RFF4.2.3.6}{RFF4.2.3.6}\\
& \hyperlink{RFF4.2.3.6.1}{RFF4.2.3.6.1}\\
& \hyperlink{RFF4.2.3.6.2}{RFF4.2.3.6.2}\\
& \hyperlink{RFF4.2.3.6.3}{RFF4.2.3.6.3}\\ \hline
\hyperref[UC1.2.2.3.6.1]{UC1.2.2.3.6.1} & \hyperlink{RFF4.2.3.6.1}{RFF4.2.3.6.1}\\
& \hyperlink{RFF4.2.3.6.2}{RFF4.2.3.6.2}\\ \hline
\hyperref[UC1.2.2.3.6.2]{UC1.2.2.3.6.2} & \hyperlink{RFF4.2.3.6.3}{RFF4.2.3.6.3}\\ \hline
\hyperref[UC1.2.2.3.7]{UC1.2.2.3.7} & \hyperlink{RFO4.2.3.7}{RFO4.2.3.7}\\ \hline
\hyperref[UC1.2.2.3.8]{UC1.2.2.3.8} & \hyperlink{RFD4.2.3.8}{RFD4.2.3.8}\\
& \hyperlink{RFD4.2.3.8.1}{RFD4.2.3.8.1}\\
& \hyperlink{RFD4.2.3.8.2}{RFD4.2.3.8.2}\\ \hline
\hyperref[UC1.2.2.3.8.1]{UC1.2.2.3.8.1} & \hyperlink{RFD4.2.3.8.1}{RFD4.2.3.8.1}\\ \hline
\hyperref[UC1.2.2.3.8.2]{UC1.2.2.3.8.2} & \hyperlink{RFD4.2.3.8.2}{RFD4.2.3.8.2}\\ \hline
\hyperref[UC1.2.2.3.9]{UC1.2.2.3.9} & \hyperlink{RFD4.2.3.9}{RFD4.2.3.9}\\ \hline
\hyperref[UC1.2.2.3.10]{UC1.2.2.3.10} & \hyperlink{RFD4.2.3.10}{RFD4.2.3.10}\\ \hline
\hyperref[UC1.2.2.3.11]{UC1.2.2.3.11} & \hyperlink{RFF4.2.3.11}{RFF4.2.3.11}\\ \hline
\hyperref[UC1.2.2.3.12]{UC1.2.2.3.12} & \hyperlink{RFO4.2.3.12}{RFO4.2.3.12}\\ \hline
\hyperref[UC1.2.2.3.13]{UC1.2.2.3.13} & \hyperlink{RFO4.2.3.13}{RFO4.2.3.13}\\ \hline
\hyperref[UC1.2.2.3.14]{UC1.2.2.3.14} & \hyperlink{RFF4.2.3.14}{RFF4.2.3.14}\\ \hline
\hyperref[UC1.2.2.4]{UC1.2.2.4} & \hyperlink{RFD4.2.4}{RFD4.2.4}\\ \hline
\hyperref[UC1.2.2.5]{UC1.2.2.5} & \hyperlink{RFO4.2.5}{RFO4.2.5}\\ \hline
\hyperref[UC1.2.2.6]{UC1.2.2.6} & \hyperlink{RFO4.2.6}{RFO4.2.6}\\ \hline
\hyperref[UC1.2.2.7]{UC1.2.2.7} & \hyperlink{RFO4.2.7}{RFO4.2.7}\\ \hline
\hyperref[UC1.2.2.8]{UC1.2.2.8} & \hyperlink{RFO4.2.8}{RFO4.2.8}\\ \hline
\hyperref[UC1.2.2.9]{UC1.2.2.9} & \hyperlink{RFD4.2.9}{RFD4.2.9}\\ \hline
\hyperref[UC1.2.3]{UC1.2.3} & \hyperlink{RFD4.3}{RFD4.3}\\
& \hyperlink{RFD4.3.1}{RFD4.3.1}\\
& \hyperlink{RFD4.3.2}{RFD4.3.2}\\
& \hyperlink{RFD4.3.3}{RFD4.3.3}\\ \hline
\hyperref[UC1.2.3.1]{UC1.2.3.1} & \hyperlink{RFD4.3.1}{RFD4.3.1}\\ \hline
\hyperref[UC1.2.3.2]{UC1.2.3.2} & \hyperlink{RFD4.3.2}{RFD4.3.2}\\ \hline
\hyperref[UC1.2.3.3]{UC1.2.3.3} & \hyperlink{RFD4.3.3}{RFD4.3.3}\\ \hline
\hyperref[UC1.2.4]{UC1.2.4} & \hyperlink{RFD4.4}{RFD4.4}\\
& \hyperlink{RFD4.4.1}{RFD4.4.1}\\
& \hyperlink{RFD4.4.2}{RFD4.4.2}\\
& \hyperlink{RFD4.4.3}{RFD4.4.3}\\
& \hyperlink{RFD4.4.4}{RFD4.4.4}\\
& \hyperlink{RFD4.4.5}{RFD4.4.5}\\
& \hyperlink{RFD19}{RFD19}\\ \hline
\hyperref[UC1.2.4.1]{UC1.2.4.1} & \hyperlink{RFD4.4.1}{RFD4.4.1}\\ \hline
\hyperref[UC1.2.4.2]{UC1.2.4.2} & \hyperlink{RFD4.4.2}{RFD4.4.2}\\ \hline
\hyperref[UC1.2.4.3]{UC1.2.4.3} & \hyperlink{RFD4.4.3}{RFD4.4.3}\\ \hline
\hyperref[UC1.2.4.4]{UC1.2.4.4} & \hyperlink{RFD4.4.4}{RFD4.4.4}\\ \hline
\hyperref[UC1.2.4.5]{UC1.2.4.5} & \hyperlink{RFD19}{RFD19}\\
& \hyperlink{RFD19.1}{RFD19.1}\\ \hline
\hyperref[UC1.2.5]{UC1.2.5} & \hyperlink{RFD4.5}{RFD4.5}\\ \hline
\hyperref[UC1.2.5.1]{UC1.2.5.1} & \hyperlink{RFD4.5.1}{RFD4.5.1}\\ \hline
\hyperref[UC1.2.6]{UC1.2.6} & \hyperlink{RFF4.6}{RFF4.6}\\
& \hyperlink{RFF4.6.1}{RFF4.6.1}\\
& \hyperlink{RFF4.6.2}{RFF4.6.2}\\
& \hyperlink{RFF4.6.3}{RFF4.6.3}\\ \hline
\hyperref[UC1.2.6.1]{UC1.2.6.1} & \hyperlink{RFF4.6.1}{RFF4.6.1}\\
& \hyperlink{RFF4.6.1.1}{RFF4.6.1.1}\\
& \hyperlink{RFF4.6.1.2}{RFF4.6.1.2}\\ \hline
\hyperref[UC1.2.6.1.1]{UC1.2.6.1.1} & \hyperlink{RFF4.6.1.1}{RFF4.6.1.1}\\ \hline
\hyperref[UC1.2.6.1.2]{UC1.2.6.1.2} & \hyperlink{RFF4.6.1.2}{RFF4.6.1.2}\\ \hline
\hyperref[UC1.2.6.2]{UC1.2.6.2} & \hyperlink{RFF4.6.2}{RFF4.6.2}\\ \hline
\hyperref[UC1.2.6.3]{UC1.2.6.3} & \hyperlink{RFF4.6.3}{RFF4.6.3}\\ \hline
\hyperref[UC1.2.7]{UC1.2.7} & \hyperlink{RFD4.7}{RFD4.7}\\ \hline
\hyperref[UC1.3]{UC1.3} & \hyperlink{RFO7}{RFO7}\\
& \hyperlink{RFD7.1}{RFD7.1}\\
& \hyperlink{RFO7.2}{RFO7.2}\\
& \hyperlink{RFO7.3}{RFO7.3}\\
& \hyperlink{RFD7.4}{RFD7.4}\\
& \hyperlink{RFD7.5}{RFD7.5}\\
& \hyperlink{RFD7.6}{RFD7.6}\\
& \hyperlink{RFO7.7}{RFO7.7}\\
& \hyperlink{RFO8}{RFO8}\\
& \hyperlink{RFO9}{RFO9}\\ \hline
\hyperref[UC1.3.1]{UC1.3.1} & \hyperlink{RFD7.1}{RFD7.1}\\ \hline
\hyperref[UC1.3.2]{UC1.3.2} & \hyperlink{RFO7.2}{RFO7.2}\\ \hline
\hyperref[UC1.3.3]{UC1.3.3} & \hyperlink{RFO7.3}{RFO7.3}\\ \hline
\hyperref[UC1.3.4]{UC1.3.4} & \hyperlink{RFD7.4}{RFD7.4}\\
& \hyperlink{RFD7.4.1}{RFD7.4.1}\\
& \hyperlink{RFD7.4.2}{RFD7.4.2}\\ \hline
\hyperref[UC1.3.4.1]{UC1.3.4.1} & \hyperlink{RFD7.4.1}{RFD7.4.1}\\ \hline
\hyperref[UC1.3.4.2]{UC1.3.4.2} & \hyperlink{RFD7.4.2}{RFD7.4.2}\\ \hline
\hyperref[UC1.3.5]{UC1.3.5} & \hyperlink{RFD7.5}{RFD7.5}\\ \hline
\hyperref[UC1.3.6]{UC1.3.6} & \hyperlink{RFD7.6}{RFD7.6}\\
& \hyperlink{RFD7.6.1}{RFD7.6.1}\\
& \hyperlink{RFD7.6.2}{RFD7.6.2}\\ \hline
\hyperref[UC1.3.6.1]{UC1.3.6.1} & \hyperlink{RFD7.6.1}{RFD7.6.1}\\ \hline
\hyperref[UC1.3.6.2]{UC1.3.6.2} & \hyperlink{RFD7.6.2}{RFD7.6.2}\\ \hline
\hyperref[UC1.4]{UC1.4} & \hyperlink{RFO10}{RFO10}\\ \hline
\hyperref[UC1.4.1]{UC1.4.1} & \hyperlink{RFO10.1}{RFO10.1}\\ \hline
\hyperref[UC1.4.2]{UC1.4.2} & \hyperlink{RFO10.2}{RFO10.2}\\ \hline
\hyperref[UC1.5]{UC1.5} & \hyperlink{RFO11}{RFO11}\\
& \hyperlink{RFO11.1}{RFO11.1}\\
& \hyperlink{RFO11.2}{RFO11.2}\\
& \hyperlink{RFO11.3}{RFO11.3}\\ \hline
\hyperref[UC1.5.1]{UC1.5.1} & \hyperlink{RFO11.3}{RFO11.3}\\ \hline
\hyperref[UC1.5.2]{UC1.5.2} & \hyperlink{RFO11.2}{RFO11.2}\\ \hline
\hyperref[UC1.5.3]{UC1.5.3} & \hyperlink{RFO11.1}{RFO11.1}\\ \hline
\hyperref[UC1.6]{UC1.6} & \hyperlink{RFD12}{RFD12}\\
& \hyperlink{RFF13}{RFF13}\\
& \hyperlink{RFD14}{RFD14}\\ \hline
\hyperref[UC1.6.1]{UC1.6.1} & \hyperlink{RFD12}{RFD12}\\
& \hyperlink{RFD12.1}{RFD12.1}\\
& \hyperlink{RFD12.2}{RFD12.2}\\
& \hyperlink{RFD12.3}{RFD12.3}\\ \hline
\hyperref[UC1.6.1.1]{UC1.6.1.1} & \hyperlink{RFD12.1}{RFD12.1}\\ \hline
\hyperref[UC1.6.1.2]{UC1.6.1.2} & \hyperlink{RFD12.2}{RFD12.2}\\ \hline
\hyperref[UC1.6.1.3]{UC1.6.1.3} & \hyperlink{RFD12.3}{RFD12.3}\\ \hline
\hyperref[UC1.6.2]{UC1.6.2} & \hyperlink{RFF13}{RFF13}\\
& \hyperlink{RFF13.1}{RFF13.1}\\
& \hyperlink{RFF13.2}{RFF13.2}\\
& \hyperlink{RFF13.3}{RFF13.3}\\ \hline
\hyperref[UC1.6.2.1]{UC1.6.2.1} & \hyperlink{RFF13.1}{RFF13.1}\\ \hline
\hyperref[UC1.6.2.2]{UC1.6.2.2} & \hyperlink{RFF13.2}{RFF13.2}\\ \hline
\hyperref[UC1.6.2.3]{UC1.6.2.3} & \hyperlink{RFF13.3}{RFF13.3}\\ \hline
\hyperref[UC1.6.3]{UC1.6.3} & \hyperlink{RFD14}{RFD14}\\
& \hyperlink{RFD14.1}{RFD14.1}\\
& \hyperlink{RFD14.2}{RFD14.2}\\
& \hyperlink{RFD14.3}{RFD14.3}\\
& \hyperlink{RFD14.4}{RFD14.4}\\ \hline
\hyperref[UC1.6.3.1]{UC1.6.3.1} & \hyperlink{RFD14.4}{RFD14.4}\\ \hline
\hyperref[UC1.6.3.2]{UC1.6.3.2} & \hyperlink{RFD14.2}{RFD14.2}\\ \hline
\hyperref[UC1.6.3.3]{UC1.6.3.3} & \hyperlink{RFD14.3}{RFD14.3}\\ \hline
\hyperref[UC1.6.3.4]{UC1.6.3.4} & \hyperlink{RFD14.1}{RFD14.1}\\ \hline
\hyperref[UC1.7]{UC1.7} & \hyperlink{RFD15}{RFD15}\\
& \hyperlink{RFO16}{RFO16}\\ \hline
\hyperref[UC1.7.1]{UC1.7.1} & \hyperlink{RFD15}{RFD15}\\
& \hyperlink{RFF15.1}{RFF15.1}\\
& \hyperlink{RFF15.2}{RFF15.2}\\
& \hyperlink{RFD15.3}{RFD15.3}\\ \hline
\hyperref[UC1.7.1.1]{UC1.7.1.1} & \hyperlink{RFF15.1}{RFF15.1}\\ \hline
\hyperref[UC1.7.1.2]{UC1.7.1.2} & \hyperlink{RFF15.2}{RFF15.2}\\ \hline
\hyperref[UC1.7.1.3]{UC1.7.1.3} & \hyperlink{RFD15.3}{RFD15.3}\\ \hline
\hyperref[UC1.7.2]{UC1.7.2} & \hyperlink{RFO16}{RFO16}\\
& \hyperlink{RFO16.1}{RFO16.1}\\
& \hyperlink{RFF16.2}{RFF16.2}\\
& \hyperlink{RFF16.3}{RFF16.3}\\
& \hyperlink{RFD16.4}{RFD16.4}\\ \hline
\hyperref[UC1.7.2.1]{UC1.7.2.1} & \hyperlink{RFD16.4}{RFD16.4}\\ \hline
\hyperref[UC1.7.2.2]{UC1.7.2.2} & \hyperlink{RFF16.2}{RFF16.2}\\ \hline
\hyperref[UC1.7.2.3]{UC1.7.2.3} & \hyperlink{RFF16.3}{RFF16.3}\\
& \hyperlink{RFF16.3.1}{RFF16.3.1}\\ \hline
\hyperref[UC1.7.2.4]{UC1.7.2.4} & \hyperlink{RFO16.1}{RFO16.1}\\ \hline
\hyperref[UC1.8]{UC1.8} & \hyperlink{RFF25}{RFF25}\\
& \hyperlink{RFF25.1}{RFF25.1}\\ \hline
\hyperref[UC1.8.1]{UC1.8.1} & \hyperlink{RFF25}{RFF25}\\
& \hyperlink{RFF25.1}{RFF25.1}\\ \hline
\hyperref[UC1.9]{UC1.9} & \hyperlink{RFD26}{RFD26}\\ \hline
\hyperref[UC1.10]{UC1.10} & \hyperlink{RFD31}{RFD31}\\
& \hyperlink{RFD31.1}{RFD31.1}\\
& \hyperlink{RFF31.2}{RFF31.2}\\ \hline
\hyperref[UC1.10.1]{UC1.10.1} & \hyperlink{RFD31.1.1}{RFD31.1.1}\\
& \hyperlink{RFD31.1.2}{RFD31.1.2}\\
& \hyperlink{RFD31.1.3}{RFD31.1.3}\\ \hline
\hyperref[UC1.10.1.1]{UC1.10.1.1} & \hyperlink{RFD31.1}{RFD31.1}\\
& \hyperlink{RFD31.1.1}{RFD31.1.1}\\ \hline
\hyperref[UC1.10.1.2]{UC1.10.1.2} & \hyperlink{RFD31.1.2}{RFD31.1.2}\\ \hline
\hyperref[UC1.10.1.3]{UC1.10.1.3} & \hyperlink{RFD31.1.3}{RFD31.1.3}\\ \hline
\hyperref[UC1.10.2]{UC1.10.2} & \hyperlink{RFF31.2}{RFF31.2}\\
& \hyperlink{RFF31.2.1}{RFF31.2.1}\\ \hline
\hyperref[UC1.10.2.1]{UC1.10.2.1} & \hyperlink{RFF31.2.1}{RFF31.2.1}\\ \hline
\hyperref[UC1.11]{UC1.11} & \hyperlink{RFO32}{RFO32}\\ \hline
\hyperref[UC2]{UC2} & \hyperlink{RFO30}{RFO30}\\
& \hyperlink{RFO30.1}{RFO30.1}\\
& \hyperlink{RFO30.2}{RFO30.2}\\
& \hyperlink{RFO30.3}{RFO30.3}\\
& \hyperlink{RFO30.4}{RFO30.4}\\ \hline
\hyperref[UC2.1]{UC2.1} & \hyperlink{RFO30.1}{RFO30.1}\\
& \hyperlink{RFO30.1.1}{RFO30.1.1}\\
& \hyperlink{RFO30.1.2}{RFO30.1.2}\\
& \hyperlink{RFO30.1.3}{RFO30.1.3}\\ \hline
\hyperref[UC2.1.1]{UC2.1.1} & \hyperlink{RFO30.1.1}{RFO30.1.1}\\ \hline
\hyperref[UC2.1.2]{UC2.1.2} & \hyperlink{RFO30.1.2}{RFO30.1.2}\\ \hline
\hyperref[UC2.1.3]{UC2.1.3} & \hyperlink{RFO30.1.3}{RFO30.1.3}\\ \hline
\hyperref[UC2.2]{UC2.2} & \hyperlink{RFO30.2}{RFO30.2}\\
& \hyperlink{RFO30.2.1}{RFO30.2.1}\\
& \hyperlink{RFO30.2.2}{RFO30.2.2}\\
& \hyperlink{RFO30.2.3}{RFO30.2.3}\\ \hline
\hyperref[UC2.2.1]{UC2.2.1} & \hyperlink{RFO30.2.1}{RFO30.2.1}\\ \hline
\hyperref[UC2.2.2]{UC2.2.2} & \hyperlink{RFO30.2.2}{RFO30.2.2}\\ \hline
\hyperref[UC2.2.3]{UC2.2.3} & \hyperlink{RFO30.2.3}{RFO30.2.3}\\ \hline
\hyperref[UC2.3]{UC2.3} & \hyperlink{RFO30.3}{RFO30.3}\\ \hline
\hyperref[UC2.4]{UC2.4} & \hyperlink{RFO30.4}{RFO30.4}\\ \hline
\hyperref[UC3]{UC3} & \hyperlink{RFD34}{RFD34}\\
& \hyperlink{RFD34.1}{RFD34.1}\\
& \hyperlink{RFD34.2}{RFD34.2}\\
& \hyperlink{RFD34.3}{RFD34.3}\\
& \hyperlink{RFD34.4}{RFD34.4}\\
& \hyperlink{RFD34.5}{RFD34.5}\\ \hline
\hyperref[UC3.1]{UC3.1} & \hyperlink{RFD34.5}{RFD34.5}\\ \hline
\hyperref[UC3.2]{UC3.2} & \hyperlink{RFD34.2}{RFD34.2}\\ \hline
\hyperref[UC3.3]{UC3.3} & \hyperlink{RFD34.3}{RFD34.3}\\ \hline
\hyperref[UC3.4]{UC3.4} & \hyperlink{RFD34.4}{RFD34.4}\\
& \hyperlink{RFD34.4.1}{RFD34.4.1}\\
& \hyperlink{RFD34.4.2}{RFD34.4.2}\\ \hline
\hyperref[UC3.4.1]{UC3.4.1} & \hyperlink{RFD34.4.1}{RFD34.4.1}\\ \hline
\hyperref[UC3.4.2]{UC3.4.2} & \hyperlink{RFD34.4.2}{RFD34.4.2}\\ \hline
\hyperref[UC3.5]{UC3.5} & \hyperlink{RFD34.1}{RFD34.1}\\
& \hyperlink{RFD34.1.1}{RFD34.1.1}\\
& \hyperlink{RFD34.1.2}{RFD34.1.2}\\ \hline
\hyperref[UC3.5.1]{UC3.5.1} & \hyperlink{RFD34.1.1}{RFD34.1.1}\\ \hline
\hyperref[UC3.5.2]{UC3.5.2} & \hyperlink{RFD34.1.2}{RFD34.1.2}\\ \hline
\caption[Tracciamento Fonti-Requisiti]{Tracciamento Fonti-Requisiti}
\label{tabella:fonti-requi}
\end{longtable}
\clearpage

\endgroup
\clearpage
\begingroup
\let\clearpage\relax
\subsection{Riepilogo Requisiti}
\normalsize
\begin{longtable}{|c|c|c|c|}
\hline
\textbf{Tipo} & \textbf{Obbligatorio} & \textbf{Desiderabile} & \textbf{Facoltativo}\\
\hline
Funzionale & 55 & 74 & 29\\ \hline
Prestazionale & 0 & 0 & 0\\ \hline
Di Qualità & 8 & 2 & 2\\ \hline
Di Vincolo & 7 & 5 & 1\\ \hline
\caption[Riepilogo Requisiti]{Riepilogo Requisiti}
\label{tabella:riepilogorequi}
\end{longtable}
\clearpage

\endgroup
\subsection{Requisiti accettati}
Tutti i requisiti obbligatori saranno implementati. A causa di tempo e risorse limitati alcuni dei requisiti desiderabili o facoltativi non potranno essere soddisfatti, saranno quindi soddisfatti solamente alcuni dei requisiti desiderabili o facoltativi, ci riserviamo però la possibilità di soddisfare un maggior numero di requisiti di quelli previsti nel caso tempo e risorse lo permettano.\\
\\
\label{reqAccettati}A seguito di un incontro con il \gloxy{proponente} e a successive osservazioni effettuate dai \rPs sono stati individuati nuovi requisiti. Dal momento che questi requisiti individuati sono stati ritenuti prioritari rispetto ad altri requisiti si è scelto di implementare i nuovi requisiti e di non andare a soddisfare alcuni di quelli precedentemente individuati.\\
I requisiti che ora non saranno più soddisfatti o che hanno subito dei cambiamenti di priorità sono requisiti identificati dal gruppo e non dal \gloxy{Proponente}, il prodotto quindi continuerà a soddisfare i requisiti minimi e il costo totale rimarrà invariato.
La lista aggiornata dei requisiti desiderabili o facoltativi che andremmo a soddisfare diventa quindi:
\begin{itemize}
\item \hyperlink{RFD4.1}{RFD4.1}: Un utente può modificare il nome del progetto;
\item \hyperlink{RFD4.2.3.9}{RFD4.2.3.9}: Un utente può spostare un elemento testuale;
\item \hyperlink{RFD4.2.3.10}{RFD4.2.3.10}: Un utente può spostare un’immagine;
\item \hyperlink{RFD4.2.3.15}{RFD4.2.3.15}: Un utente può dare un titolo ad un nodo;
\item \hyperlink{RFD4.2.3.16}{RFD4.2.3.16}: Un utente può modificare il titolo di un nodo;
\item \hyperlink{RFD4.2.3.17}{RFD4.2.3.17}: Un utente può selezionare il titolo di un nodo;
\item \hyperlink{RFD4.2.3.18}{RFD4.2.3.18}: Un utente può ridimensionare il titolo di un nodo;
\item \hyperlink{RFD4.2.3.19}{RFD4.2.3.19}: Un utente può ridimensionare un elemento testuale presente in un nodo
;
\item \hyperlink{RFD4.2.3.20}{RFD4.2.3.20}: Un utente può ridimensionare un’immagine presente in un nodo;
\item \hyperlink{RFD4.3}{RFD4.3}: Un utente può creare un percorso di presentazione personalizzato;
\item \hyperlink{RFD4.3.1}{RFD4.3.1}: Un utente può scegliere un nome per un percorso di presentazione personalizzato;
\item \hyperlink{RFD4.3.2}{RFD4.3.2}: Un utente può scegliere un nodo di una mappa mentale come primo frame di un percorso di presentazione personalizzato;
\item \hyperlink{RFD4.3.3}{RFD4.3.3}: Un utente può confermare la creazione di un percorso di presentazione personalizzato;
\item \hyperlink{RFD4.4}{RFD4.4}: Un utente può modificare un percorso di presentazione personalizzato;
\item \hyperlink{RFD4.4.1}{RFD4.4.1}: Un utente può aggiungere un frame al percorso di presentazione personalizzato;
\item \hyperlink{RFD4.4.3}{RFD4.4.3}: Un utente può eliminare un frame da un percorso di presentazione personalizzato;
\item \hyperlink{RFD4.4.5}{RFD4.4.5}: Un utente può modificare il nome di un percorso di presentazione personalizzato;
\item \hyperlink{RFD4.5}{RFD4.5}: Un utente può eliminare un percorso di presentazione personalizzato;
\item \hyperlink{RFD4.5.1}{RFD4.5.1}: Un utente può confermare l'eliminazione del percorso di presentazione selezionato;
\item \hyperlink{RFF4.6}{RFF4.6}: Un utente può scegliere le impostazioni generali dell’aspetto grafico del progetto;
\item \hyperlink{RFF4.6.1}{RFF4.6.1}: Un utente può scegliere il formato di default per il testo;
\item \hyperlink{RFF4.6.1.1}{RFF4.6.1.1}: Un utente può scegliere una famiglia di font di default per il testo;
\item \hyperlink{RFF4.6.1.2}{RFF4.6.1.2}: Un utente può scegliere un colore di default per il testo;
\item \hyperlink{RFF4.6.2}{RFF4.6.2}: Un utente può scegliere un colore di sfondo per i frame del progetto;
\item \hyperlink{RFF4.6.3}{RFF4.6.3}: Un utente può confermare le impostazioni scelte;
\item \hyperlink{RFD4.7}{RFD4.7}: Un utente può selezionare un percorso di presentazione personalizzato;
\item \hyperlink{RFD7.1}{RFD7.1}: Un utente può scegliere un percorso di presentazione relativo ad un progetto;
\item \hyperlink{RFD7.5}{RFD7.5}: Un utente può chiudere una presentazione;
\item \hyperlink{RFD7.6}{RFD7.6}: Un utente può passare ad un frame correlato al frame che sta visualizzando, dove per correlato si intende che tra i due nodi della mappa mentale c’è un’associazione padre-figlio oppure è stata creata un’associazione da parte dell’utente;
\item \hyperlink{RFD7.6.1}{RFD7.6.1}: Un utente può visualizzare tutti i frame correlati al frame che sta visualizzando;
\item \hyperlink{RFD7.6.2}{RFD7.6.2}: Un utente può selezionare un frame tra quelli mostrati dal sistema;
\item \hyperlink{RFD14}{RFD14}: Un utente può esportare la presentazione che sta visualizzando in pdf;
\item \hyperlink{RFD14.1}{RFD14.1}: Un utente può confermare l'esportazione in pdf;
\item \hyperlink{RFD14.2}{RFD14.2}: Un utente può scegliere la cartella in cui salvare una presentazione in pdf;
\item \hyperlink{RFD14.3}{RFD14.3}: Un utente può scegliere un nome per la presentazione;
\item \hyperlink{RFF16.2}{RFF16.2}: Un utente può visualizzare l'anteprima di stampa di una presentazione;
\item \hyperlink{RFF16.3}{RFF16.3}: Un utente può scegliere le impostazioni di pagina per la stampa di una presentazione;
\item \hyperlink{RFD17}{RFD17}: Un utente può eseguire una presentazione non lineare;
\item \hyperlink{RFD22}{RFD22}: Un utente può creare una mappa mentale;
\item \hyperlink{RFF25}{RFF25}: L'utente può chiudere il progetto correntemente caricato nel sistema;
\item \hyperlink{RFD26}{RFD26}: L'utente può consultare il manuale direttamente dall'applicazione;
\item \hyperlink{RFD33}{RFD33}: Il sistema deve notificare all’utente che è già presente un altro progetto con lo stesso nome;
\item \hyperlink{RFD35}{RFD35}: Il sistema deve comunicare all'utente gli errori di comunicazione tra la componente Back-End e la componente Front-End;
\end{itemize}

\subsection{Requisiti soddisfatti}
\label{soddisfatti}
Al termine dell'attività di \nogloxy{progetto}, come risulta dal tracciamento incluso nella tabella, risulta soddisfatta la \textbf{totalità dei requisiti obbligatori} e dei \textbf{requisiti desiderabili/facoltativi accettati}; è stato inoltre possibile implementare qualche requisito desiderabile/facoltativo non accettato anche se non vi era alcun vincolo al riguardo.
