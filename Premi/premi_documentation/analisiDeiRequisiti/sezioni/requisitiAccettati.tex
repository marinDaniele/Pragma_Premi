\begin{itemize}
\item \hyperlink{RFD4.1}{RFD4.1}: Un utente può modificare il nome del progetto;
\item \hyperlink{RFD4.2.3.9}{RFD4.2.3.9}: Un utente può spostare un elemento testuale;
\item \hyperlink{RFD4.2.3.10}{RFD4.2.3.10}: Un utente può spostare un’immagine;
\item \hyperlink{RFD4.2.3.15}{RFD4.2.3.15}: Un utente può dare un titolo ad un nodo;
\item \hyperlink{RFD4.2.3.16}{RFD4.2.3.16}: Un utente può modificare il titolo di un nodo;
\item \hyperlink{RFD4.2.3.17}{RFD4.2.3.17}: Un utente può selezionare il titolo di un nodo;
\item \hyperlink{RFD4.2.3.18}{RFD4.2.3.18}: Un utente può ridimensionare il titolo di un nodo;
\item \hyperlink{RFD4.2.3.19}{RFD4.2.3.19}: Un utente può ridimensionare un elemento testuale presente in un nodo
;
\item \hyperlink{RFD4.2.3.20}{RFD4.2.3.20}: Un utente può ridimensionare un’immagine presente in un nodo;
\item \hyperlink{RFD4.3}{RFD4.3}: Un utente può creare un percorso di presentazione personalizzato;
\item \hyperlink{RFD4.3.1}{RFD4.3.1}: Un utente può scegliere un nome per un percorso di presentazione personalizzato;
\item \hyperlink{RFD4.3.2}{RFD4.3.2}: Un utente può scegliere un nodo di una mappa mentale come primo frame di un percorso di presentazione personalizzato;
\item \hyperlink{RFD4.3.3}{RFD4.3.3}: Un utente può confermare la creazione di un percorso di presentazione personalizzato;
\item \hyperlink{RFD4.4}{RFD4.4}: Un utente può modificare un percorso di presentazione personalizzato;
\item \hyperlink{RFD4.4.1}{RFD4.4.1}: Un utente può aggiungere un frame al percorso di presentazione personalizzato;
\item \hyperlink{RFD4.4.3}{RFD4.4.3}: Un utente può eliminare un frame da un percorso di presentazione personalizzato;
\item \hyperlink{RFD4.4.5}{RFD4.4.5}: Un utente può modificare il nome di un percorso di presentazione personalizzato;
\item \hyperlink{RFD4.5}{RFD4.5}: Un utente può eliminare un percorso di presentazione personalizzato;
\item \hyperlink{RFD4.5.1}{RFD4.5.1}: Un utente può confermare l'eliminazione del percorso di presentazione selezionato;
\item \hyperlink{RFF4.6}{RFF4.6}: Un utente può scegliere le impostazioni generali dell’aspetto grafico del progetto;
\item \hyperlink{RFF4.6.1}{RFF4.6.1}: Un utente può scegliere il formato di default per il testo;
\item \hyperlink{RFF4.6.1.1}{RFF4.6.1.1}: Un utente può scegliere una famiglia di font di default per il testo;
\item \hyperlink{RFF4.6.1.2}{RFF4.6.1.2}: Un utente può scegliere un colore di default per il testo;
\item \hyperlink{RFF4.6.2}{RFF4.6.2}: Un utente può scegliere un colore di sfondo per i frame del progetto;
\item \hyperlink{RFF4.6.3}{RFF4.6.3}: Un utente può confermare le impostazioni scelte;
\item \hyperlink{RFD4.7}{RFD4.7}: Un utente può selezionare un percorso di presentazione personalizzato;
\item \hyperlink{RFD7.1}{RFD7.1}: Un utente può scegliere un percorso di presentazione relativo ad un progetto;
\item \hyperlink{RFD7.5}{RFD7.5}: Un utente può chiudere una presentazione;
\item \hyperlink{RFD7.6}{RFD7.6}: Un utente può passare ad un frame correlato al frame che sta visualizzando, dove per correlato si intende che tra i due nodi della mappa mentale c’è un’associazione padre-figlio oppure è stata creata un’associazione da parte dell’utente;
\item \hyperlink{RFD7.6.1}{RFD7.6.1}: Un utente può visualizzare tutti i frame correlati al frame che sta visualizzando;
\item \hyperlink{RFD7.6.2}{RFD7.6.2}: Un utente può selezionare un frame tra quelli mostrati dal sistema;
\item \hyperlink{RFD14}{RFD14}: Un utente può esportare la presentazione che sta visualizzando in pdf;
\item \hyperlink{RFD14.1}{RFD14.1}: Un utente può confermare l'esportazione in pdf;
\item \hyperlink{RFD14.2}{RFD14.2}: Un utente può scegliere la cartella in cui salvare una presentazione in pdf;
\item \hyperlink{RFD14.3}{RFD14.3}: Un utente può scegliere un nome per la presentazione;
\item \hyperlink{RFF16.2}{RFF16.2}: Un utente può visualizzare l'anteprima di stampa di una presentazione;
\item \hyperlink{RFF16.3}{RFF16.3}: Un utente può scegliere le impostazioni di pagina per la stampa di una presentazione;
\item \hyperlink{RFD17}{RFD17}: Un utente può eseguire una presentazione non lineare;
\item \hyperlink{RFD22}{RFD22}: Un utente può creare una mappa mentale;
\item \hyperlink{RFF25}{RFF25}: L'utente può chiudere il progetto correntemente caricato nel sistema;
\item \hyperlink{RFD26}{RFD26}: L'utente può consultare il manuale direttamente dall'applicazione;
\item \hyperlink{RFD33}{RFD33}: Il sistema deve notificare all’utente che è già presente un altro progetto con lo stesso nome;
\item \hyperlink{RFD35}{RFD35}: Il sistema deve comunicare all'utente gli errori di comunicazione tra la componente Back-End e la componente Front-End;
\end{itemize}
