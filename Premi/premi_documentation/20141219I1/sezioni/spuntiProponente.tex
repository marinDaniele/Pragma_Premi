\section{Spunti proponente}
\begin{itemize}
\item Può capitare di dover costruire una presentazione senza sapere quanto tempo si avrà 
a disposizione per esporla. 
Ad esempio, si può creare una presentazione strutturandola per essere esposta in 2 ore di tempo e, magari solo all'ultimo momento, 
si scopre che si hanno a disposizione solo 10 minuti per l'esposizione.
Dunque, il presentatore dovrà saltare molte slide per mancanza di tempo, e ciò ridurrà di molto il gradimento degli spettatori, che si vedranno nascondere molte slide;
\item Se consideriamo un'infografica (tabellone) con tanti piccoli riquadri (che sarebbero slide e sotto-slide di presentazione),
in versione stampata si vedranno tanti piccoli scarabocchi, difficilmente leggibili, 
mentre in presentazione sarà possibile fare una zoomata \textit{alla Prezi} con cui si andrà in dettaglio a vedere ciascuna presentazione.
\`E qui che lo \textit{storytelling} (cioè l'idea secondo cui la presentazione è un'infografica che racconta il tema di ciò che 
si vuole esporre) me lo presenta come un tabellone, ma, quando diventa una presentazione, quei puntini (che in forma stampata non riuscivo a capire cosa fossero), si rivelano delle sotto-slide che specificano ed espongono un gran numero di idee e concetti, impossibili da trattare nel tabellone per questioni di spazio;
\item \underline{Visual Understanding Environment(VUE)} progetto software open source universitario di presentazioni, realizzato in Java. 
Permette di realizzare una mappa mentale, partendo dai nodi, per poi creare una presentazione, definendo vari percorsi di esposizione;
\item Se in fase di costruzione si impone all'utente la creazione di una mappa mentale, 
ogni \underline{associazione} potrebbe essere caratterizzata con una domanda, ad esempio \textit{approfondisco? quindi? perché?}
Inoltre, i nodi potrebbero essere catalogati in nodi che espongono fatti, tesi, controtesi, in modo tale che la mappa mentale abbia una semantica;
\item Categorizzazione delle slide: slide che enuncia una tesi, slide che racconta un fatto, slide che giustifica,ecc.
Ancora prima di completare la stesura della presentazione è possibile che il sistema evidenzi che è presente una slide di tesi che non ha supporto alla tesi, 
perché quando qualcuno vedrà questa slide potrà chiedere argomentazioni a supporto di tale tesi.
Il sistema potrebbe, quindi, suggerire di aggiungere un'argomentazione per una slide "tesi" che ne sia sprovvista;
%Se questa tesi è vera, come arrivi a sostenerla? cioè se ci sono esempi o controesempi, %ma se si arriva a sostenerla e quindi?
%Quali sono le conseguenze di questa tesi?
\item Se la presentazione viene fatta in collaborazione è molto facile che ci siano nodi orfani, perché qualcuno li ha aggiunti ma magari non ha più collaborato; potrebbe essere utile mantenere lo storico delle versioni diverse;
\item Utilizzo di un tablet per controllare la presentazione effettuata dal computer.
Lo schermo del tablet può essere diviso in parti, in modo tale da poter visualizzare contemporaneamente sia la slide che si sta presentando (fornendo anche la possibilità di aggiungere elementi o disegni in tempo reale) sia l'intera mappa mentale;
\item L'utente deve essere agevolato nell'esposizione con delle tecniche che lo aiutino a non perdere il filo del discorso;
\item Esempio software \textit{Pioggia delle idee}: chiedeva di scrivere keyword e frasi e poi le faceva scorrere a caso sul video.
Sfruttando una mappa mentale semantica, uno strumento di questo genere può essere molto utile nel suggerire punti chiave su cui proseguire la discussione durante la presentazione, potrebbe andare nel web a cercare qualcosa in relazione con ciò che si è scritto.

\end{itemize}
