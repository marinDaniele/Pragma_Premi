\section{Domande e risposte}
Di seguito vengono riportate in grassetto le domande sollevate da \gruppo e in corsivo le risposte date da \referenteProponente, 
referente di \proponente.
\subsection{Capitolato}
\conversationEntry{\`E richiesta la realizzazione di effetti grafici a supporto dello storytelling$_G$, ma tra i requisiti opzionali funzionali 
viene citato appunto il supporto allo storytelling$_G$. Lo storytelling$_G$ è opzionale?}%
{Si, lo storytelling$_G$ è opzionale, ma devono essere disponibili effetti grafici per la
presentazione.
Infatti, sfruttando la potenza offerta dal CSS$_G$, si hanno a disposizione strumenti, come le transizioni, 
che permettono facilmente di produrre effetti grafici di alta qualità.
Ad esempio, Prezi ha introdotto l'idea di zoomata all'interno di una slide, mentre nella maggior parte dei software di presentazione 
è possibile effettuare solo transizioni da slide a slide. 
In ogni caso, il tema generale è fare in modo che la presentazione di slide superi l'idea di successione di slide, per giungere 
al concetto di storytelling$_G$, ossia quello di fare qualcosa che sia coerente dall'inizio alla fine.
Spesso, in fase di creazione di una presentazione, si cerca di creare delle slide che possano stupire lo spettatore, 
anzichè cercare di raccontare qualcosa che abbia un filo logico, però per poter raccontare qualcosa bisogna avere qualcosa da 
raccontare.}
\chrule
\conversationEntry{I requisiti minimi obbligatori, oltre che necessari, sono anche sufficienti?}%
{Si, sono necessari e sufficienti. Sono minimi nel senso che dal punto di vista della 
realizzazione sarà valutata l'aderenza a tali requisiti, però tutto ciò che verrà fuori in più sarà gradito.
In ogni caso, ciò che viene richiesto non è tanto la realizzazione, quanto la \underline{fantasia} e la \underline{ricerca}.}
\chrule
\conversationEntry{Lo scopo del progetto è la realizzazione di un software di presentazione di “slide” innovativo e non basato 
sul modello a cui la maggior parte delle applicazioni di questo genere, come PowerPoint, si sono finora ispirate. Però anche 
PowerPoint permette di creare, mostrare e stampare una presentazione. Rispetta dunque i requisiti minimi anch'esso?}
{Si, anche PowerPoint rispetta i requisiti minimi. 
E non esiste un requisito obbligatorio che vieta di costruire un software sulla falsariga di PowerPoint, perché la valutazione finale non potrebbe essere che soggettiva. 
Infatti, non è scontato che chiunque riesca a trovare qualcosa di innovativo, perciò non è stato inserito come requisito obbligatorio.}
\chrule
\conversationEntry{Qual è la differenza tra visualizzazione di presentazioni e visualizzazione di presentazioni via browser$_G$?}%
{Se il prodotto finale richiede plugin$_G$ esterni per funzionare 
nel browser$_G$, come Flash, viene ugualmente accettato. Però prima di andare a cercare tra le funzionalità offerte da plugin$_G$ esterni 
è bene avere già controllato che quella stessa funzionalità non venga già offerta dal browser$_G$.}
\chrule
\conversationEntry{La creazione di presentazioni da tablet è obbligatoria?}%
{No, la creazione di presentazioni da tablet è opzionale, però l'esecuzione di 
presentazioni da tablet è obbligatoria. 
In ogni caso, sarebbe comodo avere la possibilità di annotare le idee appena vengono in mente, eliminando il rischio di dimenticarle,
anche quando non si ha a disposizione un computer,
e ciò sarebbe possibile se la creazione e la modifica di presentazioni fossero disponibili nel proprio dispositivo mobile.}
\subsection{Funzionalità prodotto}
\conversationEntry{Potrebbe essere ragionevole imporre all'utente la creazione di un filo logico da condurre 
attraverso la presentazione, permettendogli comunque di avere a disposizione più strade alternative a partire da ciascun nodo?}%
{si.}
\chrule
\conversationEntry
{Potrebbe essere utile avere a disposizione una funzionalità che calcola/fa previsioni sul tempo mancante alla conclusione della presentazione, in 
relazione al tempo impiegato per spiegare le slide precedenti?}%
{Si.}
\chrule
\conversationEntry{Può essere accettabile inziare con la costruzione di una mappa mentale e trasformarla in presentazione?}%
{si.}
\chrule
\conversationEntry{Vorremmo consentire all'utente di qualificare i nodi con keywords inventate da loro stessi. Lo ritiene utile?}{Si.}
\chrule
\conversationEntry{Può essere ragionevole pensare di introdurre un formalismo da associare al concetto puro di mappa mentale,
per agevolare la realizzazione di presentazioni?}%
{Si, e potrebbe essere utile dare la possibilità di aggiungere tesi, supporto di tesi, contro tesi, controesempi, ecc.}
\chrule
\conversationEntry{In fase di creazione, potrebbe essere pratico, poter aggiungere/disporre di diversi layer: uno dedicato alla mappa mentale in 
formalismo puro e altri per eventuali aggiunte. Cosa ne pensa?}%
{Buona idea, si potrebbero tenere separati il layer della sequenza principale da quello degli approfondimenti, magari, in alcune occasioni semplicemente non si ha necessità (o possibilità) di esporre argomentazioni aggiuntive.
Quindi, si avrebbe una mappa mentale a layer con suddivisione basata sulla semantica e non sul disegno.}
\chrule
\conversationEntry{Pensiamo che, in fase di visualizzazione, possa essere utile disporre di un menù gerarchico, che 
mostri dove ci si trova rispetto al resto della mappa mentale e che permetta di passare ad un qualsiasi altro nodo della stessa. Che ne pensa?}%
{Certo, mostrando solo quelli appartenenti al layer corrente; poi può essere fornita la possibilità di vedere se c'è qualcosa anche in un altro layer, 
ma il suo contenuto potrebbe essere visualizzato solo se chiamato esplicitamente (cambio di layer), così da ridurre il numero di voci visibili
nel menù gerarchico, semplificandone la fruizione.}
\chrule
\conversationEntry{Potrebbe esserci una funzionalità, in fase di costruzione di presentazioni, che permetta l'aggiunta di tesi
e un pulsante ``validazione'' che controlli se nel canvas ci sono nodi isolati, se tutte le tesi hanno un nodo di argomentazione e uno di ''conseguenze''e, in caso negativo, presenti un avviso, ossia un suggerimento di aggiunta/modifica. Pensa che possa essere interessante offrire all'utente una funzione del genere?}{Si.}
\chrule
\conversationEntry{Crediamo possa essere utile poter creare, in fase di costruzione, una lista, ordinata o meno, degli argomenti ineludibili.}%
{Si, e man mano che, in visualizzazione, vengono attraversati possano anche essere tolti dalla lista degli argomenti da trattare.}
\chrule
\conversationEntry{Nel caso in cui l'utente scelga una sequenza per la trattazione di alcuni argomenti ineludibili, crediamo possa essere 
utile avere un \underline{reminder} durante la presentazione che ricordi qual è la prossima cosa da trattare.}%
{Si, e in questo lo schermo del presentatore sarebbe molto ricco e potrebbe presentare: prossima slide, tempo trascorso, argomenti trattati e 
non trattati, cose da dire e non dire. Inoltre potrebbe esserci la tesi esposta al centro e attorno le varie ragioni e motivazioni.}
