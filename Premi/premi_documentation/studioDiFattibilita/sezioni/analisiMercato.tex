\section{Analisi programmi proposti}
Il capitolato contiene una lista di programmi, già presenti sul mercato, dai quali è possibile trarre degli spunti per delle funzionalità da inserire il prodotto finale.\\
Questi programmi sono stati analizzati e, per ogni programma, è stata fatta una lista delle caratteristiche che si ritengono significative.
\subsection{PowerPoint 2013}
\paragraph{Funzionalità}
 \begin{itemize}
  \item \textbf{Modalità relatore:}
  \begin{itemize}
   \item \textbf{Zoom} per andare nel dettaglio di una slide;
   \item \textbf{Non sequenzialità:} passaggio da una diapositiva a qualsiasi altra, mediante un elenco delle slide visualizzabile solo nello schermo del relatore, 
   mentre il pubblico vede solo la diapositiva selezionata;
   \item \textbf{Note:} suggerimenti per il relatore.
  \end{itemize}
  \item \textbf{Condivisione e collaborazione} tramite cloud\ped{G};
  \item \textbf{Commenti} accanto al testo o all'immagine a cui fanno riferimento.
 \end{itemize}
\subsection{Impress}
\paragraph{Informazioni generali}
\begin{itemize}
 \item Creazione disegni e diagrammi;
 \item \textbf{Formati:} 
 \begin{itemize}
  \item OpenDocument;
  \item PowerPoint;
  \item Flash (solo esportazione).
 \end{itemize}
\end{itemize}
\paragraph{Modalità creazione}
 \subparagraph{Viste}
 \begin{itemize}
 \item \textbf{Normale:} per le modifiche generali;
 \item \textbf{Struttura:} per l'organizzazione e la struttura del contenuto testuale;
 \item \textbf{Note:} per vedere e modificare le note relative a una slide;
 \item \textbf{Stampati:} per la stampa delle diapositive;
 \item \textbf{Ordine dipositive:} per un'anteprima che permetta di trovare e ordinare velocemente le slide.
 \end{itemize}
\paragraph{Modalità visualizzazione}
\subparagraph{Funzionalità}
 \begin{itemize}
  \item Cambio diapositiva: manuale o a tempo;
  \item Puntatore o pulsanti navigazione: visibili o invisibili;
  \item Supporto di più monitor.
 \end{itemize}
\subparagraph{Presenter Console\ped{G}}
Estensione che permette di avere maggiore controllo sulle slide.

Funzionalità schermo relatore:
\begin{itemize}
  \item Pagina successiva;
  \item Note;
  \item Orologio;
  \item Timer.
\end{itemize}
\subsection{Keynote}
\begin{itemize}
  \item Inserimento immagini e filmati;
  \item Riflessi e cornici;
  \item Creazione grafici interattivi;
  \item Transizioni ed effetti per oggetti e testi;
  \item Anteprima, per veder il risultato finale direttamente nell’area di lavoro;
  \item Righelli e guide di allineamento;
  \item Editor di immagini integrato;
  \item Vista presentatore personalizzabile;
  \item Modifica note durante l'esposizione;
  \item Uso fino a sei monitor;
  \item Controllo a distanza, tramite dispositivo mobile equipaggiato con iOS\ped{G}, anche senza una rete Wi-Fi;
  \item Temi e design predefiniti;
  \item Pannello Formattazione dinamico: mostra le opzioni più adatte in base a ciò che viene selezionato;
  \item Suggeritore virtuale;
  \item Trasferimento presentazioni tra Mac, iPhone e iPad, senza mutarne l'aspetto;
  \item Accessibile via web\ped{G};
  \item Condivisione in sola lettura o anche in scrittura, anche protetta da password, via link, email, social, servizi di archiviazione \dots;
  \item Collaborazione, anche con utenti Powerpoint;
  \item Importazione PowerPoint;
  \item Esportazione in PowerPoint, PDF\ped{G}, QuickTime, HTML\ped{G}, Immagini o Keynote ’09;
  \item Evidenziatore, per mettere in risalto alcune parti delle slide durante la presentazione;
  \item Stampa commenti.
\end{itemize}
\subsection{Presentazioni Google}
\begin{itemize}
  \item Temi;
  \item Font;
  \item Video incorporati;
  \item Animazioni;
  \item Applicazioni web\ped{G} e mobile;
  \item Cloud\ped{G};
  \item Lavoro collaborativo;
  \item Condivisione;
  \item Commenti;
  \item Modifica in tempo reale: visualizzazione cursore corrispondente all'utente che sta effettuando la modifica;
  \item Chat\ped{G} di gruppo, inoltre aggiungendo il simbolo ``+'' seguito dall'indirizzo email del destinatario del messaggio è possibile inviargli una notifica;
  \item Salvataggio automatico in digitazione;
  \item Cronologia revisioni: permette di rivedere le vecchie versioni della stessa presentazione, ordinate per data e in base alla persona che ha eseguito la modifica;
  \item Importazione ed esportazione in formato PowerPoint;
  \item Modifica e esportazione offline.
\end{itemize}
\subsection{Prezi}
\subparagraph{Frame} (slide)
Cornice per il proprio contenuto o per controllare esattamente ciò che viene visto dagli spettatori;
chiamato \emph{frame} anziché \emph{slide} per differenziarlo dai software, che non usano un canvas 3D;
Tipi: rettangolo, cerchio, parentesi angolate o invisibile.
Operazioni disponibili: \textbf{cambio tipo} e \textbf{rimozione}.
\begin{itemize}
  \item \textbf{Contenuto}: frame, testo, immagini e video possono essere \emph{ruotati} in qualsiasi angolazione;
  \item \textbf{Layouts}: frame con layout preimpostato pronti per essere editati o layout multi-frame (diagramma) per semplificare l'organizzazione del contenuto;
  \item \textbf{My content}: semplifica l'aggiunta di contenuti di presentazioni precedenti, salvati dall'utente tra i preferiti.
  \`E comunque possibile aggiungere un frame creato in una qualsiasi presentazione precedente cercandola tramite search bar o scorrendo il menù delle presentazioni recenti.
  \item \textbf{Sfondi 3D} usando:
  \begin{itemize}
   \item Template\ped{G} realizzato in 3D, con frame pronti per essere edidati. Ogni template\ped{G} si adatta automaticamente quando si zooma avanti/indietro per creare effetti
   3D per il proprio contenuto;
   \item Immagini per creare uno sfondo 3D personalizzato;
  \end{itemize}
  \item \textbf{Sfondi 3D multipli} fino a 3: il secondo layer entra in dissolvenza appena si supera il primo terzo della strada, 
  e il terzo layer quando si supera il secondo terzo della strada. \`E necessario cambiare la dimensione e la dimensione del proprio contenuto per vedere
  apparire un certo sfondo.
\end{itemize}
\subparagraph{Modalità creazione}
\begin{itemize}
  \item \textbf{Navigazione} all'interno del canvas, attraverso \emph{drag and drop}, \emph{scroll}, \emph{bottone a lato} per zoom in/out o tornare all'overview, 
  \emph{doppio click} su un oggetto o \emph{click} sull'oggetto e sull'opzione di zoom a tale oggetto;
  \item \textbf{Inserimento di contenuti}:
  \begin{itemize}
   \item Video, da file o tramite link Youtube (necessaria connessione a internet);
   \item Immagine, da file o dal web\ped{G}, a cui è possibile applicare numerosi effetti;
   \item Collegamenti ipertestuali, aggiungendo il prefisso ``http://'', il cui click in modalità presentazione comporta l'apertura di una nuova finestra;
   \item audio:
   \begin{itemize}
    \item Come colonna sonora dell'intera presentazione;
    \item Per specifici punti del percorso di presentazione.
   \end{itemize}
  \end{itemize}
  \item \textbf{Importazione presentazione PowerPoint}:
  \begin{itemize}
   \item Barra laterale destra mostra slide importate;
   \item Inserimento di slide singole;
   \item Inserimento di un gruppo di slide, con la possibilità di sceglierne il layout e l'aggiunta al percorso di presentazione;
  \end{itemize}
  \item \textbf{Personalizzazione} dell'\textbf{aspetto} della \textbf{presentazione}, 
  scegliendo un tema preesistente, inserendo codici RGB, o scegliendo i colori da una palette;
  \item \textbf{Aggiunta} di \textbf{frame} al canvas, tramite \emph{click} o \emph{drag and drop}. 
  Viene automaticamente aggiunto in coda come passo del percorso di presentazione;
  \item \textbf{Scelta} del \textbf{tipo} di \textbf{frame} (\emph{Bracket}, \emph{Circle}, \emph{Rectangle} o \emph{Invisible}) da aggiungere;
  \item \textbf{Modifica percorso}:
  \begin{itemize}
    \item \textbf{Aggiunta passo} al percorso di presentazione:
    \begin{itemize}
      \item \textbf{Vista corrente};
      \item \textbf{Oggetto}: 
      \begin{itemize}
	\item \textbf{In coda}, aggiungendo un nuovo frame o cliccando su un oggetto del canvas non presente nel percorso;
	\item \textbf{In mezzo} a 2 punti del percorso, trascinando la linea di collegamento tra i due frame corrispondenti.
      \end{itemize}
    \end{itemize}
    \item \textbf{Rimozione} di passi dal percorso di presentazione, tramite \emph{click};
    \item \textbf{Cambio ordine passi} all'interno del percorso di presentazione, tramite \emph{drag and drop};
    \item \textbf{Animazione} del contenuto di un frame con l'effetto dissolvenza in entrata, 
    tramite \emph{click} nell'ordine di comparsa desiderato.
  \end{itemize}
\end{itemize}
\subparagraph{Modalità presentazione}
\begin{itemize}
 \item \textbf{Visualizzazione a schermo intero}, uscita con tasto ``ESC'';
 \item \textbf{Spostamento} nella presentazione:
 \begin{itemize}
  \item in \textbf{avanti} nel percorso, tramite l'uso della freccia a destra del \emph{menù di navigazione} o della \emph{tastiera};
  \item all'\textbf{indietro} nel percorso, tramite l'uso della freccia a sinistra del \emph{menù di navigazione} o della \emph{tastiera};
  \item \textbf{Zoom in} a frame, tramite \emph{click} sul frame stesso;
  \item \textbf{Zoom out}, tramite \emph{click} fuori da frame.
 \end{itemize}
 \item \textbf{Navigazione automatica}, impostazione del tempo di attesa prima di uno spostamento automatico: 4, 10 o 20 secondi.
\end{itemize}
\subsection{Visme.co}
Piattaforma\ped{G} per creare presentazioni, infografiche, banner\ped{G} pubblicitari e animazioni.
\paragraph{Presentazioni}
\begin{itemize}
 \item Stile sfondo: nessuno, tinta unita, gradiente o immagine;
 \item \textbf{Inserimento} di 
 \begin{itemize}
  \item \textbf{Testo}: classico, animato o artistico;
  \item \textbf{Forme} e \textbf{icone}: linee, frecce, clipart, loghi, \dots;
  \item \textbf{Immagini} da file o ricercate sul web\ped{G} tramite campo di ricerca integrato;
  \item \textbf{Elementi infografici};
  \item \textbf{Video} tramite link \textbf{Youtube o Vimeo};
  \item \textbf{Audio} tramite url;
  \item \textbf{iframe} via codice.
 \end{itemize}
 \item \textbf{Creazione grafici} e \textbf{diagrammi}, possibile importazione dati;
 \item \textbf{Animazione} oggetti con effetti di entrata e/o di uscita;
 \item \textbf{Condivisione} pubblica/privata(password) tramite link/social/email;
 \item \textbf{Integrazione} via codice in pagine web\ped{G};
 \item \textbf{Esportazione} nei formati JPG, PDF\ped{G} e HTML\ped{G};
 \item \textbf{Stampa}.
\end{itemize}
\subsection{RealTime board}
Piattaforma web\ped{G} che permette di creare lavagne virtuali illimitate che mantengono i dati nel cloud\ped{G}.\\
Il prodotto è stato creato per agevolare la collaborazione fra più persone, per esempio: brainstorm, working on projects, product design, educazione, \dots

\paragraph{Funzionalità}
\begin{itemize}
\item Creazione lavagna: pubblica o privata (su invito del creatore);
\item Aggiunta delle idee sulla lavagna;
\item Coinvolgimento di tutto il team\ped{G} nello sviluppo del proprio concetto;
\item Collaborazione real-time, aggiunta di immagini e video, note, commenti, tickers;
\item Caricamento immagini o PDF\ped{G} dal proprio dispositivo o dal web\ped{G};
\item Rotazione, ridimensionamento immagini, spostamento nella lavagna e anche scorrere pagine e PDF\ped{G};
\item Connessione al proprio account Google Drive\ped{G} e uso dei propri documenti nella lavagna, possibile editazione e salvataggio dei cambiamenti e scaricamento del file;
\item Pennello per disegnare e gomma per cancellare;
\item Inserimento forme, possibile cambiare colore e stile di riempimento;
\item Selezionando più oggetti e possibile applicare cambiamenti a tutto il gruppo;
\item Inserimento post-it;
\item Inserimento testo ruotabile e ridimensionabile;
\item Inserimento commenti;
\item Collegamento tra oggetti;
\item Per costruire presentazioni viene fornito un toolkit e per creare le slide basta muoversi sulla lavagna e fotografare le schermate che vogliamo vedere in presentazione;
\item Esportabile come pdf\ped{G}.
\end{itemize}
Lavagna virtuale inesauribile accessibile via web\ped{G} browser\ped{G}.
La classica lavagna riadattata per il mondo web\ped{G}, in cui è possibile memorizzare tutte proprie idee in un tavolo illimitato e collaborare insieme ad altre persone allo sviluppo delle proprie idee.

\begin{itemize}
\item Ogni tavolo è \emph{illimitato}: è possibile aggiungere centinaia di file e oggetti;
\item \`E disponibile nel browser\ped{G}, e salvato online;
\item Si può invitare un intero team\ped{G} e collaborare in tempo reale.
\end{itemize}
Usi: design, scuola, brainstorming, project management, \dots.
Caratteristiche principali
\begin{itemize}
 \item \textbf{Aggiunta} di
 \begin{itemize}
  \item Immagini;
  \item Video;
  \item Note;
  \item Documenti.
 \end{itemize}
 \item \textbf{Collaborazione}
\end{itemize}
\subsection{Mural.ly}
Permette di creare lavagne virtuali in cui è possibile collezionare qualsiasi contenuto. 

\paragraph{Funzionalità}
\begin{itemize}
\item Trascinare immagini, testo, suoni e video da qualunque sito web\ped{G} o dal proprio computer;
\item Permette di sistemare il proprio contenuto in modo ordinato e flessibile in un grande spazio;
\item Collaborazione;
\item Creazione presentazioni con il contenuto presente nella lavagna.
\end{itemize}

\subsection{canva.com}
Canva fornisce tutto il necessario per poter trasformare agevolmente le proprie idee in opere di design.
Viene usato per creare design sia per il web\ped{G} che per la stampa: grafiche, presentazioni, copertine, volantini, poster, inviti, \dots.
\begin{itemize}
 \item \textbf{Ricerca e trascina} creazione design sfruttando la potenza del web\ped{G}. 
 \item Layout personalizzati 
 \item \textbf{Online e gratuito}
 \`ricerca/caricamento grafiche, foto e font 
 
 Centinaia di elementi e font gratuiti a disposizione per creare un design, 
 o scegliendoli dalla propria libreria\ped{G} contenente più di un milione di immagini.  
 \item \textbf{Editor foto} integrato con
 filtri di luminosità, constrasto, saturazione, colore, ombra, \dots
 \item \textbf{Collaborazione}: condivisione e modifica.
\end{itemize}
\subsection{Easel.ly}
Easel è un'applicazione web\ped{G} per la creazione di infografiche e visualizzazioni di dati.

\paragraph{Funzionalità}
\begin{itemize}
\item \textbf{Scelta} tema, disposizione foglio e sfondo;
\item \textbf{Inserimento} oggetti, figure e testo;
\item \textbf{Creazione} diagrammi e grafici;
\item \textbf{Caricamento} immagini da file;
\item \textbf{Esportazione} in PDF\ped{G} o JPG di alta o bassa qualità;
\item \textbf{Condivisione} tramite URL, tramite integrazione del codice in una pagina web\ped{G} o ad un gruppo;
\item \textbf{Undo}.
\end{itemize}

\subsection{Piktochart}
Creatore di infografiche facile da usare.

\paragraph{Funzionalità}
\begin{itemize}
\item Selezione di un tema da una galleria che contiene più di 100 temi organizzati per categorie;
\item Personalizzazione infografiche mediante l'uso di intuitivi strumenti di editazione;
\item Inserimento immagini scelte tra più di un migliaio presenti in libreria\ped{G} o caricane di personali;
\item Dare vita ai dati con una varietà di opzioni di visualizzazione;
\item Condivisione:
 \begin{itemize}
 \item Stampe ad alta risoluzione o condivisione online;
 \item Link, integrazione, email o condivisione nei social.
 \end{itemize}
\item Integrazione Slideshare e Evernote.
\end{itemize}

\paragraph{Tipologie di infografiche}
\begin{itemize}
\item \textbf{Standard}:
i template\ped{G} di infografica\ped{G} sono una guida per aiutare l'utente nella personalizzazione e creazione delle infografiche.
Una infografica\ped{G} contiene dati provenienti svariate sorgenti - complesse e semplici. 
Sono versatili e possono essere integrate in una pagina web\ped{G}, condivise via social o allegate via email;
\item \textbf{Report}:
piuttosto di restringerci a diagrammi e tabelle, l'utente viene potenziato per poter creare infografiche di report.
\`E stato realizzato per essere stampato in 2 fogli A4;
\item \textbf{Banner\ped{G}}:
l'utente può trasformare le proprie infografiche in poster, immagini copertina, banner\ped{G} pubblicitari o qualsiasi altra cosa che attiri l'attenzione.
Qualsiasi sia la dimensione richiesta, è possibile aggiustare, è sufficiente sistemare la dimensione del canvas. 
\`E stato realizzato per adattarsi perfettamente in un singolo foglio A4, disposto in verticale o orizzontale;
\item \textbf{Presentazione}:
è possibile creare presentazioni sottoforma di infografiche.
Una presentazione è composta da una serie di blocchi e ciascun blocco è adatto a schermi 4:3 e la sua dimensione è fissata a 800px di larghezza per 600px di lunghezza.
Una funzionalità permette una rapida condivisione su SlideShare.
\end{itemize}

\paragraph{Funzionalità}
\begin{itemize}
\item Infografiche senza curva di apprendimento;
\item Creazione di infografiche esplicative e coinvolgenti in 3 semplici passi;
\item Inserire rapido di un elemento tramite ricerca nel menù e trascinamento nel canvas;
\item Strumenti familiari e di intuitivo utilizzo;
\item Rapidità nella creazione e modifica;
\item Visualizzazione dati con grafici, diagrammi e mappe;
\item Importazione di file xls, xlsx, csv o l'uso di Google Sheets;
\item Possibilità di avere un alto livello di personalizzazione infografiche;
\item Ampia selezione di grafiche a tema di alta risoluzione e a qualità di stampa;
\item Scelta fra più di 2000 grafici in formato SVG\ped{G} per personalizzare l'infografica\ped{G};
\item Strumenti di editazione facili da utilizzare semplificano la creazione di infografiche;
\item Condivisione via Facebook, Twitter, Pinterest e Google+;
\item Formati e taglie multiple per scaricare le infografiche e stamparle;
\item Aggiungi grafico;
\item Uso del trascinamento per aggiungere elementi grafici al canvas;
\item Inserisci la parola chiave nel campo di ricerca;
\item Editazione testo, tipo e dimensione del carattere, colori e allineamento del testo;
\item Caricamento immagini;
\item Inserimento forme e linee;
\item Organizzazione a blocchi clonabili, spostabili e eliminabili;
\item Personalizzazione dello sfondo con la scelta di un colore o un tema e con l'aggiustamento dell'opacità;
\item Creazione diagrammi;
\item Esportazione nei formati JPEG, PNG\ped{G} e PDF\ped{G};
\item Condivisione via email, Slideshare, Evernote e social;
\item Inserimento video Youtube e Vimeo.
\end{itemize}

\subparagraph{Modalità presentazione}
Modo interattivo per vedere un'infografica\ped{G} un pezzo alla volta.
\subsection{Applicazioni mobile}
\subsubsection{Imprys lite}
Nuovo modo di fare presentazioni utilizzando una tela infinita.
\paragraph{Funzionalità}
\begin{itemize}
\item Scorrimento del dito sullo schermo per far mostrare la diapositiva successiva;
\item Esportazione su file unico HTML\ped{G} per facilitare invio e condivisione;
\item Testo e immagini posizionabili liberamente su una tela infinita;
\item Zoom in qualsiasi luogo, in qualsiasi orientamento e qualsiasi scala sulla tela per una sequenza di presentazione;
\item Esportazione di qualsiasi presentazione in un singolo file HTML\ped{G}.
\end{itemize}
Sono presenti varie differenze fra La versione Lite e quella completa che vengono descritte in seguito.
La versione lite permette di salvare un solo file alla volta, completamente accessibile dalla zona di condivisione file di iTunes, non è però possibile lavorare con più di un file alla volta. 
Invece, la versione completa ha una finestra di dialogo di caricamento del file in modo da poter lavorare con più file diversi alla volta.
Con la versione lite è possibile creare presentazione costituite al più di 15 elementi, al contrario della versione completa, che permette di gestire quanti elementi si voglia.

\subsubsection{FlowVella}
Applicazione per la creazione di presentazioni interattive con immagini, testo, video, PDF\ped{G}, link e gallerie di foto e con la possibilità di condividerle.\\
Da la possibilità di creare presentazioni interattive, divertenti, che rendono coinvolgente l'esplorazione.
Disponibile per iPad e Mac.
\paragraph{Funzionalità}
\begin{itemize}
 \item Selezione di template\ped{G} da una galleria;
 \item Possibilità di aggiungere testo, immagini, video, YouTube \& Vimeo, documenti e gallerie;
 \item Interfaccia pulita e intuitiva che permette a ciascuno di disporre slide con precisione al pixel;
 \item Strumenti per la ricerca, posizionamento, rotazione e ritaglio di immagini;
 \item Video integrati per consentirne la riproduzione anche offline o streaming da YouTube;
 \item Integrazione di PDF\ped{G} o vecchie presentazioni PowerPoint nel flusso di presentazione per poterle lanciare durante la presentazione;
 \item Condivisione tramite link URL univoco del flusso di presentazione;
 \item Accesso tramite le risorse cloud\ped{G} più usate, in modo tale da rendere i media facilmente raggiungibili;
 \item Lavoro al sicuro anche in caso di perdita o rottura del dispositivo, poichè avviene il salvataggio automatico nel cloud\ped{G} delle bozze;
 \item \textbf{Undo \& Redo}.
\end{itemize}
\paragraph{Crea e personalizza}
\begin{itemize}
\item Aggiungi transizioni al tuo screen links;
\item Aggiungi collegamenti a pagine web\ped{G} o ad altre slide;
\item Annulla e ripeti ogni azione o cambiamento;
\item Duplica, cancella o riarrangia schermate;
\item Duplica ciascun oggetto, inclusi interi flussi.
\end{itemize}
\paragraph{Condividi, mostra e presenta}
\begin{itemize}
\item Esportazione in PDF\ped{G};
\item Stampa tramite AirPrint;
\item Presentazione diretta dall'iPad connesso al proiettore o tramite AirPlay;
\item Lavoro offline;
\item Condivizione tramite URL, Facebook, Twitter o email;
\item Integrazione all'interno di una pagina web\ped{G}.
\end{itemize}
\subsubsection{Haiku Deck}
Disponibile sul web\ped{G} e per iPad.
\paragraph{Funzionalità}
\begin{itemize}
 \item Scelta di temi grafici;
 \item Selezione del formato;
 \item Inserimento dello sfondo, solo nelle forme: tinta unita o immagine; 
 \item Offre un motore di ricerca iconografico interno; 
 \item Caricamento immagini da memoria, URL o Flickr, Picasa, Instagram, Facebook, Dropbox, Google Drive\ped{G}, Evernote, box;
 \item Inserimento diagrammi;
 \item Selezione layout per il testo;
 \item Esportazione in PDF\ped{G} o PowerPoint;
 \item Condivisione via URL.
\end{itemize}
\subsection{InkScape plugin}
\subsubsection{JessyInk}
Javascript\ped{G} che può essere incorporato in una immagine SVG\ped{G} creata con Inkscape e contenente più livelli. 
Ciascun livello può essere convertito in una slide di una presentazione. 

Le funzionalità correnti includono: , effetti, un foglio indice, 
\paragraph{Funzionalità}
\begin{itemize}
\item Transizioni di slide;
\item Effetti grafici;
\item Possibilità di avere un foglio indice, una slide master e testo automatico come titolo e numero della slide e numero complessivo di slide. 
\end{itemize}
\subsubsection{Sozi}
Editor e player di presentazioni zoomate.
Un documento Sozi non è organizzato come una sequenza di slide, ma come un poster 
e il contenuto della presentazione può essere disposto liberamente al suo interno.
Una presentazione così formata consiste di una serie di transizioni, zoomate e rotazioni 
che permettono di mettere a fuoco l'elemento che si vuole mostrare.

\begin{landscape}
\def\arraystretch{2}
\scriptsize
\setlength{\tabcolsep}{3pt}
\begin{longtable}{|>{\centering}>{\columncolor{gray!30}}p{1.6cm}|>{\centering}>{\columncolor{gray!30}}p{1.6cm}|>{\centering}m{3cm}|>{\centering}m{3cm}|>{\centering}m{1.5cm}|>{\centering}m{2cm}|>{\centering}m{1.5cm}|m{1.8cm}<{\centering}|}\hline
\multicolumn{2}{|c|}{\diagbox[width=3.5cm,dir=NW]{Funzionalità}{Prodotti}}						& \textbf{Visme}																		& \textbf{Canva}								& \textbf{Easelly}				& \textbf{Piktochart}								& \textbf{RealTime board}			& \textbf{Murally}\\ \hline \endhead
\multicolumn{2}{|c|}{\cellcolor{gray!30} layout predefiniti}								& \si																				& \si										& \si						& \si										& \no						& \no\\ \hline
\multicolumn{2}{|c|}{\cellcolor{gray!30} personalizzazione sfondo}							& tinta unita, gradiente, tema e immagine															& tinta unita, tema e immagine							& tinta unita e tema				& tinta unita e tema								& \no						& bianco, grigio o nero\\ \hline
inserimento								& testo						& classico, animato e artistico																	& classico e artistico								& \si						& classico e artistico								& \si						& \si\\ \Cline{1}{7}
									& forme e icone					& linee, frecce, clipart, loghi, \dots																& \no										& \si						& \si										& \si						& \si\\ \Cline{1}{7}
									& immagini					& da file, motore di ricerca integrato																& da file, motore di ricerca integrato, Facebook				& \si						& \si										& \si						& \si\\ \Cline{1}{7}
									& video						& tramite URL Youtube o Vimeo																	& \no										& \no						& tramite URL Youtube o Vimeo							& \si						& tramite URL\\ \Cline{1}{7}
									& audio						& tramite URL																			& \no										& \no						& \no										& \si						& tramite URL\\ \Cline{1}{7}
									& iframe					& \si																				& \no										& \no 						& \no										& \no						& \no\\ \hline
\multicolumn{2}{|c|}{\cellcolor{gray!30} creazione grafici e diagrammi}							& \si \\ possibile importazione dati																& \no										& \si						& \si \\ possibilità di importazione da file o da Google Sheets (dinamico)	& \si						& \no\\ \hline
\multicolumn{2}{|c|}{\cellcolor{gray!30} animazioni}									& \si \\ effetti di entrata e/o di uscita: scorrimento da sinistra/destra/alto/basso, dissolvenza in entrata e salto fuori, scelta durata entrata e uscita	& \no										& \no						& \si \\ effetti di transizione: default, concave, fade, linear e zoom		& \no						& \no\\ \hline
condivisione								& link						& \si \\ accesso proteggibile con password															& \si										& \si						& \si \\ pubblico o accessibile solo da amici					& \si						& \begin{center}\si\end{center}con privilegi di lettura o scrittura proteggibile con password\\ \Cline{1}{7}
									& invito via email				& \si																				& \si										& \no						& \si										& \si						& \si\\ \Cline{1}{7}
									& social					& \si																				& Facebook, Twitter e Pinterest							& \no						& Facebook, Twitter, Google+ e Pinterest, \dots					& Facebook					& \no\\ \Cline{1}{7}
									& hosting\ped{G} service				& \si																				& \no										& \no						& SlideShare e Evernote, \dots							& Google Drive\ped{G}					& \no\\ \Cline{1}{7}
									& embed code generator				& \si																				& \no										& \si						& \si										& \si						& \si\\ \Cline{1}{7}
									& gruppo interno				& \no																				& \no										& \si						& \no										& \si						& \no\\ \hline
\multicolumn{2}{|c|}{\cellcolor{gray!30} esportazione}									& JPG, PDF\ped{G} e HTML\ped{G}																		& PDF\ped{G} di alta qualità e PNG\ped{G}							& JPG di alta e bassa qualità e PDF\ped{G}		& PNG\ped{G}, JPG e PDF\ped{G} di varie qualità						& PNG\ped{G} di bassa, media e alta qualità e PDF\ped{G}	& PNG\ped{G} e HTML\ped{G}\\ \hline
\multicolumn{2}{|c|}{\cellcolor{gray!30} stampa}									& \si																				& \no										& \no						& \si										& \no						& \no\\ \hline
\multicolumn{2}{|c|}{\cellcolor{gray!30} editor foto}									& \no																				& \si										& \no						& \no										& \no						& \no\\ \hline
\multicolumn{2}{|c|}{\cellcolor{gray!30} Undo}										& \si																				& \si										& \si						& \si										& \si						& \si\\ \hline
\multicolumn{2}{|c|}{\cellcolor{gray!30} Redo}										& \no																				& \si										& \no						& \si										& \si						& \si\\ \hline
\caption{Tabella di comparazione delle principali funzionalità offerte da prodotti Web\ped{G} per la realizzazione di infografiche e presentazioni.}
\label{tabella:confrontoProdottiWebInfografichePresentazioni}
\end{longtable}
\end{landscape}

