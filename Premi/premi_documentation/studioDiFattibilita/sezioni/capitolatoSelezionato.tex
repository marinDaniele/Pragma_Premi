\section{Capitolato scelto: C4 - Premi}

\subsection{Descrizione}
Il capitolato proposto da \proponente riguarda la realizzazione di un software per un sistema di presentazione di slide. È richiesto che il software disponga almeno delle funzionalità di \emph{creazione}, \emph{esecuzione} e \emph{stampa} di una presentazione. \\
Tutte le funzionalità del software devono essere direttamente disponibili sul browser\ped{G} del computer dell'utente. Deve essere inoltre garantita la possibilità di visualizzare le presentazioni create su dispositivi mobile quali smartphone e tablet.

\subsection{Dominio tecnologico}

Per realizzare il prodotto oggetto del capitolato vengono richieste al gruppo conoscenze legate all'ambito web\ped{G} quali:
\begin{itemize}
\item \textbf{HTML5\ped{G}:} necessarie per strutturare l'applicazione e le presentazioni create con essa;
\item \textbf{CSS3:} necessarie per definire l'aspetto grafico dell'applicazione e per poter creare effetti di presentazione accattivanti:
\item \textbf{Javascript\ped{G}:} necessario per poter implementare la logica di funzionamento dell'applicativo.
\end{itemize}

Nel capitolato è presente una lista di framework\ped{G} e librerie\ped{G} che possono essere utili allo sviluppo del prodotto. L'analisi di queste librerie\ped{G} viene riassunta nella seguente tabella.

\begin{landscape}
\def\arraystretch{2}
\scriptsize
\begin{longtable}{|>{\centering}>{\columncolor{gray!30}}p{1.8cm}|>{\centering}p{1.7cm}|>{\centering}p{1.3cm}|>{\centering}p{0.9cm}|>{\centering}p{1.7cm}|>{\centering}p{1.3cm}|>{\centering}p{1.0cm}|>{\centering}p{1.3cm}|>{\centering}p{1.3cm}|>{\centering}p{1.5cm}|>{\centering}p{1.3cm}|>{\centering}p{1.2cm}|p{1.3cm}<{\centering}|}\hline
\multicolumn{1}{|c|}{\diagbox[width=2.2cm,dir=NW]{Caratteristiche}{Librerie\ped{G}}}	& \textbf{Jmpress.js}																																	& \textbf{Reveal.js}																											& \textbf{Shower}			& \textbf{Google slides template}																																																						& \textbf{Deck.js}																								& \textbf{Tacion.js}																				& \textbf{DZ Slides}																											& \textbf{Impress.js}																														& \textbf{Presenteer.js}																				& \textbf{Slides}																											& \textbf{Fathom.js}								& \textbf{Perkele.js}\\ \hline \endhead
Vista panoramica								& \si																																			& \si																													& \si					& \si																																																										& \si																										& \no																						& \no																													& \si																																& \si																							& \no																													& \no										& \no\\ \hline
3D										& \si																																			& \no																													& \no					& \no																																																										& \no																										& \no																						& \no																													& \si																																& \no																							& \no																													& \no										& \no\\ \hline
Vista frammentata								& \si																																			& \si																													& \si					& \si																																																										& \si																										& \si																						& \si																													& \no																																& \no																							& \no																													& \no										& \no\\ \hline
Nested slides									& \si																																			& \si																													& \no					& \no																																																										& \si																										& \no																						& \no																													& \no																																& \no																							& \no																													& \no										& \no\\ \hline
Stampa										& \si																																			& \no																													& \si					& \no																																																										& \no																										& \no																						& \no																													& \no																																& \no																							& \no																													& \no										& \no\\ \hline
Esportazione in PDF\ped{G}								& \no																																			& \si																													& \si					& \no																																																										& \no																										& \no																						& \no																													& \no																																& \no																							& \no																													& \no										& \no\\ \hline
Scelta percorso									& \si																																			& \no																													& \no					& \no																																																										& \no																										& \no																						& \no																													& \no																																& \no																							& \no																													& \no										& \no\\ \hline
Controllo remoto (per presentatore)						& \no																																			& \si																													& \no					& \si																																																										& \no																										& \si																						& \si																													& \no																																& \no																							& \no																													& \no										& \no\\ \hline
Note (per presentatore)								& \si																																			& \si																													& \no					& \si																																																										& \no																										& \si																						& \si																													& \no																																& \no																							& \no																													& \no										& \no\\ \hline
Zoom (per presentatore)								& \si																																			& \no																													& \no					& \no																																																										& \no																										& \no																						& \no																													& \no																																& \no																							& \no																													& \no										& \no\\ \hline
\multicolumn{1}{|>{\centering}p{1.8cm}|}{\cellcolor{gray!30}Tecnologia}		& HTML5\ped{G}, CSS3 e jQuery																																	& JavaScript\ped{G}																												& HTML5\ped{G}, CSS3 e Vanilla JS		& HTML5\ped{G}, Compass, Flexbox, CSS3 e RequireJS																																																					& jQuery e Modernizr																								& jQuery mobile																					& HTML5\ped{G} e CSS3 																												& HTML5\ped{G}, CSS3 e JavaScript\ped{G}																													& HTML5\ped{G}, CSS3 e jQuery																					& HTML5\ped{G}, CSS3 e OOCSS																											& jQuery									& JavaScript\ped{G} e Ruby\\ \hline
\multicolumn{1}{|>{\centering}p{1.8cm}|}{\cellcolor{gray!30}Browser\ped{G} supportati}	& IE TP, Firefox\ped{G} 10+, Firefox\ped{G} per Android\ped{G} 33+, Chrome\ped{G} 12+, Chrome\ped{G} per Android\ped{G} 39+, Safari 5.1+, iOS\ped{G} Safari 6.1+, Opera 15+, Opera Mobile 24+, Android\ped{G} Browser\ped{G} 4+, Blackberry Browser\ped{G} 7+ e UC Browser\ped{G} for Android\ped{G} 9.9+	& IE 10+ (anche mobile), Firefox\ped{G} 4+, Firefox\ped{G} per Android\ped{G} 33+, Chrome\ped{G} 4+, Chrome\ped{G} per Android\ped{G} 39+, Safari 3.1+, ioS\ped{G} Safari 3.2+, Opera 10.5+, Opera Mobile 10+, Android\ped{G} Browser\ped{G} 2.1+, Blackberry Browser\ped{G} 7+ e UC Browser\ped{G} per Android\ped{G} 9.9+	& Chrome\ped{G}, IE, Firefox\ped{G}, Opera e Safari	& Firefox\ped{G} 28+, Firefox\ped{G} per Android\ped{G} 33+, Chrome\ped{G} 21+, Chrome\ped{G} per Android\ped{G} 39+, Safari 6.1+, iOS\ped{G} Safari 7.1+, Opera 12.1+ anche Mobile, Android\ped{G} Browser\ped{G} 4.4+, Blackberry Browser\ped{G} 10+	& IE 10+ anche Mobile, Firefox\ped{G} 4+, Firefox\ped{G} per Android\ped{G} 33+, Chrome\ped{G} 4+, Chrome\ped{G} per Android\ped{G} 39+, Safari 5.1+, iOS\ped{G} Safari 6.1+, Opera 12.1+, Opera Mobile, Android\ped{G} Browser\ped{G} 2.3 e 4.0+, Blackberry Browser\ped{G} 7+	& IE 7+, Firefox\ped{G} 4+, Firefox\ped{G} Mobile 10+, Safari 5+, iOS\ped{G} Safari 6.1+, Chrome\ped{G} 11+, Chrome\ped{G} per Android\ped{G}, Opera 12.1+, Opera Mobile 11.5+, Android\ped{G} Browser\ped{G} 2.3 e 4.0+ e UC Browser\ped{G}	& IE 10+ anche Mobile, Firefox\ped{G} 4+, Firefox\ped{G} per Android\ped{G} 33+, Chrome\ped{G} 4+, Chrome\ped{G} per Android\ped{G} 39+, Safari 3.1+, iOS\ped{G} Safari 3.2+, Opera 10.5+, Opera Mobile 11.5+, Android\ped{G} Browser\ped{G} 2.1+, Blackberry Browser\ped{G} 7+ e UC Browser\ped{G} per Android\ped{G} 9.9+	& IE TP, Firefox\ped{G} 10+, Firefox\ped{G} per Android\ped{G} 33+, Chrome\ped{G} 12+, Chrome\ped{G} per Android\ped{G} 39+, Safari 4+, iOS\ped{G} Safari 3.2+, Opera 15+, Opera Mobile 24+, Android\ped{G} Browser\ped{G} 3+, Blackberry Browser\ped{G} 7+ e UC Browser\ped{G} for Android\ped{G} 9.9+	& IE 10+ anche Mobile, Firefox\ped{G} 4+, Firefox\ped{G} per Android\ped{G} 33+, Chrome\ped{G} 4+, Chrome\ped{G} per Android\ped{G} 39+, Safari 5.1+, iOS\ped{G} Safari 6.1+, Opera 12.1+, Opera Mobile, Android\ped{G} Browser\ped{G} 2.3 e 4.0+	& IE 10+ anche mobile, Firefox\ped{G} 4+, Firefox\ped{G} per Android\ped{G} 33+, Chrome\ped{G} 4+, Chrome\ped{G} per Android\ped{G} 39+, Safari 3.1+, iOS\ped{G} Safari 3.2+, Opera 10.5+, Opera Mobile 10+, Android\ped{G} Browser\ped{G} 2.1+, Blackberry Browser\ped{G} 7+ e UC Browser\ped{G} per Android\ped{G} 9.9+	& IE6+, Safari 5.1+, iOS\ped{G} Safari 6.1+, Opera 12.1+, Android\ped{G} Browser\ped{G} 2.3 e 4.0+	& IE 10+ anche Mobile, Firefox\ped{G} 5+, Firefox\ped{G} per Android\ped{G} 33+, Chrome\ped{G} 4+, Chrome\ped{G} per Android\ped{G} 39+, Safari 4+, iOS\ped{G} Safari 3.2+, Opera 12+, Opera Mobile 12.1+, Android\ped{G} Browser\ped{G} 4+, Blackberry Browser\ped{G} 7+ e UC Browser\ped{G} per Android\ped{G} 9.9+\\ \hline
\caption{Tabella di comparazione delle principali funzionalità offerte dai framework\ped{G} analizzati.}
\label{tabella:confrontoFramework}
\end{longtable}
\end{landscape}



\subsection{Criticità}
\begin{itemize}
\item \textbf{Individuazione dei requisiti:} nel capitolato sono forniti molti spunti riguardo le funzionalità opzionali che l'applicazione può avere e viene lasciata la piena scelta al gruppo di che funzionalità decidere di implementare. Per questi motivi l'identificazione dei requisiti e la scelta di quali soddisfare richiederà una grande quantità di tempo;
\item \textbf{Scelta del framework\ped{G}:} il capitolato non vincola l'utilizzo di un determinato framework\ped{G}, ma propone molte alternative, tutte queste alternative potrebbero portare i \rPs a scegliere un framework\ped{G} non adatto.
\end{itemize}


\subsection{Valutazione finale}
Il gruppo alla fine ha scelto questo capitolato perché, nonostante le criticità, presenta varie caratteristiche che sono state valutate positivamente:
\begin{itemize}
 \item Interesse del gruppo nei confronti del mondo mobile e delle tecnologie web\ped{G};
 \item Creazione di un prodotto utile a tutti;
 \item Acquisizione di esperienza e conoscenze tecniche utili e spendibili nel mondo del lavoro;
 \item Conoscenza del dominio all'interno del gruppo, derivata da esperienze lavorative e da altri corsi universitari;
% Dobbiamo usare JS, questo punto non vale \item Non viene imposto un linguaggio di programmazione ;
\item Software rilasciato sotto licenza open source.
\end{itemize}
