\section{Capitolati scartati}

\subsection{C1 - BDSMApp}

\subsubsection{Descrizione}
L'obiettivo del progetto è la realizzazione di una infrastruttura web\ped{G} che permetta, all'utente con le autorizzazioni necessarie, di interrogare big data\ped{G} dai social Facebook,Twitter e Instagram.
Tale infrastruttura dev'essere scalabile e deve sfruttare al meglio le potenzialità del cloud\ped{G}.
L'applicazione dev'essere formata da una parte che offra la consultazione e l'interrogazione tramite interfaccia web\ped{G} per gli utenti, e un'altra i servizi REST\ped{G} interrogabili. 
Viene proposto l'utilizzo dello stack\ped{G} tecnologico di Google Cloud Platform, il quale, come linguaggi di programmazione, prevede Java, PHP\ped{G} e Python. 
Viene lasciata piena libertà per l'implementazione dell'interfaccia, anche se viene consigliato l'uso di HTML5\ped{G}, CSS3, jQuery o un framework\ped{G} responsive come Twitter Bootstrap, e dei servizi Web\ped{G}, anche se viene consigliato l'uso di Google Endpoints.
\subsubsection{Dominio tecnologico}
\begin{itemize}
\item \textbf{Java e/o PHP\ped{G} e/o Python:} linguaggi di programmazione consigliati perché disponibili in Google Cloud Platform;
\item \textbf{HTML5\ped{G}, CSS3, jQuery o Twitter Bootstrap:} consigliati per l'implementazione dell'interfaccia;
\item \textbf{Google App Engine:} una Platform as a Service ideale per applicazioni web\ped{G} scalabili, riesce a scalare automaticamente al crescere 
 delle risorse richieste e gestisce automaticamente il carico sui server;
\item \textbf{Google Compute Engine:} una Infrastructure as a Service che abilita l'utente al lancio di macchine virtuali (VMS) on demand;
\item \textbf{Google Cloud Storage:} un servizio scalabile per lo storage\ped{G} di file online;
\item \textbf{Google Cloud Datastore:} database NoSQL ad alte prestazioni;
\item \textbf{Google Cloud SQL:} database MySQL\ped{G};
\item \textbf{Google BigQuery:} tool per l'analisi dei dati che utilizza query SQL-like per processare big data\ped{G} in pochi secondi;
\item \textbf{Google Endpoints:} strumento consigliato per creare web\ped{G} services in App Engine che possono essere utilizzati con iOS\ped{G}, Android\ped{G} e client Javascript\ped{G}.
\end{itemize}

\subsubsection{Criticità}
\begin{itemize}
\item Il gruppo ha la piena libertà nella selezione dei dati da recuperare dai social network e nella scelta di come sviluppare l'architettura;
\item Viene vincolato l'utilizzo dei servizi offerti da Google.
\end{itemize}

\subsubsection{Valutazione}
Malgrado l'interesse verso un settore altamente tecnologico come quello proposto da Zing S.r.l., il gruppo \gruppo ritiene di non avere le 
conoscenze necessarie per poter affrontare in modo adeguato le numerose scelte lasciate a sua discrezione.

 
\subsection{C2 - GUS}

\subsubsection{Descrizione}
Tale capitolato riguarda il settore del controllo della qualità nella produzione industriale del vetro.
Il progetto richiede la costruzione di un sistema software, dedicato al settore del vetro, con le caratteristiche seguenti:

\begin{itemize}
\item \textbf{Efficacia:} deve poter gestire il 100\% delle casistiche, prevedendo qualsiasi tipo di non conformità;
\item \textbf{Facilità di utilizzo:} l'utilizzatore finale dev'essere in grado di gestire e configurare in modo facile e veloce l'intera interfaccia;
\item \textbf{Gestione varie tipologie di "recipes" (ricette) per il controllo di non conformità:} dev'essere in grado di analizzare l'intera immagine e classificarne i difetti per "difettosità" rispetto a dimensione, forma e intensità;
\item \textbf{Segnalazione:} di errori/difetti e di produzione di report;
\item \textbf{Raccolta di dati e analisi statistica:} intervenire sui processi produttivi al fine di identificare le cause della non conformità e prendere le 
iniziative di correzione degli stessi;
\item \textbf{Fruibilità via web\ped{G}:} per controllare la produzione da remoto e vederne lo stato in tempo reale, ovunque ci si trovi.
\end{itemize}

\subsubsection{Dominio tecnologico}

\begin{itemize}
\item \textbf{C++ e IDE\ped{G} di sviluppo QtEditor:} per lo sviluppo della versione stand-alone, devono essere usate librerie\ped{G} standard, ma vengono accettate anche eventuali nuove librerie\ped{G} sviluppate ad-hoc;
\item \textbf{MySQL\ped{G} o PostgreSQL:} database relazionale;
\item \textbf{PHP\ped{G} e Javascript\ped{G}:} per sviluppare l'interfaccia web\ped{G};
\item \textbf{AngularJS e Bootstrap:} framework\ped{G} consigliati per la realizzazione di una interfaccia grafica, con design responsivo\ped{G} per adattarsi
automaticamente a tutti i dispositivi mobili (smartphone e tablet).
\end{itemize}

\subsubsection{Criticità}

\begin{itemize}
\item Scansione e elaborazione di immagini molto grandi, circa 500 MB, in pochi secondi;
\item Costruzione di un algoritmo complesso per il riconoscimento del contorno e delle imperfezioni.
\end{itemize}

\subsubsection{Valutazione}
Nonostante il dominio tecnologico sia quasi totalmente noto ai membri componenti del gruppo, si è scelto di scartare questo capitolato con le seguenti motivazioni:
\begin{itemize}
 \item Scarso interesse da parte dei componenti del gruppo verso le tecnologie richieste;
 \item Difficile previsione del tempo necessario per la realizzazione di un algoritmo in grado di soddisfare i vincoli imposti dalle elevate dimensioni delle immagini da scansionare e dalla scarsità del tempo a disposizione per farlo.
\end{itemize}

\subsection{C3 - Nor(r)is}
\subsubsection{Descrizione}
Lo scopo del progetto è di produrre un framework\ped{G} per lo stack\ped{G} tecnologico formato da Node.js\ped{G}, Express.js e Socket.io in grado di generare grafici 
i cui dati sono letti da sorgenti arbitrarie. Il fruitore finale dei grafici è l'esperto di dominio.\\
Lo scopo del progetto è quello di realizzare un framework\ped{G} che permetta di raccogliere dati provenienti da sorgenti arbitrarie e visualizzare 
tali dati sotto forma di grafici, in modo semplice e veloce. La veste grafica e i dati di ciascun grafico devono essere configurabili 
programmaticamente usando le API\ped{G} fornite da Norris.
Il framework\ped{G} deve mettere a disposizione funzioni di aggiornamento dei grafici lato server tramite tecnologia websocket\ped{G}.

\subsubsection{Dominio tecnologico}

\begin{itemize}
\item \textbf{Node.js\ped{G}:} Sistema run time cross platform per applicazioni lato server e applicazioni di rete. Le applicazioni scritte per Node.js\ped{G} sono progettate per massimizzare l'efficienza di esecuzione, usando un sistema di I/O non bloccante e eventi asincroni. Node.js\ped{G} è ampiamente usato per applicazioni real-time\ped{G} grazie alla sua natura asincrona. Node.js\ped{G} contiene un modulo nativo asincrono per fare I/O su file, sockets e HTTP, grazie a questo modulo, Node.js\ped{G} può essere utilizzato come web\ped{G} server senza ricorrere a software quali Apache HTTP Server o Microsoft IIS;
\item \textbf{Express:} framework\ped{G} per applicazioni web\ped{G} Node.js\ped{G}, che fornisce un insieme di funzionalità per applicazioni web\ped{G} e mobile;
\item \textbf{Socket.io:} libreria\ped{G} Javascript\ped{G} per applicazioni web\ped{G} real-time\ped{G}, che permette di creare comunicazioni bidirezionali tra il web\ped{G} client e il server. \`E costituita da due parti: una libreria\ped{G} lato client che esegue nel browser\ped{G} e una lato server per Node.js\ped{G}. Entrambe le componenti hanno API\ped{G} quasi identiche, similmente a Node.js\ped{G}, la libreria\ped{G} è event-driven\ped{G}. Esistono implementazioni sia per Android\ped{G} che per iOS\ped{G}. Socket.io usa principalmente il protocollo WebSocket\ped{G}, ma se necessario, può usare altri metodi di comunicazione bidirezionali quali: Adobe Flash sockets, JSON polling, ecc. Sebbene Socket.io può essere utilizzato come un semplice wrapper per websocket\ped{G}, essa fornisce molte altre features tra le quali: broadcasting su socket multipli, salvare dati associati a ciascun client e I/O asincrono. La libreria\ped{G} può essere installata utilizzando NPM, il package manager di Node.js\ped{G};
\item \textbf{AngularJs:} framework\ped{G} Javascript\ped{G} per applicazioni web\ped{G}, che supporta il design pattern\ped{G} MVC\ped{G}.

\end{itemize}

\subsubsection{Criticità}
\begin{itemize}
\item \textbf{Costruzione di un framework\ped{G}:} viene richiesto al gruppo di creare un framework\ped{G} per la traduzione di dati provenienti da varie sorgenti in un formato adatto ad essere disegnato su una pagina web.
\end{itemize}

\subsubsection{Valutazione} \label{valutazioneC3}
Nonostante l'interesse per questo capitolato fosse alla pari di quello verso il capitolato C4, alla fine questo capitolato è stato scartato perché:
\begin{itemize}
\item Nessun componente del gruppo aveva esperienza riguardo la creazione di un framework\ped{G};
\item Il gruppo preferiva sviluppare un'applicazione rivolta a diverse categorie d'utenza, che non si limitino ai soli sviluppatori.
\end{itemize}

\subsection{C5 - sHike}

\subsubsection{Descrizione}

Lo scopo del capitolato consiste nella realizzazione di una piattaforma\ped{G} di supporto agli escursionisti, in grado di monitorare il loro stato fisico e la loro posizione e che renda la loro esperienza più divertente e sicura.
Il progetto si propone di sviluppare un'applicazione basata su piattaforma cloud\ped{G} con i seguenti obiettivi:

\begin{itemize}
 \item Fornire agli utenti informazioni sul percorso in montagna, lunghezza e tempo di viaggio, informazioni storiche e/o naturali, di meteo, sull'apertura di rifugi e su come accedere a servizi esistenti;
 \item Collezionare dati aggregati sul comportamento dell'escursionista per lo studio dell'attività fisica nelle montagne come un modo per promuovere la salute.
\end{itemize}

Il progetto ha lo scopo sia di fornire all'utente informazioni aggiornate relative alle attività in montagna sia di collezionare dati per lo studio del comportamento degli escursionisti.
Un obiettivo del progetto è incrementare il numero delle persone che possono camminare in montagna in sicurezza, come normale attività di fitness, con la massima riduzione dei rischi di sicurezza per assenza di informazioni locali.

\subsubsection{Dominio tecnologico}
\begin{itemize}
\item \textbf{Wearable technology:} il prodotto finale sarà un'applicazione per uno smartwatch;
\item \textbf{Android 4.4.2 e SDK:} è il più popolare sistema operativo\ped{G} per dispositivi mobili, è basato sul kernel\ped{G} Linux\ped{G} ed correntemente sviluppato da Google. Android\ped{G} è progettato principalmente per dispositivi mobili touchscreen, ma anche per televisioni, auto e orologi da polso. WearIT usa Android\ped{G} 4.4.2 una release della famiglia di "KitKat" Android\ped{G} 4.4. WearIT Android 4.4.2 viene rilasciato con un insieme di estensioni sviluppate appositamente per smartwatch WearIT – tali estensioni sono disponibili con un insieme documentato di API\ped{G} (WearIT API). Lo sviluppo di App per lo smartwatch WearIT è molto semplice grazie a l'SDK\ped{G} di Android\ped{G};
\item \textbf{Extension WearIT API:} insieme documentato di API\ped{G} appositamente sviluppate per lo smartwatch WearIT;
\item \textbf{JSON Schema:} formato dati intuitivo, chiaro e leggibile da umani e macchine, per aiutare la realizzazione di strutture dati complesse e validare dati JSON scambiati tra colleghi nella rete. 
JSON, o Javascript\ped{G} Object Notation, è un formato standard aperto\ped{G} che usa testo leggibile dagli umani per trasmettere oggetti di dati consistenti di coppie attributo-valore. Viene usato principalmente per trasmettere dati tra un server e un'applicazione web\ped{G}, come alternativa ad XML\ped{G}. 
Nonostante sia originalmente derivato da linguaggi di scripting Javascript\ped{G}, JSON è un linguaggio indipendente dal formato dati. Codice per parsing e generando dati JSON è disponibile alla lettura in un'ampia varietà di linguaggi di programmazione. WearIT usa JSON per lo scambio di dati tra dispositivi WearIT e il server WearIT Cloud;
\item \textbf{The Spring Framework\ped{G}:} framework\ped{G} per applicazioni open source contenente inversione di controllo per la piattaforma\ped{G} Java. La caratteristica centrale di questo framework\ped{G} è che può essere usata da ogni applicazione Java, ma ci sono estensioni per la creazione di applicazioni web\ped{G} nella vetta della piattaforma\ped{G} Java EE. WearIT Cloud Server usa Java e Spring come framework\ped{G} per la maggior parte dello sviluppo e per la distribuzione;
\item \textbf{WearIT Cloud API:} insieme di specifiche e moduli sw che hanno a che fare con moduli cloud\ped{G} specifici di WearIT che funzionano su WearIT Cloud Servers. L'API\ped{G} permette agli sviluppatori di interagire cone le funzioni WearIT Cloud, abilitando la creazione di applicazioni web\ped{G} WearIT.
\end{itemize}
 
\subsubsection{Criticità}
\begin{itemize}
\item \textbf{Realizzazione GUI\ped{G}:} uno degli aspetti principali del capitolato è la realizzazione di una GUI\ped{G} di semplice utilizzo e ricca di informazione. La progettazione di tale GUI\ped{G} è un attività complessa che richiede un'esperienza non ancora acquisita dal gruppo;
\item \textbf{API\ped{G} proprietarie:} per la realizzazione del progetto è necessario l'uso di API\ped{G} proprietarie realizzate dal produttore del dispositivo e difficilmente la conoscenza di queste API\ped{G} tornerà utile in futuro.
\end{itemize}
 
\subsubsection{Valutazione}
Buon interesse da parte del gruppo rispetto all'area di studio e al settore applicativo, ma l'obbligo d'uso di API\ped{G} proprietarie del proponente\ped{G} è stato ritenuto penalizzante dal punto di vista formativo.
