\newglossaryentry{browser}
{
name={Browser},
description={Programma che consente di usufruire dei servizi di connettività in Internet, o di una rete di computer, e di navigare sul World Wide Web.},
first={browser web},
firstplural={web browser}
}


\newglossaryentry{framework}
{
name={Framework},
description={Architettura logica di supporto (spesso un'implementazione logica di un particolare design pattern) su cui un software può essere progettato e realizzato facilitandone lo sviluppo da parte del programmatore.}
}

\newglossaryentry{frontend}
{
name={Front-end},
description={Parte di un sistema software che gestisce l'interazione con l'utente o con sistemi esterni che producono dati in ingresso.},
first={front-end},
text={front end}
}

\newglossaryentry{hardware}
{
name={Hardware},
description={Parte fisica di un computer, ovvero tutte quelle parti elettroniche, elettriche, meccaniche, magnetiche, ottiche che ne consentono il funzionamento.},
text={hw}
}

\newglossaryentry{mongodb}
{
name={MongoDB},
description={Database non relazionale orientato ai documenti di tipo NoSQL. MongoDB si allontana dalla struttura tradizionale basata su tabelle dei database relazionali utilizzando documenti in un formato ispirato allo stile JSON con schema dinamico (denominato BSON).
}
}

\newglossaryentry{mongoose}
{
name={Mongoose},
description={Libreria Javascript per Node.js per l’Object Data Mapping (ODM) che consente di definire schemi con i quali creare e modificare documenti nei database MongoDB. Mongoose consente di trattare gli schemi realizzati come classi e quindi di sfruttare conversioni di tipo, metodi di istanza e metodi statici.}
}

\newglossaryentry{nodejs}
{
name={Node.js},
description={Framework event-driven per il motore JavaScript V8, su piattaforme UNIX like, relativo all'utilizzo server-side di Javascript.}
}

\newglossaryentry{server}
{
name={Server},
description={Componente o sottosistema informatico di elaborazione che fornisce, a livello logico e a livello fisico, un qualunque tipo di servizio ad altre componenti (client) che ne fanno richiesta attraverso una rete di computer, all'interno di un sistema informatico o direttamente in locale su un computer.}
}

\newglossaryentry{sistemaoperativo}
{
name={Sistema Operativo},
description={Insieme di componenti software, che consente l'utilizzo di varie apparecchiature informatiche (ad esempio di un computer) da parte di un utente.}
}