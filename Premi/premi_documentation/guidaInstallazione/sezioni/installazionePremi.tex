\section{Installazione di Premi}\label{installazione}
Una volta accertato che tutti i requisiti sono soddisfatti, \`{e} possibile procedere con l'installazione di premi.
\subsection{Ottenere il Prodotto}
Il prodotto è disponibile nel CD allegato, in formato di archivio compresso \texttt{Premi.zip}.
\subsection{Installazione e avvio di Premi}
Per installare l'applicativo e quindi il \gloxy{server} con Node.js, è necessario innanzitutto scompattare l'archivio \texttt{Premi.zip}, in qualsiasi posizione del file system. L'applicativo si trova nella cartella \texttt{Premi}. \\
Aprire una finestra del terminale e posizionarsi all'interno della cartella dell'applicazione, con il comando:
\begin{center}
\texttt{\$ cd path/to/application/Premi}
\end{center}
Successivamente sarà necessario installare le dipendenze di \Premi, mediante il comando:
\begin{center}
\texttt{\$ npm install}
\end{center}
che utilizza automaticamente il file di configurazione \texttt{package.json}.\\
Al termine dell'installazione, sempre all'interno della cartella dell'applicazione, sarà possibile avviare il \gloxy{server} con il comando:
\begin{center}
\texttt{\$ grunt}
\end{center}
In caso di corretta esecuzione, verrà mostrato nel terminale il messaggio \texttt{\textasciitilde Premi\textasciitilde \ application started on port 3000}. Sarà possibile utilizzare \Premi tramite \gloxy{browser web} all'indirizzo \url{http://localhost:3000}.
Per terminare l'applicazione è sufficiente chiudere il terminale o digitare nello stesso la sequenza di tasti \texttt{ctrl+C}.
\subsection{Installare Premi su un server con IP pubblico}
Se si desidera adibire un \gloxy{server} con indirizzo IP pubblico all'esecuzione di \Premi è necessario installare il prodotto come illustrato nella sezione precedente. In seguito è necessario modificare il file \texttt{serverUrl.service.js} situato nella cartella \texttt{Premi/public/modules/premi/app/services}. Dovrà essere modificata la riga di codice che configura l'indirizzo del \gloxy{server} da:
\begin{center}
\texttt{.constant('SERVER$\_$URL', 'http://localhost:3000');}
\end{center}
a:
\begin{center}
\texttt{.constant('SERVER$\_$URL', 'http://your-server-ip:3000');}
\end{center}
\Premi potrà successivamente essere avviata come spiegato in precedenza.
