\documentclass[12pt,a4paper]{article} % Specify the font size (10pt, 11pt and 12pt) and paper size (letterpaper, a4paper, etc)
\usepackage[italian]{babel}
\usepackage[utf8]{inputenc} % permette l'inserimento di caratteri accentati da tastiera nel documento sorgente.
\usepackage[T1]{fontenc} % specifica la codifica dei font da usare nel documento stampato.
\usepackage{lscape}
\usepackage{times} % per caricare un font scalabile
\usepackage{indentfirst} % rientra il primo capoverso di ogni unità di sezionamento.
\usepackage{xspace}
\usepackage{xstring}
\usepackage{graphicx} % permette l'inserimento di immagini
\usepackage{multirow}
\usepackage{microtype} % migliora il riempimento delle righe
\usepackage{hyperref} % per gestione url
\hypersetup{
    colorlinks=true,       % false: boxed links; true: colored links
    linkcolor=black,          % color of internal links (change box color with linkbordercolor)
    citecolor=green,        % color of links to bibliography
    filecolor=magenta,      % color of file links
    urlcolor=blue           % color of external links
}
\usepackage{url} % per le url in monospace
\usepackage{eurosym} % simbolo euro
\usepackage{lastpage} % permette di sapere l'ultima pagina
\usepackage{fancyhdr} % gestione personalizzata header e footer
\usepackage[a4paper,portrait,top=3.5cm,bottom=3.5cm,left=3cm,right=3cm,bindingoffset=5mm]{geometry} % imposta i margini di pagina nelle classi standard.
\usepackage{hyperref} % abilita i riferimenti ipertestuali.
\usepackage{caption} %per le immagini
\usepackage{subcaption} %per le immagini
\usepackage{placeins} %per i floatbarrier
\usepackage{float} %per il posizionamento delle figure
\usepackage{verbatim} %per i commenti multiriga
\usepackage[table]{xcolor}
\usepackage{longtable} % per le tabelle multipagina
\usepackage{diagbox}
\usepackage{hhline}
\usepackage{array} % per il testo nelle tabelle
\usepackage{multirow}
\usepackage{dirtree}
\usepackage{placeins} % \FloatBarrier per fare il flush delle immagini
\usepackage{tabularx} 

\usepackage[titletoc,title]{appendix}
%membri

%\usepackage{gfsdidot} % Use the GFS Didot font: http://www.tug.dk/FontCatalogue/gfsdidot/
%\usepackage[T1]{fontenc} % Required for accented characters

% Create a new command for the horizontal rule in the document which allows thickness specification
\makeatletter
\def\vhrulefill#1{\leavevmode\leaders\hrule\@height#1\hfill \kern\z@}
\makeatother

%----------------------------------------------------------------------------------------
%	DOCUMENT MARGINS
%----------------------------------------------------------------------------------------

\textwidth 6.75in
\textheight 9.25in
\oddsidemargin -.25in
\evensidemargin -.25in
\topmargin 0in
\parindent 0.4in
\input{../template/comandi.tex}
\begin{document}

\begin{center}
\includegraphics[scale=0.6]{../template/icone/logo.pdf}
\end{center}
\hspace{\fill}\parbox[t]{8cm}{
\noindent
Alla cortese attenzione dei Committenti:\\
\committente \\
\committenteAlt \\
Università degli Studi di Padova \\
Dipartimento di Matematica \\
Via Trieste 63 35121, Padova \\
17 Giugno 2015
}
\\
Responsabile di Progetto\\
\gruppo \\
\groupmail \\
\\
\\
Oggetto: \textbf{Consegna documenti per la Revisione di Accettazione} \\
%\vspace{3em}
\\
\\
\\
\noindent Egregio Prof. Vardanega Tullio,\\
\\
con la presente, il gruppo \gruppo intende comunicarLe ufficialmente la partecipazione alla \RA per il progetto:\\
\begin{center}
\textbf{Premi: Software di presentazione \textit{better than Prezi}}
\end{center}
proposto dall'azienda \proponente \\
La proposta è corredata dai seguenti documenti, allegati alla presente lettera:
\begin{itemize}
\item \normeDiProgetto \texttt{(Interni/normeDiProgetto\_v4.0.0.pdf)} ;
\item \analisiDeiRequisiti \texttt{(Esterni/analisiDeiRequisiti\_v3.0.0.pdf)};
\item \definizioneDiProdotto \texttt{(Esterni/definizioneDiProdotto\_v3.0.0.pdf)};
\item \manualeUtente \texttt{(Esterni/manualeUtente\_v2.0.0.pdf)};
\item \pianoDiProgetto \texttt{(Esterni/pianoDiProgetto\_v4.0.0.pdf)};
\item \pianoDiQualifica \texttt{(Esterni/pianoDiQualifica\_v4.0.0.pdf)};
\item \textit{Manuale d'installazione v1.0.0} \texttt{(Esterni/guidaInstallazione\_v1.0.0.pdf)};
\item \glossario \texttt{(Esterni/glossario\_v3.0.0.pdf)};
\item \eVII \texttt{(Verbali/Esterni/E7\_v1.0.0.pdf)};
\item \iIII \texttt{(Verbali/Interni/I3\_v1.0.0.pdf)}.
\end{itemize}
Viene inoltre consegnato il codice sviluppato, la relativa documentazione e i test che sono stati implementati:
\begin{itemize}
\item Applicativo/sorgente;
\item Applicativo/documentazione;
\item Applicativo/test.
\end{itemize}
Il gruppo \gruppo ha rispettato la scadenza prevista, con un costo totale di \textbf{\euro13.099}, inferiore di \textbf{\euro36} a quello preventivato. \newline
In caso di problemi durante l'installazione dell'applicativo, il gruppo rimane a disposizione per fornire assistenza. Inoltre l'applicazione può essere provata all'indirizzo \url{http://188.166.2.103:3000}.
\newline
\noindent Rimango a Sua disposizione per ogni ulteriore chiarimento. \\
La ringrazio per la Sua attenzione. \\
\\
\\
\\
\\
Cordiali Saluti, \\
Il \rRPt \\
\dm

\end{document}