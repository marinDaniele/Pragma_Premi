\documentclass[12pt,a4paper]{article} % Specify the font size (10pt, 11pt and 12pt) and paper size (letterpaper, a4paper, etc)
\usepackage[italian]{babel}
\usepackage[utf8]{inputenc} % permette l'inserimento di caratteri accentati da tastiera nel documento sorgente.
\usepackage[T1]{fontenc} % specifica la codifica dei font da usare nel documento stampato.
\usepackage{lscape}
\usepackage{times} % per caricare un font scalabile
\usepackage{indentfirst} % rientra il primo capoverso di ogni unità di sezionamento.
\usepackage{xspace}
\usepackage{xstring}
\usepackage{graphicx} % permette l'inserimento di immagini
\usepackage{multirow}
\usepackage{microtype} % migliora il riempimento delle righe
\usepackage{hyperref} % per gestione url
\hypersetup{
    colorlinks=true,       % false: boxed links; true: colored links
    linkcolor=black,          % color of internal links (change box color with linkbordercolor)
    citecolor=green,        % color of links to bibliography
    filecolor=magenta,      % color of file links
    urlcolor=blue           % color of external links
}
\usepackage{url} % per le url in monospace
\usepackage{eurosym} % simbolo euro
\usepackage{lastpage} % permette di sapere l'ultima pagina
\usepackage{fancyhdr} % gestione personalizzata header e footer
\usepackage[a4paper,portrait,top=3.5cm,bottom=3.5cm,left=3cm,right=3cm,bindingoffset=5mm]{geometry} % imposta i margini di pagina nelle classi standard.
\usepackage{hyperref} % abilita i riferimenti ipertestuali.
\usepackage{caption} %per le immagini
\usepackage{subcaption} %per le immagini
\usepackage{placeins} %per i floatbarrier
\usepackage{float} %per il posizionamento delle figure
\usepackage{verbatim} %per i commenti multiriga
\usepackage[table]{xcolor}
\usepackage{longtable} % per le tabelle multipagina
\usepackage{diagbox}
\usepackage{hhline}
\usepackage{array} % per il testo nelle tabelle
\usepackage{multirow}
\usepackage{dirtree}
\usepackage{placeins} % \FloatBarrier per fare il flush delle immagini
\usepackage{tabularx} 

\usepackage[titletoc,title]{appendix}
%membri

%\usepackage{gfsdidot} % Use the GFS Didot font: http://www.tug.dk/FontCatalogue/gfsdidot/
%\usepackage[T1]{fontenc} % Required for accented characters

% Create a new command for the horizontal rule in the document which allows thickness specification
\makeatletter
\def\vhrulefill#1{\leavevmode\leaders\hrule\@height#1\hfill \kern\z@}
\makeatother

%----------------------------------------------------------------------------------------
%	DOCUMENT MARGINS
%----------------------------------------------------------------------------------------

\textwidth 6.75in
\textheight 9.25in
\oddsidemargin -.25in
\evensidemargin -.25in
\topmargin 0in
\parindent 0.4in
\input{../template/comandi.tex}
\begin{document}

\begin{center}
\includegraphics[scale=0.6]{../template/icone/logo.pdf}
\end{center}
\hspace{\fill}\parbox[t]{8cm}{
\noindent
Alla cortese attenzione dei Committenti:\\
\committente \\
\committenteAlt \\
Università degli Studi di Padova \\
Dipartimento di Matematica \\
Via Trieste 63 35121, Padova \\
27 Maggio 2015
}
\\
Responsabile di Progetto\\
\gruppo \\
\groupmail \\
\\
\\
Oggetto: \textbf{Consegna documenti Realizzazione di Prodotto} \\
%\vspace{3em}
\\
\\
\\
\noindent Egregio Prof. Vardanega Tullio,\\
\\
con la presente, il gruppo \gruppo intende comunicarLe ufficialmente la partecipazione alla \RQ per il progetto:\\
\begin{center}
\textbf{Premi: Software di presentazione \textit{better than Prezi}}
\end{center}
proposto dall'azienda \proponente \\
La proposta è corredata dai seguenti documenti, allegati alla presente lettera:
\begin{itemize}
\item \normeDiProgetto \texttt{(Interni/normeDiProgetto\_v3.0.0.pdf)} ;
\item \analisiDeiRequisiti \texttt{(Esterni/analisiDeiRequisiti\_v2.0.0.pdf)};
\item \definizioneDiProdotto \texttt{(Esterni/definizioneDiProdotto\_v2.0.0.pdf)};
\item \manualeUtente \texttt{(Esterni/manualeUtente\_v1.0.0.pdf)};
\item \pianoDiProgetto \texttt{(Esterni/pianoDiProgetto\_v3.0.0.pdf)};
\item \revisioneDiBilancio \texttt{(Esterni/revisioneDiBilancio\_v1.0.0.pdf)};
\item \pianoDiQualifica \texttt{(Esterni/pianoDiQualifica\_v3.0.0.pdf)};
\item \glossario \texttt{(Esterni/glossario\_v2.0.0.pdf)};
\item \eV \texttt{(Verbali/Esterni/E5\_v1.0.0.pdf)};
\item \eVI \texttt{(Verbali/Esterni/E6\_v1.0.0.pdf)}.
\end{itemize}
\noindent I documenti \textit{Glossario} e \textit{Analisi Dei Requisiti} non hanno subito modifiche rispetto la revisione precendente.
\\
Viene inoltre consegnato il codice sviluppato, la documentazione relativa (completa solo per la parte Back-End) e i test che sono stati finora implementati:
\begin{itemize}
\item Premi/Codice;
\item Premi/Documentazione;
\item Premi/Test.
\end{itemize}
Il gruppo \gruppo ha stimato di consegnare il prodotto richiesto entro la fine del secondo semestre dell'anno accademico 2014-2015, con un preventivo di costo pari a \textbf{\euro13.135}. \newline
\\
Nella fase di \fC, durante l'attività di incremento dei contenuti del \PP è emersa la presenza di alcuni errori di contabilità, che hanno portato ad un'incongruenza con la somma proposta al Committente per la realizzazione del progetto.

Nello specifico è stata riscontrata una differenza di \textbf{\euro590} rispetto a quanto pianificato.
Dei quali, \textbf{\euro501} derivano dal ricalcolo dei consuntivi delle fasi di \fPA e \fPD, utilizzando i costi corretti; mentre i restanti \textbf{\euro89} risultano un eccedenza dai preventivi delle fasi di \fC e \fVV, sempre ricalcolati utilizzando i costi corretti. I \textbf{\euro501}, riferendosi a fasi già trascorse, risultano già spesi, mentre gli \textbf{\euro89} sono calcolati su preventivi di fasi non ancora trascorse. Dunque, nel \PP, sono stati riformulati i preventivi delle fasi a venire per rientrare nei \textbf{\euro13.135}.

Per maggiori dettagli relativi all'errore riscontrato e alle operazioni attuate per la revisione di bilancio si rimanda al documento allegato \RB, nel quale sono descritte dettagliatamente le correzioni che hanno permesso la stesura della nuova pianificazione.
\\
\\
\noindent I dettagli qualitativi, di pianificazione e di progettazione del prodotto sono trattati in maniera approfondita nei restanti documenti allegati.
\\
\\
\noindent Rimango a Sua disposizione per ogni ulteriore chiarimento. \\
La ringrazio per la Sua attenzione. \\
\\
\\
\\
\\
Cordiali Saluti, \\
Il \rRPt \\
\gmi

\end{document}