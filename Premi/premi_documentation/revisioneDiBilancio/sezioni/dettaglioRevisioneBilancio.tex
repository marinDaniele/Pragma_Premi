\section{Dettaglio di ripianificazione del bilancio}\label{detRevBilancio}
Nella fase di \fC, durante l'attività di incremento dei contenuti del \PP è emersa la presenza di alcuni errori di contabilità, che hanno portato ad un'incongruenza con la somma proposta al \gloxy{Committente} per la realizzazione del \gloxy{progetto}.
Nello specifico è stata riscontrata una differenza di \textbf{\euro590} rispetto a quanto pianificato.
Dei quali, \textbf{\euro501} derivano dal ricalcolo dei consuntivi delle fasi di \fPA e \fPD, utilizzando i costi corretti; mentre i restanti \textbf{\euro89} risultano un eccedenza dai preventivi delle fasi di \fC e \fVV, sempre ricalcolati utilizzando i costi corretti. I \textbf{\euro501}, riferendosi a fasi già trascorse, risultano già spesi, mentre gli \textbf{\euro89} sono calcolati su preventivi di fasi non ancora trascorse.
Tale errore è stato causato da un'inversione dei costi dei ruoli di \rAP, \rA e \rP, come riportato in tabella:
\begin{table}[h]
\begin{center}
\begin{tabular}{|l|c|c|}
\hline Ruolo & Costo Corretto (\euro) & Costo Errato (\euro) \\
\hline
Responsabile & 30 & 30 \\
Amministratore & \textcolor{Green}{20} & \textcolor{red}{25}\\
Analista & \textcolor{Green}{25} & \textcolor{red}{22}\\
Progettista & \textcolor{Green}{22} &\textcolor{red}{20}\\
Programmatore & 15 & 15\\
Verificatore & 15 & 15\\
\hline
\end{tabular}
\caption{Costi corretti ed errati dei ruoli}
\end{center}
\end{table}
\FloatBarrier
Si è quindi dovuto procedere con un'accurata ispezione di quanto pianificato, sia in termini di costi che di ore, al fine di rientrare nella somma preventivata.
A tale scopo sono state apportate diverse modifiche che si possono riassumere nei seguenti punti:
\begin{itemize}
\item Le ore da \rA, per quanto riguarda la fase di \fC e \fVV, sono state diminuite in maniera sostanziale: si è scelto di far figurare le ore effettivamente impegate da \rA come investimento interno, senza quindi doverle rendicontare al \gloxy{Committente}, per \textit{E6-2}.
\item \`E stata rivista la pianificazione delle ore per quanto riguarda la fase di \fC e \fVV, tenendo anche conto dello stato dei lavori e dell'esperienza maturata durante le precedenti revisioni.
In particolare, per alcuni ruoli sono state fatte le seguenti considerazioni:
\begin{itemize}
\item	\textbf{\rRP}: le ore preventivate per la fase di \fC e \fVV sono state diminuite a fronte degli accorgimenti che saranno riportati di seguito.
Il flusso di lavoro all'interno del \gloxy{team} risulta ottimo, i ritardi sono minimi e tutti gli imprevisti vengono gestiti ottimamente minimizzandone l'impatto. Ma cosa più importante, grazie allo strumento di prevenzione e analisi dei rischi, si è riuscito a prevenire l'occorenza di situazioni anomale e a mitigarne gli effetti ove si verificassero.
L'utilizzo massivo da parte del \gloxy{team} degli strumenti organizzativi ha permesso di agevolare il lavoro svolto dal \rRP, per quanto riguarda l'organizzazione nel dettaglio delle attività e la pianificazione del lavoro.
Nonostante ciò, le sue ore di lavoro non sono state ridotte in modo eccessivo, poichè la risoluzione di eventuali conflitti interni tra i membri del gruppo può richiedere particolare attenzione.
\item \textbf{\rA}: in seguito ai \textit{feedback} positivi ricevuti in sede di correzione, per quanto riguarda il documento di \AR e la conferma da parte del \gloxy{Proponente} dei requisiti stessi, hanno permesso una riduzione delle ore preventivate.
Alcune ore sono state comunque mantenute nel caso si dovesse verificare la necessità di lavoro da parte dell'\rA.
\item \textbf{\rAP}: come descritto in precedenza la maggior parte delle ore di \rAP verranno fatte figurare come investimento interno. Ciò è dato dal fatto che il miglioramento dei processi interni, degli strumenti di lavoro e delle risorse non è da imputare ai costi del \gloxy{Committente}.
\end{itemize}
\end{itemize}
Di seguito verranno mostrate le incongruenze trovate e verranno marcate le differenze tra quanto era stato pianificato in precendeza e la nuova pianificazione per ogni revisione.
\subsection{Progettazione Architetturale}
\begin{table}[h]
\begin{center}
\begin{tabular}{|l|c|c|c|c|}
\hline Ruolo & Ore & Costo Corretto (\euro) & Costo Errato (\euro) & Differenza (\euro) \\
\hline
Responsabile & 3 & 90  & 90 & 0\\
Amministratore & 4 & 80 & 100 & \textcolor{Green}{-20}\\
Analista &  8 & 200 & 176 & \textcolor{red}{+24}\\
Progettista & 125 & 2750 & 2500 & \textcolor{red}{+250}\\
Programmatore & 0 & 0 & 0 & 0\\
Verificatore & 45 & 675 & 675 & 0\\
\hline
\end{tabular}
\caption{Differenze preventivo Progettazione Architetturale}
\end{center}
\end{table}
\FloatBarrier
\begin{table}[h]
\begin{center}
\begin{tabular}{|l|c|c|c|c|}
\hline Ruolo & Ore & Costo Corretto (\euro) & Costo Errato (\euro) & Differenza (\euro)\\
\hline
Responsabile & 3 & 90 & 90 & 0\\
Amministratore & 4 & 80 & 100 & \textcolor{Green}{-20}\\
Analista & 5 & 125 & 110 & \textcolor{red}{+15}\\
Progettista & 125 & 2750 & 2500 & \textcolor{red}{+250}\\
Programmatore & 0 & 0 & 0 & 0\\
Verificatore & 48 & 720 & 720 & 0\\
\hline
\end{tabular}
\caption{Differenze consuntivo Progettazione Architetturale}
\end{center}
\end{table}
\FloatBarrier
\begin{table}[h]
\begin{center}
\begin{tabular}{|l|c|c|c|}
\hline & Costo Corretto (\euro) & Costo Errato (\euro) & Differenza (\euro)\\
\hline
Preventivo & 3795 & 3541& \textcolor{red}{+254}\\
Consuntivo &3765 & 3520& \textcolor{red}{+245}\\
\hline
\end{tabular}
\caption{Incidenza totali budget Progettazione Architetturale}
\end{center}
\end{table}
\FloatBarrier
La tabella dei costi totali per la fase di \fPA mostra un aumento della spesa di \euro245.
\newpage
\subsection{Progettazione di Dettaglio}
\begin{table}[h]
\begin{center}
\begin{tabular}{|l|c|c|c|c|}
\hline Ruolo & Ore & Costo Corretto (\euro) & Costo Errato (\euro) & Differenza (\euro) \\
\hline
Responsabile & 3 & 90  & 90 & 0\\
Amministratore & 4 & 80 & 100 & \textcolor{Green}{-20}\\
Analista &  2 & 50 & 44 & \textcolor{red}{+6}\\
Progettista & 128 & 2816 & 2560 & \textcolor{red}{+256}\\
Programmatore & 0 & 0 & 0 & 0\\
Verificatore & 41 & 615 & 615 & 0\\
\hline
\end{tabular}
\caption{Differenze preventivo Consolidamento Architettura}
\end{center}
\end{table}
\FloatBarrier
\begin{table}[h]
\begin{center}
\begin{tabular}{|l|c|c|c|c|}
\hline Ruolo & Ore & Costo Corretto (\euro) & Costo Errato (\euro) & Differenza (\euro)\\
\hline
Responsabile & 3 & 90 & 90 & 0\\
Amministratore & 2 & 40 & 50 & \textcolor{Green}{-10}\\
Analista & 2 & 50 & 44 & \textcolor{red}{+6}\\
Progettista & 130 & 2860 & 2600 & \textcolor{red}{+260}\\
Programmatore & 0 & 0 & 0 & 0\\
Verificatore & 41 & 615 & 615 & 0\\
\hline
\end{tabular}
\caption{Differenze consuntivo Consolidamento Architettura}
\end{center}
\end{table}
\FloatBarrier
\begin{table}[h]
\begin{center}
\begin{tabular}{|l|c|c|c|}
\hline & Costo Corretto (\euro) & Costo Errato (\euro) & Differenza (\euro)\\
\hline
Preventivo & 3651 & 3409& \textcolor{red}{+242}\\
Consuntivo & 3655 & 3399& \textcolor{red}{+256}\\
\hline
\end{tabular}
\caption{Incidenza totali budget Consolidamento Architettura}
\end{center}
\end{table}
\FloatBarrier
La tabella dei costi totali per la fase di \fPD mostra un aumento della spesa di \euro259.
\newpage
\subsection{Realizzazione di Prodotto}
\begin{table}[h]
\begin{center}
\begin{tabular}{|l|c|c|c|c|}
\hline Ruolo & Ore & Costo Corretto (\euro) & Costo Errato (\euro) & Differenza (\euro) \\
\hline
Responsabile & 8 & 240  & 240 & 0\\
Amministratore & 8 & 160 & 200 & \textcolor{Green}{-40}\\
Analista & 11 & 275 & 242 & \textcolor{red}{+33}\\
Progettista & 40 & 880 & 800 & \textcolor{red}{+80}\\
Programmatore & 116 & 1740 & 1740 & 0\\
Verificatore & 93 & 1395 & 1395 & 0\\
\hline
\end{tabular}
\caption{Differenze preventivo Realizzazione Prodotto}
\end{center}
\end{table}
\FloatBarrier
\begin{table}[h]
\begin{center}
\begin{tabular}{|l|c|c|c|c|}
\hline Ruolo & Ore & Costo (\euro) \\
\hline
Responsabile & 7 & 210 \\
Amministratore & 2 & 40 \\
Analista & 6 & 150 \\
Progettista & 40 & 880\\
Programmatore & 116 & 1740\\
Verificatore & 93 & 1395\\
\hline
\end{tabular}
\caption{Preventivo Realizzazione Prodotto ripianificato}
\end{center}
\end{table}
\FloatBarrier
La ripianificazione della fase di \fC ha portato ad un risparmio di \euro300 considerando la differenza del preventivo aggiornato con i costi esatti e il nuovo preventivo redatto.
Questa somma verrà utilizzata per coprire l'esubero di \euro504 derivante dal ricalcolo dei costi delle fasi precedenti.
\newpage
\subsection{Collaudo Finale}
\begin{table}[h]
\begin{center}
\begin{tabular}{|l|c|c|c|c|}
\hline Ruolo & Ore & Costo Corretto (\euro) & Costo Errato (\euro) & Differenza (\euro) \\
\hline
Responsabile & 7 & 210  & 210 & 0\\
Amministratore & 4 & 80 & 100 & \textcolor{Green}{-20}\\
Analista & 4 & 100 & 88 & \textcolor{red}{+12}\\
Progettista & 12 & 264 & 240 & \textcolor{red}{+24}\\
Programmatore & 16 & 240 & 240 & 0\\
Verificatore & 46 & 690 & 690 & 0\\
\hline
\end{tabular}
\caption{Differenze preventivo Collaudo Finale}
\end{center}
\end{table}
\FloatBarrier
\begin{table}[h]
\begin{center}
\begin{tabular}{|l|c|c|c|c|}
\hline Ruolo & Ore & Costo (\euro) \\
\hline
Responsabile & 4 & 120 \\
Amministratore & 2 & 40 \\
Analista & 2 & 50 \\
Progettista & 10 & 220\\
Programmatore & 12 & 180\\
Verificatore & 46 & 690\\
\hline
\end{tabular}
\caption{Preventivo Realizzazione Prodotto ripianificato}
\end{center}
\end{table}
\FloatBarrier
La ripianificazione della fase di \fVV ha portato ad un risparmio di \euro284 considerando la differenza del preventivo aggiornato con i costi esatti e il nuovo preventivo redatto.
Questa somma verrà utilizzata per coprire l'esubero di \euro504 derivante dal ricalcolo dei costi delle fasi precedenti.
\subsection{Conclusioni}
Considerando i costi effettivamente sostenuti durante le fasi di \fPA e \fPD, i preventivi delle fasi \fC e \fVV sono stati ripianificati facendo in modo di rientrare nel budget precedentemente stabilito di \textbf{\euro13135}.
Di seguito vengono mostrati i contenuti dei costi preventivati e sostenuti in data 2015-05-14.
\begin{table}[h]
\begin{center}
\begin{tabular}{|l|c|c|c|c|}
\hline Fase progetto & Costo (\euro) \\
\hline
Consuntivo Progettazione Architetturale & 3765 \\
Consuntivo Consolidamento Architettura& 3665 \\
Preventivo Realizzazione Prodotto & 4415 \\
Preventivo Collaudo Finale & 1300\\
& \\
\textbf{Totale} & \textbf{13135}\\
\hline
\end{tabular}
\caption{Riepilogo costi aggiornati in data 2015-05-14}
\end{center}
\end{table}
\FloatBarrier
