\newglossaryentry{browser}
{
name={Browser},
description={Programma che consente di usufruire dei servizi di connettività in Internet, o di una rete di computer, e di navigare sul World Wide Web.},
first={browser web},
firstplural={web browser}
}

\newglossaryentry{bug}
{
name={Bug},
description={Errore in un programma software.},
text={baco}
}

\newglossaryentry{chrome}
{
name={Google Chrome},
description={Browser basato su WebKit sviluppato da Google.},
text={chrome}
}



\newglossaryentry{css}
{
name={CSS},
description={Acronimo di \textit{Cascading Style Sheets}, linguaggio usato per definire la formattazione di documenti HTML e XML, ad esempio di siti web e relative pagine web.},
first={Cascading Style Sheets (CSS)}
}

\newglossaryentry{firefox}
{
name={Mozilla Firefox},
description={Web browser open source multipiattaforma prodotto da Mozilla Foundation.},
text={firefox}
}

\newglossaryentry{frame}
{
name={Frame},
description={Nodo della mappa mentale visto come contenitore di elementi grafici quali: testo, immagini e video. Lo stesso frame può essere presente in più percorsi di presentazione e può comparire più volte in uno stesso percorso.}
}


\newglossaryentry{gmail}
{
name={GMail},
description={Servizio gratuito di posta elettronica offerto da Google.}
}

\newglossaryentry{gplus}
{
name={Google+},
description={Rete sociale gratuita creata da Google Inc. nel 2011.},
text={plus}
}

\newglossaryentry{html}
{
name={HTML},
description={Acronimo di \textit{HyperText Markup Language}, linguaggio di markup solitamente usato per la formattazione di documenti ipertestuali disponibili nel World Wide Web sotto forma di pagine web.},
first={HyperText Markup Language (HTML)}
}

\newglossaryentry{htmlcinque}
{
name={HTML5},
description={Linguaggio di markup per la strutturazione delle pagine web, da ottobre 2014 pubblicato come W3C Recommendation. \`{E} stato sviluppato con lo scopo di migliorare il disaccoppiamento fra struttura, definita dal markup, caratteristiche di resa (tipo di carattere, colori, eccetera), definite dalle direttive di stile, e contenuti di una pagina web, definiti dal testo vero e proprio.}
}

\newglossaryentry{javascript}
{
name={JavaScript},
description={Linguaggio di scripting orientato agli oggetti e agli eventi, comunemente utilizzato nella programmazione Web lato client per la creazione, in siti web e applicazioni web, di effetti dinamici interattivi tramite funzioni di script invocate da eventi innescati a loro volta in vari modi dall'utente sulla pagina web in uso (mouse, tastiera ecc...).}
}

\newglossaryentry{mailinglist}
{
name={Mailing List},
description={Servizio/strumento offribile da una rete di computer verso vari utenti e costituito da un sistema organizzato per la partecipazione di più persone ad una discussione o per la distribuzione di informazioni utili agli interessati/iscritti attraverso l'invio di email ad una lista di indirizzi di posta elettronica di utenti iscritti.},
plural={mailing lists}
}

\newglossaryentry{mappam}
{
name={Mappa mentale},
description={Rappresentazione schematizzata di un tema o argomento. Consiste in un albero che ha come radice l’argomento della mappa mentale e come figli i varie idee ad essa correlate. A queste idee possono essere correlate ulteriori idee di secondo livello e così via. Su questa struttura è inoltre possibile definire delle associazioni tra i vari nodi della mappa, aumentandone l’espressività e evidenziando la presenza di legami trasversali tra i vari elementi.}
}

\newglossaryentry{pdf}
{
name={PDF},
description={Acronimo di \textit{Portable Document Format}, formato di file basato su un linguaggio di descrizione di pagina sviluppato da Adobe Systems nel 1993 per rappresentare documenti in modo indipendente dall'hardware e dal software utilizzati per generarli o per visualizzarli.},
first={Portable Document Format (PDF)}
}

\newglossaryentry{pdp}
{
name={Percorso di presentazione},
description={Ordine nel quale vengono presentati alcuni frame della mappa mentale. Un progetto può contenere più percorsi di presentazione. In un progetto esiste sempre il percorso di presentazione di default che contiene tutti i frame ordinati in base all’ordine di creazione.},
plural={percorsi di presentazione}
}

\newglossaryentry{pdv}
{
name={Percorso di visualizzazione},
description={Sinonimo di percorso di presentazione.},
plural={percorsi di visualizzazione}
}

\newglossaryentry{percorso}
{
name={Percorso},
description={\begin{enumerate}
\item Abbreviazione di percorso di presentazione;
\item Posizione logica del file all’interno di un filesystem.
\end{enumerate}},
plural={percorsi}
}

\newglossaryentry{piattaforma}
{
name={Piattaforma},
description={Base software e/o hardware su cui sono sviluppate e/o eseguite applicazioni.}
}

\newglossaryentry{premiproj}
{
name={Progetto},
description={\begin{enumerate}
\item \textit{Progetto Premi}: mappa mentale realizzata con Premi sulla quale è possibile definire dei percorsi di presentazione sfruttando il contenuto dei nodi della mappa mentale. La presentazione potrà essere eseguita in modo lineare, seguendo l'ordine prestabilito, oppure non lineare, visitando liberamente i nodi della mappa. Quando viene creato un progetto, questo contiene il nodo radice della mappa mentale e un percorso di presentazione di default.
\item \textit{Progetto software}: insieme di attività organizzate sotto vincoli di efficacia ed efficienza, che consentono di passare tra i vari stati del ciclo di vita del software
\end{enumerate}
},
plural={progetti}
}

\newglossaryentry{team}
{
name={Team},
description={Insieme di persone che collaborano tra di loro e che hanno lo stesso scopo, in genere lavorativo, scientifico, culturale o sportivo. In questo progetto, questo termine indica i componenti del gruppo \textit{Pragma}.}
}

\newglossaryentry{www}
{
name={World Wide Web},
description={Uno dei principali servizi di Internet che permette di navigare e usufruire di un insieme vastissimo di contenuti (multimediali e non) collegati tra loro e di ulteriori servizi accessibili a tutti o ad una parte selezionata degli utenti di Internet.},
text={web}
}
