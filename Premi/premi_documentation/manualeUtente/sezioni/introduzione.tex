\section{Introduzione} \label{intro}
\subsection{Scopo documento}
Questo documento ha lo scopo di illustrare il funzionamento dell'applicazione \Premi coprendo tutti gli aspetti e funzionalità che offre. All'utente non è richiesta nessuna conoscenza informatica particolare per l'utilizzo dell'applicazione, poiché andrà ad interfacciarsi, tramite \gloxy{web browser}, alle funzionalità offerte da \Premi che verranno fornite con le stesse modalità di un sito \gloxy{web}.
\subsection{Scopo del prodotto}
\scopoProdotto
\subsection{Glossario} %non è il nome del documento.
Al fine di evitare ogni ambiguità di linguaggio e massimizzare la comprensione dei documenti, i termini tecnici, di dominio, gli acronimi e le parole che necessitano di essere chiarite, sono riportate nel glossario interno del documento. Ogni occorrenza dei vocaboli presenti nel glossario è marcata da una ``G'' maiuscola in pedice ed è scritta in corsivo (es: \gloxy{Esempio}).
\subsection{Prerequisiti}
L'utente deve possedere una connessione ad internet, un \gloxy{web browser} (\textit{Chrome >=31, \gloxy{Firefox} >=33}) che supporti \gloxy{CSS3}, \gloxy{HTML5} e \gloxy{JavaScript}, una \gloxy{piattaforma} che fornisca \Premi come servizio raggiungibile tramite un \textit{URL} e le relative credenziali di accesso (ottenibili mediante registrazione).
\subsection{Come accedere al manuale}
All'interno del \textit{header} delle pagine di accesso, registrazione e dashboard è presente un riferimento, identificato da un'icona con un punto di domanda \includegraphics[scale=0.5]{immagini/manualButton.pdf}, che permette di accedere alla documentazione relativa all'applicazione \textbf{\progetto}.\\
Se l'utente si trova nelle pagine di modifica mappa mentale, modifica \gloxy{percorsi} o presentazione, potrà trovare il riferimento al manuale utente all'interno del menù, al quale potrà accedere tramite il pulsante \includegraphics[scale=0.5]{immagini/buttonMenu.pdf}.\\
Il manuale integrato nell'applicazione è contestuale, perciò se l'utente vi accede da una determinata vista otterrà le informazioni relative agli strumenti di quella vista. Per accedere al manuale utente in versione completa è necessario selezionare il link \textit{Manuale completo} presente nella parte bassa di ogni finestra del manuale contestuale.
