\section{Lista degli errori frequenti}\label{erroriFrequenti}
\begin{itemize}
\item Norme stilistiche:
\begin{itemize}
\item Mancato rispetto delle norme relative a elenchi puntati e numerati;
\item Mancato utilizzo delle macro;
\item Mancato rispetto delle norme relative alla redazione del diario;
\item Mancata precisazione della versione nei riferimenti a documenti esterni.
\end{itemize}
\item Linguistica:
\begin{itemize}
\item Mancato rispetto degli accenti: uso dell'accento acuto quando è richiesto quello grave, in particolare ``è'' confusa con ``é'';
\item ``HTML'', ``URL'', ``URI'' sono acronimi, vanno con tutte le lettere maiuscole;
\item Periodi eccessivamente lunghi, che complicano la lettura;
\item Errato uso delle doppie, in particolare raddoppio della lettera z davanti a parole che finiscono in ``-ione'';
\item Mancato rispetto della terza persona singolare del congiuntivo presente;
\item Mancato uso dei pronomi, per evitare la ripetizione di una componente della frase;
\item Non si scrive ``Lo scopo è quello di definire \dots'' ma ``Lo scopo è definire \dots'';
\item Uso errato delle persone dei verbi, in particolare persone singolari usate al posto di quelle plurali, a causa di un'errata individuazione del relativo soggetto all'interno delle frasi;
\item Uso errato dei pronomi relativi, in particolare ``il/di/in/a cui'' usati al posto di ``i/dei/nei/ai quali'' e ``le/delle/nelle/alle quali''.
\end{itemize}
\item \LaTeX:
\begin{itemize}
\item Per inserire caratteri speciali, quali ``\$'' e ``\&'', è necessario inserire il prefisso ``\textbackslash'', ad esempio ``\textbackslash\$'' e ``\textbackslash\&'';
\item La maiuscola della lettera ``è'' si scrive ``\textbackslash\ $\grave{}$ E'' oppure ``\textbackslash\ $\grave{}$ \{E\}''.
\end{itemize}
\end{itemize}
