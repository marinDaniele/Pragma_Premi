\subsubsection{Attività}\label{attivitaValidazione}
\subsubsubsection{Test}\label{testProcessoValidazione}
\subsubsubsubsection{Test di sistema}
Il processo di \textit{validazione} può essere istanziato in modo efficace lato \textit{\gloxy{fornitore}} attraverso i \textit{test di sistema}.
Questo tipo di test si preoccupa di verificare che le funzionalità del prodotto concordate con il \gloxy{committente} siano effettivamente state implementate e che si comportino nel modo atteso. In questo modo risulta possibile verificare il livello di completamento dell'applicativo prima di sottoporre il lavoro svolto all'accettazione da parte del \gloxy{committente}.
I test di sistema possono essere identificati grazie alla seguente sintassi:
\begin{center}
TS[Tipo Requisito][Importanza Requisito][Codice Requisito]
\end{center}
Tipo, importanza e codice si riferiscono al requisito di cui verrà testato il soddisfacimento.
\subsubsubsubsection{Test di validazione}
Il test di validazione coincide con il collaudo del software in presenza del \gloxy{Proponente}, e in caso di esito positivo, questo test determina un grado di maturità del prodotto tale da consentirne il rilascio.
Questo tipo di test fornisce al \gloxy{Proponente} un valido strumento per verificare se tutte le funzionalità concordate con il \gloxy{Fornitore} sotto forma di requisiti sono state effettivamente implementate in modo corretto e se quanto prodotto risulta conforme alle aspettative.
I test di validazione possono essere identificati grazie alla seguente sintassi:
\begin{center}
TV[Tipo Requisito][Importanza Requisito][Codice Requisito]
\end{center}
Tipo, importanza e codice si riferiscono al requisito di cui verrà testato il soddisfacimento.
