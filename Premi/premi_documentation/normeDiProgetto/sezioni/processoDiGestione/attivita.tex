\subsubsection{Attività}
\subsubsubsection{Comunicazioni}
\subsubsubsubsection{Comunicazioni interne}
Per le comunicazioni riguardanti un esigua cerchia di componenti del gruppo è consigliato, ma non vincolato, l'uso di Google \gloxy{Hangouts}. Ad ogni modo i diretti interessati potranno usare il mezzo di comunicazione che riterranno più idoneo alle loro esigenze.
L'utilizzo di telefonate è sconsigliato in quanto troppo invasivo. Resta comunque utilizzabile nel caso in cui sia necessaria una risposta immediata o nel caso in cui i due membri si siano precedentemente accordati.
Se durante questo tipo di discussioni emergono informazioni riguardanti l'avanzamento del \gloxy{progetto}, che interessano tutti i componenti del gruppo, è necessario scrivere un verbale interno da inserire nell'apposita cartella presente in \gloxy{Google Drive}.
\subsubsubsubsubsection{Comunicazioni via mailing list}
L'utilizzo della \gloxy{mailing list} \url{sweteam201415@yahoogroups.com} è riservato per le comunicazioni che riguardano tutti i componenti del gruppo e che devono essere archiviate in modo da potervi accedere efficacemente.
\subsubsubsubsubsection{Comunicazioni via \gloxy{chat}}
Per le comunicazioni che riguardano tutto il gruppo o solo alcuni suoi elementi, si è scelto di utilizzare Google \gloxy{Hangouts}, il quale fornisce sia un servizio di \gloxy{instant messaging}, che un servizio di conferenze e videoconferenze.
La \gloxy{chat} di gruppo è da utilizzare solamente per discutere di argomenti inerenti al \gloxy{progetto}, che richiedano una scelta di gruppo da prendere a breve termine. Poiché recuperare informazioni precise dalla \gloxy{chat} è un operazione troppo costosa e disordinata, al termine di ogni discussione un volontario scriverà un verbale interno, che sarà salvato nella cartella \texttt{Verbali} di \gloxy{Google Drive}.
\subsubsubsubsection{Comunicazioni esterne}
Per le comunicazioni esterne e per le registrazioni ai servizi utili per il gruppo è stata creata una casella di posta elettronica:
\begin{center}
\groupmail
\end{center}
Questo indirizzo dev'essere l'unico canale di comunicazione esistente tra il gruppo di lavoro e l'esterno.
Solo il \rRP può accedere ed inviare mail da questa casella di posta.
Al fine di mantenere informati tutti i membri del gruppo le mail in arrivo verranno automaticamente pubblicate nella \gloxy{mailing list}.
\subsubsubsection{Riunioni}
\subsubsubsubsection{Riunioni interne}
Il \rRP ha il compito di convocare le riunioni che si svolgono tra i soli membri del gruppo. Egli deve stilare l'ordine del giorno e individuare la data di svolgimento dell'incontro, segnalandola al \gloxy{team} tramite \gloxy{mailing list}, con un preavviso di almeno 2 giorni.
Ogni membro del gruppo dovrà dare conferma tempestiva della propria disponibilità a partecipare all'incontro. Potrà eventualmente manifestare l'impossibilità di parteciparvi, fornendo le adeguate motivazioni, entro e non oltre le ore 12 del giorno successivo alla ricezione dell'invito, in modo da consentire al \rRP un'eventuale spostamento di data della riunione.
Valgono le medesime regole anche per gli incontri destinati a un insieme ristretto di componenti del \gloxy{team}.
\\La mail di convocazione delle riunioni interne deve contenere:
\begin{itemize}
\item \textbf{Oggetto}: Convocazione della riunione interna n. xxx (dove xxx rappresenta il numero crescente della riunione);
\item \textbf{Corpo}:
\begin{itemize}
\item Data e ora previste;
\item Luogo di svolgimento della riunione;
\item Ordine del giorno;
\item Durata prevista.
\end{itemize}
\end{itemize}
Alla fine di ogni riunione, un componente del \gloxy{team}, a discrezione del \rRP, avrà il compito di redigerne il verbale.
\subsubsubsubsection{Riunioni esterne}
Gli incontri esterni, con il \gloxy{Proponente} o con il \gloxy{Committente}, vengono concordati dal \rRP che, prima di prendere accordi, dovrà assicurarsi della presenza dei componenti del gruppo interessati all'incontro.
Ogni membro del \gloxy{team} può richiedere al \rRP un incontro esterno, e tale richiesta dovrà essere motivata e approvata prima che possa essere fissato un incontro.
Le informazioni quali \emph{data}, \emph{ora} e \emph{luogo} dovranno essere comunicate al gruppo.
Gli assenti all'incontro potranno essere scelti, con alta probabilità, per redigere il verbale o eseguirne la verifica. Almeno una fase tra \emph{stesura}, \emph{verifica} e \emph{approvazione} del documento dovrà essere compito di membri presenti all'incontro. I verbale relativi a riunioni esterne dovrà sempre essere resi disponibili a tutto il \gloxy{team}, e inviati al \gloxy{Committente} e ad altre eventuali entità esterne partecipanti.
\subsubsubsection{Ticketing}
Per consentire un'agevole assegnazione dei compiti ai vari componenti del \gloxy{team} e un monitoraggio istantaneo del loro stato di avanzamento, il team ha deciso di appoggiarsi ad un sistema di \gloxy{ticketing} (per i dettagli si veda \ref{sistemaTicketing}).
