\subsubsection{Strumenti}
\subsubsubsection{Google Drive}
\textbf{\gloxy{Google Drive}} è il servizio di \gloxy{webstorage} usato dal gruppo per condividere documenti con le seguenti caratteristiche:
\begin{itemize}
\item Non necessitano di controllo di versione;
\item Contengono informazioni utili allo sviluppo del \gloxy{progetto}, ma non fanno parte dei documenti di \gloxy{progetto};
\item Possono essere acceduti mediante il semplice utilizzo del \gloxy{web} \gloxy{browser}.
\end{itemize}
\gloxy{Google Drive} va inteso come strumento di supporto allo sviluppo del \gloxy{progetto}, sia per la documentazione che per il software. Consente, inoltre, ai membri del gruppo di lavorare in modo collaborativo sui documenti.
\subsubsubsection{Google Calendar}
\textbf{\gloxy{Google Calendar}} viene utilizzato come calendario condiviso dal gruppo, al fine di gestire le risorse umane. \`E fondamentale per la segnalazione dei giorni di reperibilità dei membri del \gloxy{team}, per semplificare la scelta delle date delle riunioni, e per memorizzare e notificare le date degli appuntamenti del gruppo.
\subsubsubsection{Git}
Il software di \gloxy{versionamento} scelto è \textbf{\gloxy{Git}}. Il gruppo ha preso in considerazione come altra alternativa \textbf{SVN}, ma ha ritenuto la scelta di \textbf{\gloxy{Git}} più efficace per i seguenti motivi:
\begin{itemize}
\item \textbf{Flessibilità}: essendo un \gloxy{repository} distribuito, \textbf{\gloxy{Git}} consente di lavorare in locale ed avere \textit{commit} e \textit{revert} locali;
\item \textbf{Velocità delle operazioni}: quasi tutte le operazioni sono effettuate sulla copia locale, eliminando problemi legati alla \gloxy{latenza} della rete;
\item \textbf{Esperienza del gruppo}: tutti i componenti del gruppo hanno utilizzato in passato \textbf{\gloxy{Git}}.
\end{itemize}
\subsubsubsection{Bitbucket}
Per lo svolgimento del \gloxy{progetto} sono stati creati due \gloxy{repository} \textit{privati} nello spazio di \gloxy{hosting} offerto da \textbf{\gloxy{Bitbucket}}. Affinché ogni membro del gruppo possa sincronizzarsi con i \gloxy{repository} privati è necessario che ognuno abbia un account \textbf{\gloxy{Bitbucket}}. L'\rAP si occupa di attribuire permessi di lettura e scrittura ad ogni componente del gruppo.
I due \gloxy{repository} creati sono i seguenti:
\begin{itemize}
\item \textbf{pragmadocs.git}: contiene tutti gli elementi necessari alla stesura della documentazione, quindi i sorgenti \LaTeX  ~e gli script utilizzati per la loro corretta compilazione. Il \gloxy{repository} è disponibile agli indirizzi:
\begin{center}
\url{https://nome_utente@bitbucket.org/gmidena/pragmadocs.git} \par
\url{ssh://git@bitbucket.org:gmidena/pragmadocs.git}
\end{center}
\item \textbf{pragmasrc.git}: conterrà i file dell'applicazione. Il \gloxy{repository} è disponibile agli indirizzi:
\begin{center}
\url{https://nome_utente@bitbucket.org/gmidena/pragmasrc.git} \par
\url{ssh://git@bitbucket.org:gmidena/pragmasrc.git}
\end{center}
\end{itemize}
Al termine della stesura di un documento, i \rVs lavoreranno e apporteranno modifiche in parallelo al resto dei componenti del gruppo, utilizzando un opportuno branch creato per la verifica. Al termine della verifica, verrà effettuato il merge delle modifiche.
\\Ad ogni revisione, la versione dei documenti verrà identificata mediante la creazione di un tag.
\\Le suddivisioni del \gloxy{repository} dedicato all'applicazione, \textbf{pragmasrc.git}, verranno trattate durante la fase di \fPA.
\subsubsubsection{Redmine}\label{sistemaTicketing}
Il software di gestione del \gloxy{progetto} scelto dal gruppo è \textbf{Redmine}, che fornisce i seguenti servizi:
\begin{itemize}
\item Traccia il \gloxy{diagramma di Gantt} delle attività;
\item Calendario per l'organizzazione di attività e compiti;
\item Possibilità di associare i \gloxy{repository} e quindi visualizzarne la struttura ed i contenuti direttamente dall'interfaccia \gloxy{web};
\item Interfaccia \gloxy{web} semplice e curata;
\item Gestione dei ticket personalizzabile e molto flessibile;
\item Possibilità di osservare la differenza tra il tempo stimato e quello effettivo per il completamento di una attività.
\end{itemize}
Le piattaforme alternative prese in visione sono:
\begin{itemize}
\item \textbf{YouTrack};
\item \textbf{Gitorious}.
\end{itemize}
Entrambe però si sono mostrate meno complete, personalizzabili e curate di \textbf{Redmine}.
\subsubsubsection{Microsoft Project}
Come software di supporto alla pianificazione del \gloxy{progetto} si è scelto di utilizzare Microsoft Project 2010, questo perché in grado di fornire tutte le funzionalità necessarie in modo intuitivo e completo. Per l'utilizzo di Microsoft Project è possibile utilizzare la licenza gratuita offerta agli studenti tramite l'Università degli Studi di Padova\footnote{\url{http://msdnaa.studenti.math.unipd.it/2011/}.}.\\
Alternativamente a Microsoft Project è possibile utilizzare ProjectLibre, software che offre funzionalità analoghe ma \gloxy{multipiattaforma}. La scelta del software da utilizzare è a discrezione del \rRP, dal momento che è possibile convertire un \gloxy{progetto} di Microsoft Project ad un \gloxy{progetto} di ProjectLibre e viceversa mediante l'utilizzo di file \gloxy{XML}.
\subsubsubsection{Jenkins}
Il software utilizzato per l'integrazione continua è \textbf{Jenkins CI}.
Configurato per allacciarsi ai \gloxy{repository} del \gloxy{progetto}, consente di visualizzare lo stato del codice prodotto, pianificare compilazioni ed eseguire script per i test. Nel caso specifico di questo \gloxy{progetto}, è stato configurato per eseguire la compilazione dei documenti \LaTeX  ~e nelle fasi successive verrà aggiunta la possibilità di effettuare test specifici sul codice prodotto.
